% !TEX root = ../zeth-protocol-specification.tex

\chapter{Extended discussion on the security of $\mimc{}$ in different settings}\label{appendix:mimc-security}

In the original design proposed in the $\mimc{}$ paper~\cite{albrecht2016mimc}, the round function is represented as a ``shifted'' permutation via a cubic map (i.e.~the round input message is added to the key and round constant - the ``shift'' -, and a map $x^{\exponent{}}$ permutes the element in the underlying field $\FFx{2^n}$, $n \in \NN$, where $\gcd(\exponent{}, 2^n - 1) = 1$). This function (which is a permutation and therefore invertible) acts as a substitution box (S-box) and brings non-linearity to the scheme, as usually required for security.

In other sections of the paper, however, the $\mimc{}$ authors proposed generalizations to the initial design. These allow $\mimc{}$ to be used:
\begin{itemize}
    \item over prime fields of odd characteristic (i.e.~$\FFx{p}$, $p$ odd prime),
    \item with different permutation polynomials (i.e.~using different exponents in the round function)
\end{itemize}

Understanding the relationships between these various settings is required in order to use $\mimc{}$ to operate over prime fields of odd characteristic with a non-cubic permutation monomial.

Overall, for $\mimc{}$ to be considered secure, it is important that no attack in the literature (that may provide a significant speedup compared to ``exhaustive key search'') can succesfully be mounted by a $\ppt$ adversary. Two main families of attacks exist: statistical attacks and algebraic attacks.

\paragraph{Statistical attacks.}

In the ``Security Analysis'' section~\cite[Section 4.2]{albrecht2016mimc}, the authors explain that since the cubic function is an Almost Perfect Nonlinear map (\apn)\footnote{A function $f(x) = x^{\exponent{}}$ over $\FFx{p^n}$ is said to be differentially $k$-uniform if $k$ is the maximum number of solutions $x \in \FFx{p^n}$ of $f(x + d) - f (x) = D$ where $d, D \in \FFx{p^n}$ and $d \neq 0$. If $f$ is a 2-uniform mapping, we say that $f$ is almost perfect nonlinear. See e.g.~\cite{DBLP:journals/tit/HellesethRS99} for more information.}, linear attacks pose no threat to $\mimc{}$.

We observe that this claim aligns with~\cite[Theorem 2]{DBLP:journals/tit/HellesethRS99}. In fact if $\mimc{}$ is operated over $\FFx{2^n}$, it is easy to see that the degree of the cubic map can be expressed as $3 = 2^t + 1$, where $t = 1$, and where $n$ and $1$ are trivially coprime. This case is covered by~\cite[Theorem 2]{DBLP:journals/tit/HellesethRS99} which confirms that the cubic function $S(x) = x^3$ over $\FFx{2^n}$ is an APN/2-uniform mapping, as desired for differential and linear cryptanalysis resistance.

Likewise, in~\cite[Section 5.1]{albrecht2016mimc} the authors claim that, provided that the cubic map is a permutation over the prime field of interest, $\mimc{}$ can be used to operate over prime fields of odd characteristic. In this case too, $S(x) = x^3$ is an APN, provided $p \neq 3$ (as reported by~\cite[Theorem 3, item 3]{DBLP:journals/tit/HellesethRS99}).

In~\cite[Section 5.3]{albrecht2016mimc} the choice of the map degree is relaxed to be of the general form $2^t \pm 1$. Unfortunately, the authors showed that the case $\exponent{} = 2^t + 1$ is not as good as it initially seems in $\FFx{2^n}$, due to term cancellation in fields of characteristic 2 that renders the resulting polynomial sparse\footnote{Since the round function is a polynomial, the whole scheme can be seen as a polynomial with overall degree and ``sparsity'' that depends on the underlying field characteristic, degree of the round function and number of rounds.}. More precisely the degree of the polynomial will be bounded by $3^r$, $r$ being the number of rounds, which does not constitute an improvement on the case of $\exponent{} = 3$. For this very reason, exponents of the form $2^t + 1, t > 1$, may not be of interest (sparse polynomial and more expensive arithmetic in the round function).

Likewise, if the map degree $\exponent{}$ is chosen to be of the form $2^t + 1$, with $\gcd(\exponent{}, 2^n - 1) = 1$ in the context of $\mimc{}$ over $\FFx{2^n}$, it is necessary to bear in mind that, without the extra requirement that $t$ needs to be coprime with $n$, then this case is not covered by~\cite[Theorem 2]{DBLP:journals/tit/HellesethRS99}, and $S(x) = x^\exponent{}$ does not have differential 2-uniformity anymore - violating the claim made in~\cite[Section 4.2]{albrecht2016mimc} paragraph ``Linear Attacks'' about optimal resistance against linear and differential cryptanalysis. (In fact, depending on the value of $g = \gcd(n, t)$, the map $S(x) = x^{2^t + 1}$ would be differentially $2^g$-uniform~\cite{DBLP:conf/eurocrypt/Nyberg93} - contrasting with the setting considered in paragraph ``Linear attacks'' where $\exponent{} = 3$).
The case $\exponent{} = 2^t - 1$ does not yield an APN over $\FFx{2^n}$ either (except in the case $t = 2$ which reduces to the case $2^{t'} + 1$ for $t' = 1$). Similar observations show that picking round function degrees of the form $2^t \pm 1$ in the context where $\mimc{}$ is defined over prime fields $\FFx{p}$, $p$ odd prime, does not yield APNs.

\medskip

Overall, when studying the resistance of $\mimc{}$ against statistical attacks, we are interested in the probability that an input difference ($d$) is mapped into an output difference ($D$). That is, we are studying the probability of the following event:
\[
    F(x + d) - F(x) = D
\]
, where $F$ is a function that may either represent a single round, a set of rounds, or the full cipher.

Over a single round of $\mimc{}$ (i.e.~$F(x) = S(x)$), this probability is bounded by $(\exponent{}-1)/p$ \emph{provided that the exponent $\exponent{}$ is ``small''} (i.e.~small compared to the size of the field).
By assuming that the different rounds of the scheme are independent, the probability that an input difference gets mapped to an output difference, when $F$ represents the full cipher, becomes bounded as $\Pr[F(x + d) - F(x) = D] \leq ((\exponent{}-1)/p)^\rounds$.

For security, we want $((\exponent{}-1)/p)^\rounds$ to be bounded by $2^{-\secpar}$, where $\secpar$ is the security level (e.g.~128). Hence, we need
\[
    \left(\frac{\exponent{}-1}{p}\right)^\rounds \leq 2^{-\secpar}
\]
that is, we want
\[
    \rounds \geq \frac{\secpar}{\log_2(\frac{p}{\exponent{}-1})}
\]

As such, if the exponent is much smaller than the size of the field, few rounds are sufficient to prevent the differential attacks.

\paragraph{Algebraic attacks.}

While permutation monomials of degree $\exponent{} = 2^t \pm 1$ may not constitute APNs in the various $\mimc{}$ settings, it is important to note (as highlighted by Grassi in~\cite{mimc-security-communications}) that when working over finite fields $\FFx{p}$ of \emph{large} prime characteristic $p$ or extension fields $\FFx{2^n}$ of large extension degrees $n$, the algebraic attacks (exploiting the low-degree of the cipher) are much more efficient than the statistical attacks (i.e.~they can break a much higher number of rounds).

In fact, when considering security against algebraic attacks, we want (roughly speaking) the polynomial that defines the cipher to be of maximum degree and full (or at least, dense). That is we want the degree of the polynomial to be higher than $2^{\secpar}$. Since in $\mimc{}$ the S-box is defined as $S(x) = x^{\exponent{}}$, then after $\rounds$ rounds the degree of the polynomial describing the cipher is $\exponent{}^{\rounds}$. Hence, we need

\[
    \exponent{}^{\rounds} \geq 2^{\secpar}
\]
that is, we want
\[
    \rounds \geq \secpar \log_{\exponent}(2)
\]

\begin{remark}
    More rounds may be required as advised in~\cite{DBLP:conf/asiacrypt/Eichlseder0LORS20} to prevent some algebraic attacks that can be mounted when $\mimc{}$ is used over binary fields $\FFx{2^n}$.
\end{remark}

It is important to note that the security analysis related to algebraic attacks relies on the fact that the polynomial describing the cipher is dense/full. If this assumption is violated, a more granular security analysis needs to be carried out for the setting of interest.

\begin{notebox}
    For \emph{small exponents and large prime fields} (e.g.~for $\secpar = 128$, $p = 2^{128}$ and $\exponent{} = 3$), we see that the lower bound on the number of rounds is (much) smaller in the context of statistical attacks than in the context of algebraic attacks. As such, we see that in such settings algebraic attacks are much more powerful than statistical attacks. Hence, when instantiating $\mimc{}$ with a \emph{small} exponent of the form $2^t \pm 1$, it is crucial to make sure that, even if the resulting map is not an APN, the polynomial describing the cipher remains full/dense. Importantly, if the setting is changed (e.g.~to use exponents that are ``big'' w.r.t.~the field size) the security analysis proposed in~\cite{albrecht2016mimc} must be changed.
\end{notebox}
