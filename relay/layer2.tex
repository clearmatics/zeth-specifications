% !TeX spellcheck = en_GB
\documentclass[11pt,a4paper,oneside]{report}

%% Define hyphen required by macros.tex
\mathchardef\mhyphen="2D

% Import the specifications configuration
% !TEX root = ../zeth-protocol-specification.tex

\usepackage[utf8]{inputenc}
\usepackage[english]{babel}

% Display line numbers
\usepackage[edtable,longtable]{lineno}
\linenumbers%
% Manage document versioning
\usepackage{vhistory}
% For pdf annotations
% See: https://tex.stackexchange.com/questions/6306/how-to-annotate-pdf-files-generated-by-pdflatex
\usepackage{pdfcomment}
% Marking things to do
\usepackage[color=yellow!20]{todonotes}
% Highlight table rows/columns with color
\usepackage{color, colortbl}
% Use subfiles and make the project modular
\usepackage{subfiles}
% Typesetting theorems
\usepackage{subcaption}
%for subfigures
\usepackage{amsthm}
% AMS mathematical facilities
\usepackage{amsmath}
% An extended set of fonts for use in mathematics
\usepackage{amsfonts}
% Useful math symbols
% See: http://milde.users.sourceforge.net/LUCR/Math/mathpackages/amssymb-symbols.pdf
\usepackage{amssymb}
% St Mary Road symbols for theoretical computer science
\usepackage{stmaryrd}
% Customising captions in floating environments
\usepackage[justification=centering]{caption}
% Deal with space insertion and "eaten" spaces when defining commands
\usepackage{xspace}
% Split long sequences of letters or numbers
\usepackage{seqsplit}
% Verbatim with URL-sensitive line breaks
\usepackage{url}
% Improves the interface for defining floating objects such as figures and tables.
\usepackage{float}
% Key-value interface for optional arguments to the \includegraphics command
\usepackage{graphicx}
% Handle cross-referencing commands to produce hypertext links in the document
\usepackage{hyperref}
% Intelligent cross-referencing
\usepackage[capitalise]{cleveref}
% Glossary management (to include after hyperref to have hyperlinks)
\usepackage[acronym]{glossaries}
% Avoid widow lines
% see: https://www.techrepublic.com/blog/web-designer/how-to-fix-your-rags-widows-and-orphans/
\usepackage[all]{nowidow}
% Framed environments that can split at page boundaries
\usepackage[innerleftmargin=5pt,innerrightmargin=5pt]{mdframed}
% Interface to document dimensions
\usepackage[a4paper,centering]{geometry}
% Hyphenation for letterspacing, underlining, and more
\usepackage{soul}
% Enhances the quality of tables
\usepackage{booktabs}
% Quoting environments
\usepackage[font=itshape]{quoting}
% Nice looking colored, rounded-corners box
\usepackage{tcolorbox}
\newtcolorbox{notebox}{colback=red!5!white,colframe=red!75!black,fonttitle=\bfseries,title=Note}
\newtcolorbox{todobox}{colback=yellow!5!white,colframe=blue!75!black,fonttitle=\bfseries,title=TODO}
% Create various type of PDF comments
\usepackage{pdfcomment}
%for proper scaling abs values in multilines
\usepackage{mathtools}
% and for that purpose we introduce an abs* command
  \newcommand\MTkillspecial[1]{% helper macro
  \bgroup
  \catcode`\&=9
  \let\\\relax%
  \scantokens{#1}%
  \egroup
  }
  \DeclarePairedDelimiter\abs\lvert\rvert
  \reDeclarePairedDelimiterInnerWrapper\abs{star}{
  \mathopen{#1\vphantom{\MTkillspecial{#2}}\kern-\nulldelimiterspace\right.}
  #2
  \mathclose{\left.\kern-\nulldelimiterspace\vphantom{\MTkillspecial{#2}}#3}}

\usepackage{paralist} % for compact lists
\usepackage{multirow} % for multirow / multicolumn table entries

% Main package for cryptography commands
\usepackage[advantage, asymptotics, adversary, complexity, sets, keys, ff, notions, lambda, primitives, events, operators, probability, logic, mm, landau]{cryptocode}

%%%%% Projects %%%%%
\newcommand{\projectstyle}[1]{\texttt{#1}}
\newcommand{\zeth}{\projectstyle{Zeth}}
\newcommand{\zcash}{\projectstyle{Zcash}}
\newcommand{\zerocash}{\projectstyle{ZeroCash}}
\newcommand{\ethereum}{\projectstyle{Ethereum}}

%%%%% Relation/Circuit of Zeth %%%%%
\newcommand{\REL}{\mathbf{R}}
\newcommand{\RELCIRC}{\REL^{\projectstyle{z}}} % Zeth relation
\newcommand{\RELMAL}{\REL^{\projectstyle{mal}}} % Malleability Relation
\newcommand{\LANG}{\mathbf{L}} % language corresponding to relation \REL
\newcommand{\RELGEN}{\mathcal{R}} % relation generator

%%%%% Units %%%%%
\newcommand{\unitstyle}[1]{\mathsf{#1}}
\newcommand{\ether}{\ensuremath{\unitstyle{Ether}}\xspace}
\newcommand{\wei}{\ensuremath{\unitstyle{Wei}}\xspace}

%%%%% Parties %%%%%
\newcommand{\partystyle}[1]{\mathcal{#1}}
\newcommand{\partyp}{\ensuremath{\partystyle{P}}} % General notation for a party P
\newcommand{\challenger}{\ensuremath{\partystyle{C}}} % General notation for a challenger C
\newcommand{\zparty}[1]{\ensuremath{\partystyle{{#1}_{Z}}}} % Notation for a Zeth user (owner of a DAP address)
\newcommand{\eparty}[1]{\ensuremath{\partystyle{{#1}_{E}}}} % Notation for an Ethereum user (owner of an Ethereum address)

%%%%% Ethereum accounts %%%%%
\newcommand{\accountstyle}[1]{\mathbf{#1}}
%% Smart-contracts accounts
\newcommand{\contractstyle}[1]{\widetilde{\accountstyle{#1}}}
\newcommand{\mixer}{\ensuremath{\contractstyle{Mixer}}} % Notation for the Mixer contract in Zeth

%%%%% Math
\let\emptyset\varnothing% amssymb
\let\implies\Rightarrow%
\newcommand{\bytestyle}[1]{\texttt{0x#1}} % bytes style
\newcommand{\nuppt}{\pcmachinemodelstyle{NUPPT}} % Non-Uniform PPT
\newcommand{\suchthat}{\ensuremath{\text{s.t.}}\xspace} % Syntactic sugar for "such that"
\newcommand{\BB}{\mathbb{B}} % Set of bits {0, 1}
\newcommand{\BBy}[1]{\ensuremath{\mathbb{B}\ifthenelse{\equal{#1}{}}{}{^{#1}}_\mathbb{Y}}\xspace} % Set of bytes
\newcommand{\smallset}[1]{\left\{{#1}\right\}}
\newcommand{\brak}[1]{\left(#1 \right)}
\newcommand{\indexedset}[2]{\left\{{#1}\right\}_{#2}}
\newcommand{\range}[2]{\{{#1},\ldots,{#2}\}}
\newcommand{\cardinality}[1]{\#{#1}}
\newcommand{\smalltuple}[1]{({#1})}
\newcommand{\FFx}[1]{\ensuremath{\FF_{#1}}\xspace} % Finite field of {#1} elements
\newcommand{\gset}{\ensuremath{\mathbb{G}}\xspace} % Notation for a group
\newcommand{\gel}[1]{\ensuremath{\mathfrak{#1}}\xspace} %group element (without brackets)
\newcommand{\ggen}{\ensuremath{\mathfrak{g}}\xspace} % Generator of group
\newcommand{\groupenc}[1]{\ensuremath{\llbracket {#1} \rrbracket}\xspace} % Group encoding
\newcommand{\groupenci}[2]{\ensuremath{\left\llbracket {#1} \right\rrbracket_{#2}}\xspace} % Group encoding in group i
\newcommand{\pair}{\bullet} % bilinear pairing operator (inline)
\newcommand{\keyspace}{\ensuremath{\mathcal{K}}\xspace} % Space of keys
\newcommand{\weakSet}{\ensuremath{\mathcal{W}}\xspace} % Space of weak values
\newcommand{\strongSet}{\ensuremath{\mathcal{S}}\xspace} % Space of strong values
\newcommand{\primP}{\ensuremath{\mathcal{P}}\xspace} % Primitive
\newcommand{\primR}{\ensuremath{\mathcal{R}}\xspace} % Primitive
\newcommand{\primC}{\ensuremath{\mathcal{C}}\xspace} % Primitive

%%%%% Variables
\newcommand{\varstyle}[1]{\mathit{#1}}
\renewcommand{\pckeystyle}[1]{\ensuremath{\varstyle{#1}}} % change cryptocode key style to match with our
\newcommand{\iv}{\ensuremath{\varstyle{IV}}\xspace} % Notation for an initiation vector
\newcommand{\tablevar}{\ensuremath{\varstyle{Table}}\xspace} % Notation for a (lookup) table
\newcommand{\digest}{\ensuremath{\varstyle{digest}}\xspace} % Notation for a digest variable
\newcommand{\ek}{\ensuremath{\varstyle{ek}}\xspace} % Symmetric Encryption key
\newcommand{\mk}{\ensuremath{\varstyle{mk}}\xspace} % Mac key
\newcommand{\sharedSecret}{\ensuremath{\varstyle{ss}}\xspace} % Shared Secret
\newcommand{\epk}{\ensuremath{\varstyle{epk}}\xspace} % Notation for an ephemeral public key
\newcommand{\esk}{\ensuremath{\varstyle{esk}}\xspace} % Notation for an ephemeral secret key
\newcommand{\msg}{\ensuremath{\varstyle{m}}\xspace} % Notation for a message
\DeclareRobustCommand{\cm}[1]{\ensuremath{\varstyle{cm}\ifthenelse{\equal{#1}{}}{}{_{#1}}}\xspace} % Notation for commitment
\newcommand{\ct}{\ensuremath{\varstyle{ct}}\xspace} % Notation for ciphertext
\newcommand{\queryBound}{\ensuremath{\varstyle{q}}\xspace} % Notation for the number of queries
\newcommand{\timeBound}{\ensuremath{\varstyle{t}}\xspace} % Notation for the time steps
\newcommand{\pparams}{\ensuremath{\varstyle{pp}}\xspace} % Notation for public parameters
\newcommand{\ledger}{\ensuremath{\varstyle{L}}\xspace} % Notation for a Ledger (Ethereum here)
\newcommand{\keystore}{\ensuremath{\varstyle{keystore}}\xspace} % Notation for a keystore, i.e. set of Zeth payment addresses
\newcommand{\randomSeed}{\ensuremath{\varstyle{randomSeed}}\xspace} % randomSeed used in Zcash's fix of x malleability
\DeclareRobustCommand{\htag}[1]{\ensuremath{\varstyle{h}\ifthenelse{\equal{#1}{}}{}{_{#1}}}\xspace} % Notation for MAC for non-malleability
\newcommand{\zkp}{\ensuremath{\varstyle{\pi}}\xspace} % Notation for a zk-snark
\newcommand{\zethnote}{\ensuremath{\varstyle{ZethNote}}\xspace} % Notation for a zethNote object
\newcommand{\zethnotes}{\ensuremath{\varstyle{ZethNotes}}\xspace} % Notation for multiple zethNote objects
\newcommand{\note}{\ensuremath{\varstyle{Note}}\xspace} % Notation for a note, used in DAP definition
\newcommand{\notes}{\ensuremath{\varstyle{Notes}}\xspace} % Notation for a note, used in DAP definition
\newcommand{\datatobesigned}{\ensuremath{\varstyle{dataToBeSigned}}\xspace} % Notation for the data to be signed for TRNM
\newcommand{\rootset}{\ensuremath{\varstyle{Roots}}\xspace} % Notation for the set of Merkle roots corresponding to the various states of the Merkle tree
\newcommand{\nullifierset}{\ensuremath{\varstyle{Nulls}}\xspace} % Notation for the set of Nullifiers declared to the Mixer
%
\newcommand{\otsSigma}{\ensuremath{\varstyle{\sigma_{\otsig}}}\xspace} % Notation for one-time signature
\newcommand{\ecdsaSigma}{\ensuremath{\varstyle{\sigma_{\ecdsa}}}\xspace} % Notation for ECDSA signature
%
\DeclareRobustCommand{\blakeState}[1]{\ensuremath{\varstyle{v}\ifthenelse{\equal{#1}{}}{}{[#1]}}\xspace} % Notation for Blake state variable
%
\newcommand{\priminputs}{\ensuremath{\varstyle{prim}}\xspace} % Notation for primary inputs
\newcommand{\auxinputs}{\ensuremath{\varstyle{aux}}\xspace} % Notation for auxiliary inputs
\newcommand{\inputs}{\ensuremath{\varstyle{inp}}\xspace} % Notation for auxiliary or primary inputs
%
\newcommand{\addr}{\ensuremath{\varstyle{Addr}}\xspace} % Notation for address variable
%
\newcommand{\wstate}{\ensuremath{\varstyle{\varsigma}}\xspace} % Notation for the Ethereum world state
%
\newcommand{\tx}{\ensuremath{\varstyle{tx}}\xspace} % Notation for Ethereum transaction
\newcommand{\zethTx}{\ensuremath{\varstyle{\tx_\mix}}\xspace} % Notation for Ethereum transaction using Zeth
\newcommand{\rawTx}{\ensuremath{\varstyle{\tx_{raw}}}\xspace} % Notation for raw Ethereum transaction
\newcommand{\finalTx}{\ensuremath{\varstyle{\tx_{final}}}\xspace} % Notation for finalized Ethereum transaction
\newcommand{\zdata}{\ensuremath{\varstyle{zdata}}\xspace} % Notation for data field of a zethTx
%
\newcommand{\evMixOut}{\ensuremath{\varstyle{evMixOut}}\xspace} % Notation for mix output event
%
\newcommand{\rounds}{\ensuremath{\varstyle{rounds}}\xspace} % Notation for Mimc number of rounds
\newcommand{\exponent}{\ensuremath{\varstyle{e}}} % Notation for Mimc exponent
\newcommand{\mults}{\ensuremath{\varstyle{mults}}} % Notation for the number of multiplications
%
\newcommand{\encTag}{\ensuremath{\constantstyle{encTag}}\xspace} % Encryption tag
% Function inputs (Add "in" subscript for function inputs to avoid naming collisions)
% Use: \inp{x}, or \inp{\functionName}
\newcommand{\inp}[1]{\ensuremath{\varstyle{{#1}_{in}}}\xspace}
\newcommand{\tagged}[2]{\ensuremath{\varstyle{tag_{#1}^{#2}}}\xspace} % Notation for tagged variable use for domain separation
\newcommand{\taggedaddr}{\ensuremath{\tagged{\ask}{addr}}\xspace} % Notation for the tag input of prf^addr
\newcommand{\taggedpk}{\ensuremath{\tagged{\ask,i}{pk}}\xspace} % Notation for the tag input of prf^pk
\newcommand{\taggednf}{\ensuremath{\tagged{\ask}{nf}}\xspace} % Notation for the tag input of prf^nf
\newcommand{\taggedrho}{\ensuremath{\tagged{\ask,j}{\rho}}\xspace} % Notation for the tag input of prf^rho


%% Domain/codomain size
\newcommand{\msgLen}{\ensuremath{\varstyle{mLen}}\xspace} % Notation for the msg length
\newcommand{\symKeyLen}{\ensuremath{\varstyle{kLen}}} % Notation for symmetric cipher key bit length
\newcommand{\macKeyLen}{\ensuremath{\varstyle{mLen}}} % Notation for mac key bit length
\newcommand{\macTagLen}{\ensuremath{\varstyle{tLen}}} % Notation for mac tag bit length
\newcommand{\hashLen}{\ensuremath{\varstyle{hLen}}} % hash digest bit length
\newcommand{\hashInpLen}{\ensuremath{\varstyle{hInpLen}}} % hash input bit length
\newcommand{\groupLen}{\ensuremath{\varstyle{gLen}}} % group element bit length

%%%%% Algorithms and remarkable functions
\newcommand{\algostyle}[1]{\mathsf{#1}}
%% Literature algorithms
\newcommand{\groth}{\algostyle{Groth16}}
\newcommand{\sha}[1]{\ensuremath{\algostyle{SHA{#1}}}\xspace} % General macro for SHA algorithms, eg: \sha{256}, \sha{3}
\DeclareRobustCommand{\blake}[2]{\ensuremath{\algostyle{Blake{#1}}\ifthenelse{\equal{#2}{}}{}{(#2)}}\xspace} % General macro for BLAKE algorithms, eg: \blake{2s}, \blake{2b}
\newcommand{\blakeG}{\ensuremath{\algostyle{G}}\xspace} % Blake G function
\newcommand{\haifa}{\ensuremath{\algostyle{HAIFA}}\xspace} % HAIFA hash construct (used in BLAKE algorithms)
\newcommand{\mimc}[1]{\ensuremath{\algostyle{MIMC{#1}}}\xspace} % General macro for MiMC algorithms, eg: \mimc{256}
\newcommand{\keccak}[1]{\ensuremath{\algostyle{Keccak{#1}}}\xspace} % General macro for KECCAK algorithms, eg: \keccak{256}
\newcommand{\salsa}[1]{\ensuremath{\algostyle{Salsa#1}}\xspace} % Salsa cipher
\newcommand{\chacha}[1]{\ensuremath{\algostyle{ChaCha#1}}\xspace} % Chacha stream cipher
\newcommand{\chachaBlock}[1]{\ensuremath{\algostyle{\chacha{#1}Block}}\xspace} % Chacha block function
\newcommand{\polymac}[1]{\ensuremath{\algostyle{Poly#1}}\xspace} % Poly mac
\newcommand{\aes}{\ensuremath{\algostyle{AES}}\xspace} % Aes cipher
\newcommand{\ecdsa}{\ensuremath{\algostyle{ECDSA}}\xspace}
\newcommand{\otsig}{\ensuremath{\algostyle{OT\mhyphen{}SIG}}\xspace} % One-time signature algorithm
\DeclareRobustCommand{\mimcPrime}[1][]{\ensuremath{{\mimc{}}\ifthenelse{\equal{#1}{}}{}{_{#1}}}\xspace} % Mimc block cipher mod r
\newcommand{\mimcEnc}{\ensuremath{\algostyle{\mimc{}\pcmathhyphen Encrypt}}\xspace} % Mimc block cipher encrypt
\newcommand{\mimcMP}{\ensuremath{\algostyle{\mimc{}\pcmathhyphen MP}}\xspace} % Mimc Miyaguchi-Preneel compression function
\DeclareRobustCommand{\mimcMPPrime}[1]{\ensuremath{\mimcMP\ifthenelse{\equal{#1}{}}{}{_{#1}}}\xspace} % Mimc Miyaguchi-Preneel compression function mod r
\newcommand{\mimcSeven}{\ensuremath{\algostyle{\mimc[7]}}\xspace} % Mimc block cipher with (e=7)
\newcommand{\mimcSevenPrime}{\ensuremath{\mimcSeven_\rBN}\xspace} % Mimc block cipher with (e=7) mod r
\newcommand{\mimcSevenMP}{\ensuremath{\algostyle{\mimcSeven\pcmathhyphen\MP}}\xspace} % Mimc MP compression function with with (e=7)
\newcommand{\mimcSevenMPPrime}{\ensuremath{\mimcSevenMP_\rBN}\xspace} % Mimc MP compression function with with (e=7) mod r
%%% General algorithms
\newcommand{\xortxt}{\ensuremath{\algostyle{XOR}}\xspace} % XOR algorithm
%% Definitions and preliminaries
%%% Ethereum util algorithms
\newcommand{\getEthAddr}[1]{\ensuremath{\algostyle{EthAddress({#1})}}\xspace} % Takes an Ethereum account as input and returns its address
\newcommand{\funcSelec}[1]{\ensuremath{\algostyle{FS({#1})}}\xspace} % Takes a function on a smart contract, and returns its "function selector" (i.e. the function's address)
%%% Strings
\DeclareRobustCommand{\pad}[2]{\ensuremath{\algostyle{pad}\ifthenelse{\equal{#1}{}}{}{_{#2}(#1)}}\xspace} % Padding algorithm
\DeclareRobustCommand{\trunc}[2]{\ensuremath{\algostyle{trunc}\ifthenelse{\equal{#1}{}}{}{_{#2}(#1)}}\xspace} % Truncates a string, eg: \trunc{k}{b} takes the first k bits of string b
\DeclareRobustCommand{\slice}[3]{\ensuremath{\ifthenelse{\equal{#1}{}}{\algostyle{[:]}}{#1\algostyle{[}#2\algostyle{:}#3\algostyle{]}}}\xspace} % Returns x[a:b], from string x the substring starting at position a and finishing at position b
\DeclareRobustCommand{\len}[1]{\ensuremath{\algostyle{length}\ifthenelse{\equal{#1}{}}{}{(#1)}}\xspace} % Returns the number of symbols in a string, i.e. its length
\newcommand{\dropLeadingZeroes}{\ensuremath{\algostyle{dropLeadingZeroes}}\xspace} % Drops the leading '0' of a binary string
%%% Data types
\DeclareRobustCommand{\encode}[2]{\ensuremath{\algostyle{encode}\ifthenelse{\equal{#1}{}}{_{#2}}{_{#2}(#1)}}\xspace} % Returns binary encoding of object
\DeclareRobustCommand{\decode}[2]{\ensuremath{\algostyle{decode}\ifthenelse{\equal{#1}{}}{_{#2}}{_{#2}(#1)}}\xspace} % Returns object from binary encoding and data type
\DeclareRobustCommand{\append}[2]{\ensuremath{\algostyle{append}\ifthenelse{\equal{#1}{}}{}{(#1, #2)}}\xspace} % Append an element to a list of of elements with the same data type
%%% Block ciphers
\newcommand{\Enc}{\ensuremath{\algostyle{E}}\xspace} % Block cipher encryption algorithm
\newcommand{\Dec}{\ensuremath{\algostyle{E}^{-1}}\xspace} % Block cipher decryption algorithm
%%% Compression functions
\newcommand{\MP}{\ensuremath{\algostyle{MP}}\xspace} % Miyaguchi-Preneel construction
\newcommand{\fFunc}{\ensuremath{\algostyle{f}}\xspace} % Notation for a compression function f
\newcommand{\FEnc}{\ensuremath{\fFunc_{\Enc}}\xspace} % Compression function based on a given block cipher E
%% Zeth Spec algorithms
\newcommand{\fMP}{\ensuremath{\FEnc^{\MP}}\xspace} % MP compression function
\newcommand{\txgen}{\ensuremath{\algostyle{TxGen}}\xspace} % Algorithm that creates an Ethereum transaction
\newcommand{\txmalgen}{\ensuremath{\algostyle{TxMalGen}}\xspace} % Algorithm that creates a transaction that breaks TRNM
\newcommand{\zethVerifyTx}{\ensuremath{\algostyle{ZethVerifyTx}}\xspace} % Algorithm that verifies the validity of a \zethTx
\newcommand{\ethVerifyTx}{\ensuremath{\algostyle{EthVerifyTx}}\xspace} % Algorithm that verifies the validity of an Ethereum transaction
%%% CRHs
\newcommand{\crh}{\ensuremath{\algostyle{CRH}}\xspace} % Collision Resistant Hash function
\DeclareRobustCommand{\crhots}[1]{\ensuremath{\algostyle{\crh^{ots}}\ifthenelse{\equal{#1}{}}{}{(#1)}}\xspace} % CRH for One Time Signature Scheme
\DeclareRobustCommand{\crhhsig}[1]{\ensuremath{\algostyle{\crh^{hsig}}\ifthenelse{\equal{#1}{}}{}{(#1)}}\xspace} % CRH for Non Malleability
\newcommand{\hashSetup}{\ensuremath{\pcalgostyle{Setup}}\xspace} % hash setup function
%%% PRFs
\newcommand{\prfnf}[2]{\ensuremath{\algostyle{\prf}^\algostyle{nf}\ifthenelse{\equal{#2}{}}{}{_{#1}(#2)}}\xspace}
\newcommand{\prfrho}[2]{\ensuremath{\algostyle{\prf}^\algostyle{rho}\ifthenelse{\equal{#2}{}}{}{_{#1}(#2)}}\xspace}
\newcommand{\prfaddr}[2]{\ensuremath{\algostyle{\prf}^\algostyle{addr}\ifthenelse{\equal{#2}{}}{}{_{#1}(#2)}}\xspace}
\newcommand{\prfpk}[2]{\ensuremath{\algostyle{\prf}^\algostyle{pk}\ifthenelse{\equal{#2}{}}{}{_{#1}(#2)}}\xspace}
%%% State transitions
\newcommand{\mix}{\ensuremath{\algostyle{Mix}}\xspace} % Zeth main state transition function
%%% Algebraic groups
\newcommand{\groupSetup}{\ensuremath{\algostyle{SetupG}}\xspace} % group setup function
%%% Others
\newcommand{\setup}{\ensuremath{\algostyle{Setup}}\xspace} % Setup algorithm that outputs public parameters
\newcommand{\kdf}{\ensuremath{\algostyle{KDF}}\xspace} % Key Derivation function
\newcommand{\mkhash}{\ensuremath{\algostyle{MKHASH}}\xspace} % Hash function used in the Merkle tree
%%% Packing functions
\DeclareRobustCommand{\pack}[2]{\ensuremath{\algostyle{Pack}\ifthenelse{\equal{#1}{}}{}{_{#2}(#1)}}\xspace} % Returns field encoding of object
\DeclareRobustCommand{\packResBits}[1]{\ensuremath{\algostyle{Pack}_\resbits\ifthenelse{\equal{#1}{}}{}{(#1)}}\xspace} % Returns field encoding of object
\DeclareRobustCommand{\unpack}[2]{\ensuremath{\algostyle{Unpack}\ifthenelse{\equal{#1}{}}{}{_{#2}(#1)}}\xspace} % Returns field encoding of object
%%%% Tagging functions
\newcommand{\tagfunction}{\ensuremath{\algostyle{tag}}\xspace} % Tagging function
\DeclareRobustCommand{\tagnf}[1]{\ensuremath{\algostyle{tag^{nf}}\ifthenelse{\equal{#1}{}}{}{(#1)}}\xspace} % Tagging function to make PRF^nf distribution independent
\DeclareRobustCommand{\tagrho}[2]{\ensuremath{\algostyle{tag^{rho}}\ifthenelse{\equal{#1}{}}{}{(#2, #1)}}\xspace} % Tagging function to make PRF^rho distribution independent
\DeclareRobustCommand{\tagaddr}[1]{\ensuremath{\algostyle{tag^{addr}}\ifthenelse{\equal{#1}{}}{}{(#1)}}\xspace} % Tagging function to make PRF^addr distribution independent
\DeclareRobustCommand{\tagpk}[2]{\ensuremath{\algostyle{tag^{pk}}\ifthenelse{\equal{#1}{}}{}{(#2, #1)}}\xspace} % Tagging function to make PRF^pk distribution independent
%%%%% Schemes (set of algorithms) %%%%%
\newcommand{\schemestyle}[1]{\mathsf{#1}}
%% Signature schemes
\newcommand{\sigscheme}{\ensuremath{\schemestyle{SigSch}}\xspace}
\newcommand{\otsigscheme}{\ensuremath{\schemestyle{\sigscheme_{\otsig}}}\xspace} % Notation for One Time signature scheme
\newcommand{\ecdsasigscheme}{\ensuremath{\schemestyle{\sigscheme_{\ecdsa}}}\xspace} % Notation for ECDSA signature scheme
%% Encryption scheme
\newcommand{\encscheme}{\ensuremath{\schemestyle{EncSch}}\xspace}
\newcommand{\sym}{\ensuremath{\schemestyle{Sym}}\xspace} % Symmetric cipher
\newcommand{\aSym}{\ensuremath{\schemestyle{Asym}}\xspace} % Asymmetric cipher
\newcommand{\dhaes}{\ensuremath{\schemestyle{DHAES}}\xspace} % dhaes encryption scheme
\newcommand{\dhies}{\ensuremath{\schemestyle{DHIES}}\xspace} % dhies encryption scheme
%% Commitment scheme
\newcommand{\comm}{\ensuremath{\algostyle{ComSch}}\xspace}
\DeclareRobustCommand{\commit}[2]{\ensuremath{\algostyle{Com}\ifthenelse{\equal{#2}{}}{}{(#1 ; \allowbreak #2)}}\xspace} % commit function of a commitment scheme
\newcommand{\opening}{\ensuremath{\varstyle{r}}\xspace} % commitment opening/random value
%% Message authentication code
\newcommand{\tagg}{\ensuremath{\algostyle{Tag}}\xspace} % Tag generation algorithm
%% DAP scheme
\newcommand{\dapscheme}{\ensuremath{\schemestyle{DAP}}\xspace}
%\newcommand{\setup}{\ensuremath{\algostyle{Setup}}\xspace} % Outputs public parameters -- macro commented since already defined
\newcommand{\genadd}{\ensuremath{\algostyle{GenAddr}}\xspace} % Generates DAP address
\newcommand{\sendtx}{\ensuremath{\algostyle{SendTx}}\xspace} % Generates DAP transaction and sends in on the network
\newcommand{\receive}{\ensuremath{\algostyle{Receive}}\xspace} % Receive DAP transaction
\newcommand{\verifytx}{\ensuremath{\algostyle{VerifyTx}}\xspace} % Verify DAP transaction validity
%% zk-SNARK scheme
\newcommand{\zksnark}{\ensuremath{\schemestyle{ZkSnarkSch}}\xspace}

%%%%% Curves %%%%%
\newcommand{\curvestyle}[1]{\mathsf{#1}}
\newcommand{\secpCurve}{\ensuremath{\curvestyle{secp256k1}}\xspace}
\newcommand{\curve}[1]{\ensuremath{\curvestyle{Curve#1}}\xspace} % general curve format macro
\newcommand{\xscalarmult}[1]{\ensuremath{\algostyle{X#1}}\xspace} % Scalar multiplication on Curve25519

\newcommand{\jubjub}{\ensuremath{\curvestyle{Jubjub}}\xspace} % jubjub curve

%%%%% Data types %%%%%
\newcommand{\datatypestyle}[1]{{\mathtt{#1}}}
\newcommand{\datatype}{\ensuremath{\datatypestyle{DType}}\xspace} % Generic data type
%% Ethereum account data type
\newcommand{\nonce}{\ensuremath{\varstyle{nce}}\xspace} % Account nonce
\newcommand{\balance}{\ensuremath{\varstyle{bal}}\xspace} % Balance of the account
\newcommand{\sroot}{\ensuremath{\varstyle{sRoot}}\xspace} % Storage root
\newcommand{\codeh}{\ensuremath{\varstyle{codeh}}\xspace} % Code hash
%% Zeth address data type
\newcommand{\zaddrDType}{\ensuremath{\datatypestyle{ZAddrDType}}\xspace}
\newcommand{\pubaddr}{\ensuremath{\varstyle{pub}}\xspace} % Payment (public) address
\newcommand{\privaddr}{\ensuremath{\varstyle{priv}}\xspace} % Private address
%% Zeth payment address data type
\newcommand{\apk}{\ensuremath{\varstyle{apk}}\xspace} % Paying key
\newcommand{\pkenc}{\ensuremath{\varstyle{pkenc}}\xspace} % Transmission key
%% Zeth private address data type
\newcommand{\ask}{\ensuremath{\varstyle{ask}}\xspace} % Spending key
\newcommand{\skenc}{\ensuremath{\varstyle{skenc}}\xspace} % Receiving key
%% Zeth note data type
\newcommand{\zethNoteDType}{\ensuremath{\datatypestyle{ZethNoteDType}}\xspace}
%\newcommand{\apk}{\ensuremath{\varstyle{apk}}\xspace} % Note owner's paying key -- macro commented since already defined
\newcommand{\noter}{\ensuremath{\varstyle{r}}\xspace} % Note commitment trapdoor
\newcommand{\notev}{\ensuremath{\varstyle{v}}\xspace} % Note value
\newcommand{\rrho}{\ensuremath{\varstyle{\rho}}\xspace} % Note identifier
%% JSInput data type
\newcommand{\jsInputDType}{\ensuremath{\datatypestyle{JSInputDType}}\xspace}
\newcommand{\mkpath}{\ensuremath{\varstyle{mkpath}}\xspace} % Merkle authentication path for note commitment
\newcommand{\mkaddr}{\ensuremath{\varstyle{mkaddr}}\xspace} % Merkle address for note commitment
\newcommand{\znote}{\ensuremath{\varstyle{znote}}\xspace} % Zeth note to spend
%\newcommand{\ask}{\ensuremath{\varstyle{ask}}\xspace} % Note owner's spending key -- macro commented since already defined
\DeclareRobustCommand{\nf}[1]{\ensuremath{\varstyle{nf}\ifthenelse{\equal{#1}{}}{}{_{\!#1}}}\xspace} % Zeth note nullifier
%% Primary input data type
\newcommand{\primInputDType}{\ensuremath{\datatypestyle{PrimInputDType}}\xspace}
\newcommand{\mkroot}{\ensuremath{\varstyle{mkroot}}\xspace} % Merkle root
\DeclareRobustCommand{\nfs}[1]{\ensuremath{\varstyle{nfs}\ifthenelse{\equal{#1}{}}{}{[#1]}}\xspace} % List of nullifiers (notes spent)
\DeclareRobustCommand{\cms}[1]{\ensuremath{\varstyle{cms}\ifthenelse{\equal{#1}{}}{}{[#1]}}\xspace} % List of commitments (new notes)
\newcommand{\vin}{\ensuremath{\varstyle{vin}}\xspace} % Public value to add to the mix
\newcommand{\vout}{\ensuremath{\varstyle{vout}}\xspace} % Public value to remove from the mix
\newcommand{\hsig}{\ensuremath{\varstyle{hsig}}\xspace} % Signature hash
\DeclareRobustCommand{\htags}[1]{\ensuremath{\varstyle{htags}\ifthenelse{\equal{#1}{}}{}{[#1]}}\xspace} % List of notes' message authentication tags
%% Packed primary input data type
\newcommand{\resbits}{\ensuremath{\varstyle{rsd}}\xspace} % List of residual bits
%% Auxiliary input data type
\newcommand{\auxInputDType}{\ensuremath{\datatypestyle{AuxInputDType}}\xspace}
\DeclareRobustCommand{\znotes}[1]{\ensuremath{\varstyle{znotes}\ifthenelse{\equal{#1}{}}{}{[#1]}}\xspace} % List of notes to create
\DeclareRobustCommand{\jsins}[1]{\ensuremath{\varstyle{jsins}\ifthenelse{\equal{#1}{}}{}{[#1]}}\xspace} % List of notes (and associated data) to spend
\newcommand{\pphi}{\ensuremath{\varstyle{\phi}}\xspace} % Joinsplit randomness
%% Mix input data type
\newcommand{\mixInputDType}{\ensuremath{\datatypestyle{MixInputDType}}\xspace}
%% Mix event data type
\newcommand{\mixEventDType}{\ensuremath{\datatypestyle{MixEventDType}}\xspace}
\newcommand{\primInp}{\ensuremath{\varstyle{primIn}}\xspace} % Primary inputs
\newcommand{\zkproof}{\ensuremath{\varstyle{proof}}\xspace} % zk-SNARK
\newcommand{\otssig}{\ensuremath{\varstyle{otssig}}\xspace} % One-time signature for TRNM
\newcommand{\otsvk}{\ensuremath{\varstyle{otsvk}}\xspace} % One-time signature vk for TRNM
\newcommand{\ciphers}{\ensuremath{\varstyle{ciphers}}\xspace} % Ciphertexts of the new notes
%% ZKP data type
\newcommand{\zkpDType}{\ensuremath{\datatypestyle{ZKPDType}}\xspace}
%% EPK data type
\newcommand{\epkDType}{\ensuremath{\datatypestyle{EPKDType}}\xspace}
%% OTS vk data type
\newcommand{\vkOtsDType}{\ensuremath{\datatypestyle{VKOtsDType}}\xspace}
%% OTS sk data type
\newcommand{\skOtsDType}{\ensuremath{\datatypestyle{SKOtsDType}}\xspace}
%% OTS signature data type
\newcommand{\sigOtsDType}{\ensuremath{\datatypestyle{SigOtsDType}}\xspace}
%% Transaction data types
\newcommand{\txRawDType}{\ensuremath{\datatypestyle{TxRawDType}}\xspace}
\newcommand{\txDType}{\ensuremath{\datatypestyle{TxDType}}\xspace}
%\newcommand{\nonce}{\ensuremath{\varstyle{nce}}\xspace} % Transaction nonce -- macro commented since already defined
\newcommand{\init}{\ensuremath{\varstyle{init}}\xspace} % Deployed code (new smart contract)
\newcommand{\data}{\ensuremath{\varstyle{data}}\xspace} % Transaction data
\newcommand{\sigv}{\ensuremath{\varstyle{v}}\xspace} % Transaction v ECDSA sig
\newcommand{\sigr}{\ensuremath{\varstyle{r}}\xspace} % transaction r ECDSA sig
\newcommand{\sigs}{\ensuremath{\varstyle{s}}\xspace} % Transaction s ECDSA sig
\newcommand{\gasp}{\ensuremath{\varstyle{gasP}}\xspace} % Transaction gas price
\newcommand{\gasl}{\ensuremath{\varstyle{gasL}}\xspace} % Transaction gas limit
\newcommand{\tto}{\ensuremath{\varstyle{to}}\xspace} % Transaction recipient address (macro `\to` already exists)
\newcommand{\val}{\ensuremath{\varstyle{val}}\xspace} % Transaction value

%% Others
\newcommand{\badvar}{\ensuremath{\texttt{bad}}\xspace} % Bad variable used in DHAES ik-cca proof

%%%%% Acronyms %%%%%
\newcommand{\poc}{\acrshort{poc}}
\newcommand{\evm}{\acrshort{evm}} % Ethereum Virtual Machine
\newcommand{\eoa}{\acrshort{eoa}} % Externally owned Account
\newcommand{\rlp}{\acrshort{rlp}} % Recursive Length Prefix
\newcommand{\mmac}{\acrshort{mac}} % Message Authentication Code
\newcommand{\dos}{\acrshort{dos}} % Denial of Service attack
\newcommand{\ram}{\acrshort{ram}} % Random-access memory
\newcommand{\fft}{\acrshort{fft}} % Fast Fourier Transform

%%%%% Security Assumptions and games %%%%%
\newcommand{\gamestyle}[1]{\mathsf{#1}}
\renewcommand{\pcnotionstyle}[1]{\ensuremath{\gamestyle{#1}}} % change cryptocode notion/game style to match with our
\newcommand{\dlog}{\ensuremath{\gamestyle{DLog}}\xspace} % Discrete Logarithm
\newcommand{\omdlog}{\ensuremath{\gamestyle{om\mhyphen{}\dlog}}\xspace} % One-more DLog
\newcommand{\ufcma}{\ensuremath{\gamestyle{UF\mhyphen{}CMA}}\xspace} % Unforgeability under CMA
\newcommand{\sufcma}{\ensuremath{\gamestyle{S\ufcma}}\xspace} % Strong UFCMA
\newcommand{\trnm}{\ensuremath{\gamestyle{TR\mhyphen{}NM}}\xspace} % TRansaction Non-Malleability
\newcommand{\colres}{\ensuremath{\gamestyle{CR}}\xspace} % Collision Resistance
\newcommand{\ikcca}{\ensuremath{\gamestyle{IK\mhyphen{}CCA}}\xspace} % Indistinguishability of Key under CCA
\newcommand{\aicca}{\ensuremath{\gamestyle{AI\mhyphen{}CCA}}\xspace} % Indistinguishability of Key and Ciphertext under CCA
\newcommand{\hdhi}{\ensuremath{\gamestyle{HDHI}}\xspace} % Hash diffie-helmann independence assumption
\newcommand{\hdhii}{\ensuremath{\gamestyle{HDHI2}}\xspace} % Two hash diffie-helmann independence
\newcommand{\odh}{\ensuremath{\gamestyle{ODH}}\xspace} % Oracle diffie-helmann assumption
\newcommand{\odhii}{\ensuremath{\gamestyle{ODH2}}\xspace} % Two oracle diffie-helmann assumption
\newcommand{\advColl}{\ensuremath{\advantage{\gamestyle{coll}}{\fFunc, \adv}[]}\xspace} % Advantage collision resistant compression function
\newcommand{\advCollMP}{\ensuremath{\advantage{\gamestyle{coll}}{\fMP, \adv}[]}\xspace} % advantage collision resistant MP compression function
\newcommand{\indiff}{\ensuremath{\gamestyle{Indiff}\xspace}}

%%%%% Oracles %%%%%
\newcommand{\oracle}[1]{\ensuremath{\algostyle{O}^{#1}}} % Oracle
\newcommand{\oracleSig}{\ensuremath{\oracle{\sig_\sk}}\xspace} % Signature Oracle
\newcommand{\oracleDLog}[1]{\ensuremath{\oracle{\dlog_\mathsf{#1}}}} % Dlog oracle
\newcommand{\oracleEnc}{\ensuremath{\oracle{\Enc}}\xspace} % Encryption Oracle
\newcommand{\oracleDec}{\ensuremath{\oracle{\Dec}}\xspace} % Decryption Oracle
\newcommand{\ro}{\mathcal{R}} %random oracle
\newcommand{\oracleHdhi}[1]{\ensuremath{\oracle{\hdhi_{#1}}}\xspace} % HDHI Oracle

%%%%% Function families %%%%%
\newcommand{\ffamstyle}[1]{\mathcal{#1}}
\newcommand{\funcSet}{\ensuremath{\ffamstyle{F}}\xspace} % General notation for a function family
\newcommand{\hashSet}{\ensuremath{\ffamstyle{H}}\xspace} % Family of hash functions
\newcommand{\blockSet}{\ensuremath{\ffamstyle{BLK}}\xspace} % Family of block ciphers
\newcommand{\crhSet}{\ensuremath{\ffamstyle{CRH}}\xspace} % Family of hash functions
\newcommand{\randSet}{\ensuremath{\ffamstyle{RAND}}\xspace} % Family of random functions
\newcommand{\prfSet}{\ensuremath{\ffamstyle{PRF}}\xspace} % Family of pseudo-random functions

%%%%% Constants %%%%%
\newcommand{\constantstyle}[1]{\mathtt{#1}}
%% Ethereum constants
\newcommand{\addressLen}{\ensuremath{\constantstyle{ADDRLEN}}\xspace} % Bit length of an Ethereum address
\newcommand{\ethWordLen}{\ensuremath{\constantstyle{ETHWORDLEN}}\xspace} % Bit length of an Ethereum word
\newcommand{\txDefaultGas}{\ensuremath{\constantstyle{DGAS}}\xspace} % Intrinsic gas cost of an Ethereum transaction
%% The general group and field
\newcommand{\Curve}{\ensuremath{\curvestyle{Curve}}\xspace} % Chosen curve
\newcommand{\rCURVE}{\ensuremath{\constantstyle{r}}\xspace} % Prime that defines the scalar field \FFx{\rCURVE} of \Curve
\newcommand{\fieldBitLen}{\ensuremath{\constantstyle{FIELDLEN}}\xspace} % Bit-length of elements in \FFx{\rCURVE}
\newcommand{\fieldBitCap}{\ensuremath{\constantstyle{FIELDCAP}}\xspace} % Bit-capacity of elements in \FFx{\rCURVE}
%% Remarkable groups and fields
\newcommand{\BNCurve}{\ensuremath{\curvestyle{BN\mhyphen{}254}}\xspace}
\newcommand{\rBN}{\ensuremath{\constantstyle{r_{BN}}}\xspace} % Instance of \r for the scalar field of BN254
\newcommand{\bnFieldBitLen}{\ensuremath{\constantstyle{FIELDLEN_{BN}}}\xspace} % Instance of \fieldBitLen for the BN254 scalar field
\newcommand{\bnFieldBitCap}{\ensuremath{\constantstyle{FIELDCAP_{BN}}}\xspace} % Instance of \fieldBitCap for the BN254 scalar field
\newcommand{\BLSCurve}{\ensuremath{\curvestyle{BLS12\mhyphen{}377}}\xspace}
\newcommand{\rBLS}{\ensuremath{\constantstyle{r_{BLS}}}\xspace} % Instance of \r for the scalar field of BLS12-377
\newcommand{\blsFieldBitLen}{\ensuremath{\constantstyle{FIELDLEN_{BLS}}}\xspace} % Instance of \fieldBitLen for the BLS12-377 scalar field
\newcommand{\blsFieldBitCap}{\ensuremath{\constantstyle{FIELDCAP_{BLS}}}\xspace} % Instance of \fieldBitCap for the BLS12-377 scalar field
\newcommand{\secpFieldBitLen}{\ensuremath{\constantstyle{SECPFIELDLEN}}\xspace} % Bit length of an element of the Secp256k1 scalar field
\newcommand{\pSecp}{\ensuremath{\constantstyle{p_{SECP}}}\xspace} % Prime scalar field Secp256k1
%% Zeth constants
\newcommand{\jsmax}{\ensuremath{\constantstyle{JSMAX}}\xspace} % Number of inputs in each Joinsplit
\newcommand{\jsin}{\ensuremath{\constantstyle{JSIN}}\xspace} % Number of inputs in each Joinsplit
\newcommand{\jsout}{\ensuremath{\constantstyle{JSOUT}}\xspace} % Number of outputs in each Joinsplit
\newcommand{\mkTreeDepth}{\ensuremath{\constantstyle{MKDEPTH}}\xspace} % Zeth merkle tree depth
\newcommand{\zvalueLen}{\ensuremath{\constantstyle{ZVALUELEN}}\xspace} % Zeth value length
\newcommand{\askLen}{\ensuremath{\constantstyle{ASKLEN}}\xspace} % ask length (spending key length)
\newcommand{\phiLen}{\ensuremath{\constantstyle{PHILEN}}\xspace} % phi length (joinsplit randomness length)
\newcommand{\noteByteLen}{\ensuremath{\constantstyle{NOTEBYTELEN}}\xspace} % note byte length
\newcommand{\encZethNoteLen}{\ensuremath{\constantstyle{ENCZETHNOTELEN}}\xspace} % encrypted note length
\newcommand{\chachaNonceValue}{\ensuremath{\constantstyle{CHACHANONCEVALUE}}\xspace} % chacha nonce value
\newcommand{\chachaBlockCounterValue}{\ensuremath{\constantstyle{CHACHABLOCKCOUNTERVALUE}}\xspace} % chacha block counter initial value
%%% Abstract primitives constants
\newcommand{\prfAddrOutLen}{\ensuremath{\constantstyle{PRFADDROUTLEN}}\xspace} % \prfaddr output length
\newcommand{\prfNfOutLen}{\ensuremath{\constantstyle{PRFNFOUTLEN}}\xspace} % \prfnf output length
\newcommand{\prfRhoOutLen}{\ensuremath{\constantstyle{PRFRHOOUTLEN}}\xspace} % \prfrho output length
\newcommand{\prfPkOutLen}{\ensuremath{\constantstyle{PRFPKOUTLEN}}\xspace} % \prfpk output length
\newcommand{\crhhsigOutLen}{\ensuremath{\constantstyle{CRHHSIGOUTLEN}}\xspace} % \crhhsig output length
\newcommand{\cmFLen}{\ensuremath{\constantstyle{CMFLEN}}\xspace} % \comm output length in field elements
\newcommand{\noterLen}{\ensuremath{\constantstyle{RTRAPLEN}}\xspace} % \noter length
\newcommand{\eskByteLen}{\ensuremath{\constantstyle{ESKBYTELEN}}\xspace} % ephemeral private key byte length
\newcommand{\epkByteLen}{\ensuremath{\constantstyle{EPKBYTELEN}}\xspace} % ephemeral public key byte length
\newcommand{\ctByteLen}{\ensuremath{\constantstyle{CTBYTELEN}}\xspace} % DHAES ciphertext byte length
\newcommand{\tagByteLen}{\ensuremath{\constantstyle{TAGBYTELEN}}\xspace} % tag byte length
\newcommand{\kdfDigestByteLen}{\ensuremath{\constantstyle{KDFDIGESTBYTELEN}}\xspace} % kdf digest byte length
\newcommand{\symKeyByteLen}{\ensuremath{\constantstyle{SYMKEYBYTELEN}}\xspace} % sym key byte length
\newcommand{\macKeyByteLen}{\ensuremath{\constantstyle{MACKEYBYTELEN}}\xspace} % mac key byte length

%%% OT Signature constants
\newcommand{\crhotsOutLen}{\ensuremath{\constantstyle{CRHOTSOUTLEN}}\xspace} % \crhhsig output length
%%% Packing constant
\newcommand{\nfFLen}{\ensuremath{\constantstyle{NFFLEN}}\xspace} % \prfnf output length
\newcommand{\htagFLen}{\ensuremath{\constantstyle{HFLEN}}\xspace} % \prfpk output length
\newcommand{\hsigFLen}{\ensuremath{\constantstyle{HSIGFLEN}}\xspace} % \crhhsig output length
\newcommand{\resBitsBLen}{\ensuremath{\constantstyle{RSDBLEN}}\xspace} % Residual Bits length
\newcommand{\resBitsFLen}{\ensuremath{\constantstyle{RSDFLEN}}\xspace} % Residual Bits field elements length
%% SHA256 constants
\newcommand{\shaTwoDigestLen}{\ensuremath{\constantstyle{SHA256DLEN}}\xspace} % SHA256 digest length
\newcommand{\shaTwoMsgLen}{\ensuremath{\constantstyle{SHA256MLEN}}\xspace} % SHA256 message length
\newcommand{\shaTwoBlockLen}{\ensuremath{\constantstyle{SHA256BLEN}}\xspace} % SHA256 block length
%% Keccak256 constants
\newcommand{\keccakTwoDigestLen}{\ensuremath{\constantstyle{KEK256DLEN}}\xspace} % Keccak256 digest length
%% Blake Constants
%% SHA256 constants
\newcommand{\blakeCompLen}{\ensuremath{\constantstyle{BLAKE2sCLEN}}\xspace} % BLAKE2 Compression input and output length
\newcommand{\blakeRound}{\ensuremath{\constantstyle{rounds}}\xspace} % BLAKE rounds
\DeclareRobustCommand{\blakePermutation}[1]{\ensuremath{\constantstyle{\Sigma}\ifthenelse{\equal{#1}{}}{}{[#1]}}\xspace} % BLAKE permutation constants
\DeclareRobustCommand{\blakeIV}[1]{\ensuremath{\constantstyle{IV}\ifthenelse{\equal{#1}{}}{}{[#1]}}\xspace} % BLAKE IV
\DeclareRobustCommand{\blakePB}[1]{\ensuremath{\constantstyle{PB}\ifthenelse{\equal{#1}{}}{}{[#1]}}\xspace} % BLAKE parameter block
\DeclareRobustCommand{\blakeFlag}[1]{\ensuremath{\constantstyle{F}\ifthenelse{\equal{#1}{}}{}{[#1]}}\xspace} % BLAKE flags
\DeclareRobustCommand{\blakeInitState}[1]{\ensuremath{\constantstyle{H}\ifthenelse{\equal{#1}{}}{}{[#1]}}\xspace} % BLAKE initial state
\DeclareRobustCommand{\blakeDigestLength}[1]{\ensuremath{\constantstyle{T}\ifthenelse{\equal{#1}{}}{}{[#1]}}\xspace} % BLAKE block length
%% Others
\newcommand{\byteLen}{\ensuremath{\constantstyle{BYTELEN}}\xspace} % Bit length of a byte

% Adversaries
\newcommand{\bdvii}{\ensuremath{\overline{\bdv}}} % two oracle diffie-helmann attacker

%% vectors
\newcommand{\vectorstyle}[1]{\vec{#1}} % format for vectors

% RFC2119
\newcommand{\rfcreq}[1]{\textbf{\texttt{#1}}}
\newcommand{\MUST}{\rfcreq{MUST}}
\newcommand{\MUSTNOT}{\rfcreq{MUST\,NOT}}
\newcommand{\SHOULD}{\rfcreq{SHOULD}}
\newcommand{\SHOULDNOT}{\rfcreq{SHOULD\,NOT}}
\newcommand{\RECOMMENDED}{\rfcreq{RECOMMENDED}}
\newcommand{\MAY}{\rfcreq{MAY}}
\newcommand{\ALLCAPS}{\rfcreq{ALL\,CAPS}}

% For Groth's SNARK
\newcommand{\crs}{\varstyle{crs}\xspace} % common reference string
\newcommand{\srs}{\varstyle{srs}\xspace} % structured reference string
\newcommand{\td}{\varstyle{td}\xspace} % trapdoor for srs
\newcommand{\constno}{\varstyle{constNo}\xspace} %number of constrains in the circuit
\newcommand{\inpno}{\varstyle{inpNo}\xspace} % number of inputs in the circuit
\newcommand{\inpnoprim}{\varstyle{inpNoPrim}\xspace} %number of primary inputs in the circuit
\newcommand{\multno}{\varstyle{multNo}\xspace} % number of multiplication gates in the circuit
\newcommand{\wireno}{\varstyle{wireNo}\xspace} % number of wires in the circuit
\newcommand{\RELQAP}{\REL_\projectstyle{QAP}\xspace} % for QAP relation
\newcommand{\xchi}{X} % for a formal variable corresponding to \chi
\newcommand{\HHH}{h} % quotient polynomial h
\newcommand{\AAA}{\gel{a}} % Groth's zkproof -- first element
\newcommand{\BBB}{\gel{b}} % Groth's zkproof -- second element
\newcommand{\CCC}{\gel{c}} % Groth's zkproof -- third element

%Comments and colors
\definecolor{blueish}{rgb}{0.1,0.1,0.5}
\definecolor{pinkish}{rgb}{0.9,0.8,0.8}

\DeclareRobustCommand{\michals}[2]  {{\color{blueish}\sethlcolor{pinkish}\hl{\textsf{Michal #1:} #2}}}

%math proofs
\newtheorem{corollary}{Corollary} %Corollary environment


% configuration related to this document
% Contracts
\newcommand{\relayexec}{\ensuremath{\contractstyle{RelayExec}}\xspace}
%% Contract methods
\newcommand{\relaymethod}{\ensuremath{\algostyle{relay}}\xspace}
\newcommand{\registerstakemethod}{\ensuremath{\algostyle{registerStake}}\xspace}
\newcommand{\refundstakemethod}{\ensuremath{\algostyle{refundStake}}\xspace}
%% Contract state
\newcommand{\stakemtree}{\ensuremath{\constantstyle{StakeMTree}}}
\newcommand{\stakenullifierlist}{\ensuremath{\constantstyle{NullifierList}}}

% Parties
\newcommand{\relayP}{\ensuremath{\partystyle{R}}}
\newcommand{\userP}{\ensuremath{\partystyle{U}}}
\newcommand{\relayEthAccount}{\eparty{R}}
\newcommand{\relayZethAccount}{\zparty{R}}
\newcommand{\userEthAccount}{\eparty{U}}
\newcommand{\userZethAccount}{\zparty{U}}

% Variables
\newcommand{\relaydata}{\ensuremath{\varstyle{data_{\partystyle{R}}}}\xspace}
\newcommand{\relayfee}{\ensuremath{\varstyle{fee_{\partystyle{R}}}}}
\newcommand{\relaytx}{\ensuremath{\tx_{\partystyle{R}}}}

\newcommand{\requestPool}{\ensuremath{\varstyle{requestPool}}} % Pool of the relay keeping track of the relay requests

\newcommand{\validstake}{\ensuremath{\varstyle{validstake}}}
\newcommand{\mixsuccess}{\ensuremath{\varstyle{mixSuccess}}}

\newcommand{\prfrelayperm}{\ensuremath{\pi^{(\mixparams)}_{\partystyle{R}}}}
\newcommand{\txid}{\ensuremath{\varstyle{txid}}}

%% % RelayPermission
%% \newcommand{\relayPermission}{\ensuremath{\datatypestyle{RelayPermission}}}
%% \newcommand{\mixparamshash}{\ensuremath{\varstyle{mixParamsHash}}}
%% %% \newcommand{\relayaddr}{\ensuremath{\varstyle{relayAddr}}}
%% %% \newcommand{\fee}{\ensuremath{\varstyle{fee}}}
%% %% \newcommand{\outaddr}{\ensuremath{\varstyle{outAddr}}}

% RelayRequest
\newcommand{\relayRequestDType}{\ensuremath{\datatypestyle{RelayRequest}}}
\newcommand{\mixparams}{\ensuremath{\varstyle{mixParams}}}
\newcommand{\outaddr}{\ensuremath{\varstyle{outAddr}}}
\newcommand{\relayaddr}{\ensuremath{\varstyle{relayAddr}}}
\newcommand{\fee}{\ensuremath{\varstyle{fee}}}
\newcommand{\permission}{\ensuremath{\varstyle{permission}}}

% Instance of RelayRequest
\newcommand{\relayrequest}[1]{\ensuremath{\varstyle{#1}}}
\newcommand{\req}{\relayrequest{req}}

\newcommand{\REQ}{\relayrequest{REQ}} % Meta request

% Relay Algorithms
\newcommand{\relaycheckmixparams}{\ensuremath{\algostyle{RelayCheckMixParams}}}
\newcommand{\relaycheckrequest}{\ensuremath{\algostyle{RelayCheckRequest}}}
\newcommand{\relayhandlerequest}{\ensuremath{\algostyle{RelayHandleRequest}}}
\newcommand{\relayRecordTransaction}{\ensuremath{\algostyle{recordTransaction}}}

% Ethereum-related functionalities
\newcommand{\genCallTx}{\ensuremath{\algostyle{genCallTx}}}
\newcommand{\broadcastTx}{\ensuremath{\algostyle{broadcastTx}}}
\newcommand{\sendE}{\ensuremath{\algostyle{send}}}
\newcommand{\insertleaf}{\ensuremath{\algostyle{insertLeaf}}}
\newcommand{\getpathtoleaf}{\ensuremath{\algostyle{getPathToLeaf}}}
\newcommand{\getRoot}{\ensuremath{\algostyle{getRoot}}}

% RelayExec Contract Algorithms
\newcommand{\relayexecprocessrequest}{\ensuremath{\algostyle{ProcessRequest}}}

% User Algorithms
\newcommand{\usercreaterequest}{\ensuremath{\algostyle{UserCreateRequest}}}

% Constants
%% Stake constants
\newcommand{\stakevalue}{\ensuremath{\constantstyle{STAKEVAL}}\xspace}
\newcommand{\stakewindow}{\ensuremath{\constantstyle{STAKEWINDOW}}\xspace}
\newcommand{\cmstakelen}{\ensuremath{\constantstyle{CMSLEN}}\xspace}
\newcommand{\rholen}{\ensuremath{\constantstyle{RHOLEN}}\xspace}

% Circuit
\newcommand{\STAKEREL}{\REL^{\projectstyle{S}}}
\newcommand{\STAKEREFUNDREL}{\REL^{\projectstyle{S}}}

% Functions
\newcommand{\stakecommit}{\algostyle{COMM}}
\newcommand{\USER}{\algostyle{USER}}
\newcommand{\RELAY}{\algostyle{RELAY}}

% Variables and Parameters
\newcommand{\cmstake}{\ensuremath{cm_{\constantstyle{S}}}}
\newcommand{\stakenullifier}{\ensuremath{nf_{\constantstyle{S}}}}
\newcommand{\stakeproof}{\ensuremath\pi_{\constantstyle{S}}}
\newcommand{\stakerefundproof}{\ensuremath\pi_{\constantstyle{W}}}
\newcommand{\stakeheight}{\ensuremath{height_\constantstyle{S}}}

% Ethereum values
\newcommand{\ethsender}{\ensuremath{\constantstyle{msg.sender}}}
\newcommand{\ethvalue}{\ensuremath{\constantstyle{msg.value}}}
\newcommand{\ethheight}{\ensuremath{\constantstyle{eth.height}}}

\newcommand{\hreq}{\ensuremath{h_{req}}}


% Add watermark to the document
\usepackage[printwatermark]{xwatermark}
\newwatermark[allpages,color=gray!25,angle=45,scale=3.5,xpos=0,ypos=0]{DRAFT}
% Comment command
\definecolor{bananamania}{rgb}{0.98,0.91,0.71}
\definecolor{darkred}{rgb}{0.7,0,0}
\DeclareRobustCommand{\antoines}[2]{{\color{darkred}\sethlcolor{bananamania}\hl{\textbf{Antoine #1:} #2}}}

\newtheorem{definition}{Definition}
\newtheorem{remark}{Remark}

\providecommand{\keywords}[1]{\textbf{\textit{Keywords---}} #1}

\usepackage{authblk} % Package to track authors' affiliations

\title{On the design of Zeth transaction relays}
\author[*]{Antoine Rondelet}
\author[*]{Duncan Tebbs}
\affil[*]{Clearmatics, UK}
\date{\today}

\begin{document}
\maketitle

\begin{abstract}
Privacy preserving protocols suffer from the need to pay transaction fees on blockchain systems. While such fees constitute a sound economic barrier to a wide class of Denial of Service attacks, they also represent impediments to the design of state transitions with anonymous initiators. In this paper, we navigate the design space for ``\zeth{} cryptographic relays''. These enable \zeth{} users to carry out \zeth{} payments anonymously by relying on an extra party -- a relay -- to settle and execute state transitions on the blockchain.

\keywords{Privacy, Ethereum, Zeth, Relays, Sender Anonymity, Digital Cash}
\end{abstract}

\tableofcontents

%%
% !TEX root = zeth_relay

\chapter{Preliminaries}\label{chap:preliminaries}

\section{Prerequisites}\label{preliminaries:prerequisites}

This document assumes familiarity with $\ethereum$ and $\zeth$.
It does not, in any way, aim to replace the Ethereum yellow paper~\cite{ethyellowpaper} or the Zeth specifications~\cite{zeth-protocol}. The reader is strongly advised to read about \ethereum~and \zeth{} before delving into this document.

\section{Notation}\label{preliminaries:notations}

Unless stated otherwise, this document follows the notation of the $\zeth$ protocol specifications~\cite{zeth-protocol}.
We note in particular the following notations (some of which originate from~\cite{zeth-protocol}), used throughout the document.
\begin{description}
    \item[\mixer{}] A deployed instance of the Zeth contract.
    \item[$\partystyle{U}$] A user, with identity \eparty{U}~on the \ethereum~network, and/or \zeth~identity \zparty{U}.
    \item[$\partystyle{R}$] An entity operating a relay, with identity \eparty{R}~on the \ethereum~network, and/or \zeth~identity \zparty{R}.
    \item[\relayexec{}] A contract deployed on the blockchain acting as a proxy to $\mixer{}$ (removing the need for trust between users and relays).
    \item[\relayfee{}] The fee of the relay service $\relayP$.
    \item[\genCallTx{}] Algorithm which creates, signs and broadcasts a transaction to call a contract entry point with specific parameters. That is, given a contract method call of the form $\contractstyle{Contract}.\algostyle{method}(\algostyle{param_1}, \ldots)$, a secret key $\sk$ for some \ethereum~address, a gas price $\gasp$ and gas limit $\gasl$, the algorithm $\genCallTx(\allowbreak \contractstyle{Contract}.\algostyle{method}(\allowbreak \algostyle{param_1},\allowbreak \ldots),\allowbreak \sk,\allowbreak \gasp,\allowbreak \gasl)$ creates a signed transaction that performs the given contract call.
    \item[\broadcastTx{}] Algorithm that accepts a signed \ethereum~transaction and broadcasts it to the network, returning the transaction ID. The caller (in this document, the relay) can use the transaction ID to monitor the asynchronous completion of the transaction. The exact details will depend on the relay implementation, but once the transaction is complete, the relay can retrieve its result and update any internal state.
\end{description}

\section{Terminology}\label{preliminaries:terminology}

The key words \MUST, \MUSTNOT, \SHOULD, \SHOULDNOT, \MAY, and \RECOMMENDED~in this document are to be interpreted as described in~\cite{rfc2119} when they appear in \ALLCAPS{}. These words may also appear in this document in lower case as plain English words, absent their normative meanings.

\section{Introduction}\label{preliminaries:introduction}

The $\zeth$ protocol allows users to carry out privacy-preserving state transactions on ``smart-contract enabled blockchains'' such as $\ethereum$ or $\projectstyle{Autonity}$\footnote{\url{https://github.com/clearmatics/autonity}}. Like all $\ethereum$ transactions, $\zeth$ transactions require a fee to be paid. This is inherited from the base protocol, which uses transaction fees as a security mechanism against Denial of Service (DoS) attacks.

As pointed out in the $\zeth$ paper~\cite{zethpaper}, the need to pay transaction fees represents a challenge for designers of privacy preserving protocols, since transaction originators must carry out $\zeth$ contract calls from a funded $\ethereum$ address (which in turn must have been funded by other user(s) on the system\footnote{Unless the user is a miner.}, and for which the ``controlling user identity'' must be known by at least one network member). As such, while $\zeth$ provides strong privacy guarantees (recipient anonymity, private payment amount etc.), sender anonymity remains hard to achieve.

This document proposes some designs for ``cryptographic relays'' and investigates the space of tradeoffs for both users and relay operators. The protocols enable $\ethereum$ peer-to-peer (P2P) nodes to act as \emph{relays}, receiving requests (off-chain), and signing and broadcasting transactions (which incorporate these requests) on behalf of $\zeth$ users. In exchange for this service, relays receive some fee from the original users.

As described below, relay fees are of paramount importance in establishing a sound incentive structure, which in turn is necessary for the overall robustness of the system. The primary goal of this study on ``cryptographic relays'' is to achieve $\zeth$ sender anonymity on blockchain systems, and the proposals in this document suggest multiple ways in which relay fees can be paid while maintaining this anonymity. Furthermore, under additional network assumptions (e.g.~namely that $\zeth$ users and relay nodes communicate via an Anonymous Communication (AC) network, e.g.~\cite{DBLP:conf/uss/PiotrowskaHEMD17}), $\zeth$ users can make use of relay nodes without revealing any identifying information to the relay. See \cref{unicast-vs-broadcast:network-anonymity} for further discussion.

\subsection{Turning ``front-runners'' into relays}\label{preliminaries:introduction:front-runners}

As noted in~\cite{daian2019flash} and~\cite{surrogeth-blogpost} (among others), on blockchains such as $\ethereum$, so-called ``front-runners'' actively seek out transactions that are profitable for the sender and attempt to replace them with modified versions, in order to steal the profit from the original sender. Front-running strategies leverage the mempool ordering policy adopted by miners. Namely, they set higher gas prices in order to overtake the targeted transactions.

With the proliferation of ``bots''\footnote{see, for instance, \url{https://github.com/Uniswap/uniswap-interface/issues/248}} inspecting the mempool and front-running profitable transactions (see e.g.~\cite{danrobinson-dark-forest}), it becomes key for ``layer 2''-protocol designers to design state transitions that are secure against such replay attacks.
As presented in~\cite[Section 2.3]{zeth-protocol}, the $\zeth$ protocol prevents ``front-running''/``replay'' attacks by design (see derivation of $\hsig$ and $\datatobesigned$).

While ``front-runners'' present a threat to users of Decentralized Applications (DApps), they can potentially be leveraged to act as transaction relays~\cite{surrogeth-blogpost}. Since front-runners examine the mempool, looking for profitable transactions to overtake (by extracting the transaction payload, creating a new transaction, signing it and broadcasting it on the network with a higher gas price), a user may exploit this behavior by voluntarily broadcasting a transaction with low gas price on the network, in the hope that a front-runner will replay/overtake it. By doing so, the user may thus trigger a state transition on the blockchain without paying the associated transaction fees. Nevertheless, ``front-runners'' should be modelled as rational agents, meaning that such transactions must be profitable to them. As such, for users to leverage ``front-runners'' as ``relays'' in practice, transactions must be crafted such that ``front runners'' receive a fee in exchange for replacing them.

Finally, in order for a user's transactions to be added to a miner's transaction pool, it is necessary for the transaction to pass the ``initial tests of intrinsic validity''~\cite[Section 6]{ethyellowpaper} (see also transaction pool implementation in Geth\footnote{\url{https://github.com/ethereum/go-ethereum/blob/master/core/tx_pool.go\#L578-L583}}). This means that users who wish to have their transactions ``front-run''/``relayed'' must hold a funded \ethereum{} account. This may not be desirable in all scenarios (especially in settings where sender anonymity is a primary motivation).

\subsection{Relay incentives and risks}\label{preliminaries:introduction:incentives-risks}

Besides the potential profitability of ``front-running'', mentioned above, relaying transactions that haven't been added to a miner's mempool is inherently risky, and sound incentive structures must reward such risk appropriately.

Firstly, it must be noted that relay nodes may be vulnerable to DoS attacks. In such attacks, malicious clients ``flood'' targeted relays with an overwhelming stream of transactions. While such attacks may be mitigated by relay operators using existing network monitoring techniques (e.g.~packet filtering, rate limiting etc.), it is also important for relay operators to assess the profitability of the transactions that they relay to the blockchain. In fact, running a relay may quickly become a ``money drain'' if the cost of operating the relay service (i.e.~infrastucture costs, transaction fees etc.) outweighs the relay fees received. As such, it is necessary for relays to have an efficient way to gauge the on-chain cost and profitability of a transaction. (Carrying out this operation may also exacerbate the DoS vector on relays, since a flood of maliciously crafted transactions -- such as transactions that take a long time to execute, but fail to release any funds -- may cause a relay node to spend all of its resources on transaction verification in return for no income\footnote{Additional security measures may alleviate such issues. For example, using properly crafted verification thresholds, discarding transactions that take too long to be verified. Note however that such mechanisms ``specialize'' a relay into relaying only certain classes of transactions. Again, proper tradeoffs need to be adopted depending on the use-cases and threat model.}.)
Additionally, it is worth remembering that ``front runners'' and ``relayers'' may in turn be front-run by other competitors. As a consequence, allocating non-negligible computation resources to ``assessing the profitability'' of transactions represents a risk -- other ``front runners'' may overtake the relay's (verified) transaction to avoid carrying out this verification work locally.

\subsection{Assumptions}\label{preliminaries:assumptions}

Based on the remarks given in the previous sections, we make the following assumptions in the rest of the document.
\begin{description}
    \item[Transactions to be relayed are immune to front-running.] We assume that all relayed transactions are inputs to the $\zeth$ contract $\mix$ function. As such, the inputs prevent front-running by construction.
    \item[Transactions to be relayed target specific relays.] As mentioned above, $\zeth$ is designed to avoid ``front-running''/``malleability'' attacks. To achieve this, several parameters (e.g.~$\datatobesigned$) are derived using the address of the $\ethereum$ user that must send the transaction. As such, we assume that users \emph{choose} a relay service to process their request. Note that, as discussed in \cref{unicast-vs-broadcast:requests:emulate-free-requests}, users are free to target multiple relay services by creating multiple requests, but the underlying assumption is there exists a market of competing ``relays'', from which users can select the service that best suits their needs. For example, different relays may offer different trade-offs between settlement latency and fees, while others may offer ``aggregation services'' (see, e.g.~\cite{DBLP:journals/corr/abs-2008-05958})
    \item[Relays are ``discoverable''.] Discovery of relay nodes by users may be achieved through several possible mechanisms. For example, relay nodes may publish their IP address, current fees and $\ethereum$ address (some of which may potentially be published on-chain), allowing users to discover their services. Overall, we assume that relay operators take the necessary steps to be ``discoverable'' by users.
\end{description}

In the remainder of this document, we propose a set of protocols to relay $\zeth$ transactions, each with their own trade-offs and specific goals.

\begin{notebox}
    Importantly, it is worth keeping in mind that in most scenarios, sound ``relay economics'' will imply that for a given state transition, relay fees are greater than the on-chain fees normally paid by blockchain users. Hence, we stress that using relays should not be seen as a ``cheap way'' to transact on a blockchain, but rather as a way to achieve otherwise impossible objectives on the system (e.g.~to achieve sender anonymity).
\end{notebox}

% !TEX root = zeth_relay

\chapter{Relays with Proof of Permission}\label{relay-proof-permission}

In this section, we propose a protocol with the aim of preserving the anonymity of $\zeth$ transaction senders. This allows a user \userZethAccount{} to interact (anonymously) with a \zeth~deployment to either ``pour'' the value of some of his $\zethnotes$ into new ones (i.e.~carry out a private payment), or withdraw some value $\vout$ from $\mixer{}$ to a newly created \ethereum~address $\addr_{new}$.

We assume that \userZethAccount{} is willing to pay a fee to achieve this anonymity, and that at least one party \relayEthAccount{} is willing to act as a ``relay'' in return for this fee.

\emph{By providing a mechanism to carry out private withdrawals of \zeth{} funds to a newly generated \ethereum{} address, we allow \ethereum{} users to manipulate $\ether$ ``privately'' in future transactions - either ``plain EOA-to-EOA transactions'' or smart-contract calls.}

\medskip

We list below the set of characteristics that are desired in this setting.

\paragraph{From the user's perspective}

\begin{itemize}
  \item The user must be able to leverage relays to anonymously carry out a \zeth{} state transition on-chain. This includes, anonymous private transfer (i.e.~``pouring'' the value of existing $\zethnotes$ to new ones), and anonymous withdrawals to a newly created $\ethereum$ account.
  \item Not only must the user be anonymous with regard to the blockchain network, but he must also remain anonymous to the relay\footnote{Further assumptions need to be made about the underlying network. The user must be able to communicate with the relay without revealing any network-layer identifying information.} for the mechanism to be robust against malicious/compromised relays. The user must not be required to reveal any identifying information, including any pre-funded $\ethereum$ addresses.
  \item The user must be guaranteed (up to some negligible probability) that the relay will only call the \mix{} function, on-chain, with the \emph{correct} inputs (i.e.~the relay may not execute any state transition on behalf of the user that the user did not request).
\end{itemize}

\paragraph{From the relay's perspective}

\begin{itemize}
  \item The relay must be guaranteed (up to some negligible probability) that he will receive the agreed upon fee in exchange for carrying out his role in the protocol (and therefore he will not incur costs such as transaction fees for no reward)\footnote{This would question the profitability of operating a relay node and would jeopardize the ``relay network'' as well as the viability of the ``relaying activity''.}.
\end{itemize}

\section{Protocol Overview}\label{relay-proof-permission:protocol-overvew}

%We define the following concepts in the system:
%\begin{description}
%  \item[\mixer{}.] A pre-deployed instance of the Zeth \mixer{} contract.
%  \item[\userZethAccount{}.] A $\zeth$ user holding Zeth address and note data required %to spend one or more $\zethnotes$. We assume that \userZethAccount{} wishes to either %``pour'' the value of $\zethnotes$ into new ones (i.e.~carry out a privae %payment), or withdraw some value $\vout$ from $\mixer{}$ to a newly created %$\ethereum$ account (which associated address is denoted $\addr_{new}$), and %is willing to pay a fee in order to achieve this anonymously.
%  \item[\relayEthAccount{}.] A relayer willing to sign and broadcast transactions on %behalf of users, in exchange for some fee.
%  \item [Proof of Relay Permission.] A description of the request for \relayEthAccount{} %to relay a specific transaction on behalf of \userZethAccount{}, accompanied by a %signature proving that \userZethAccount{} permits \relayEthAccount{} to perform the operation %and receive a specific fee.
%  \item[\relayexec{}.] A contract deployed to the blockchain which removes the %need for trust between users and relays. This contract checks the %proof-of-relay-permission, performs the \mix{} call, and correctly distributes %the fees and withdrawn value ($\vout$).
%\end{description}

We start by assuming that a \zeth~user \userZethAccount{} has chosen a relay service $\relayP$ (with \ethereum~account \relayEthAccount{}) which relays transactions for a fee $\relayfee$. We further assume that \userZethAccount{} knows the address of the \relayexec{} contract.

The protocol consists of the following steps:

\begin{description}
  \item[Step 1 (User creates the \mix{} parameters).] \userZethAccount{} creates the \mix{} parameters $\mixparams$ that spend her note(s), including the public output value $\vout$. $\mixparams$ are generated such that only \relayexec{} can successfully use them. (This is achieved by leveraging the property of \mix{} parameters, described in~\cite[Sections 2.4, 2.5]{zeth-protocol}, which restricts the $\ethereum$ address of the caller, possibly a contract\footnote{This is currently the mechanism used to prevent front-runners from claiming $\vout$.}.)

  \item[Step 2 (User generates a \emph{proof-of-relay-permission}).] With $\mixparams$ properly created, \userZethAccount{} generates a \emph{proof-of-relay permission} $\prfrelayperm$ (described in further detail below) for $\mixparams$. This proves that the owner of the $\zeth$ notes to be spent by $\mixparams$ agrees that \relayEthAccount{} may relay the \mix{} parameters via \relayexec{} for a fee $\relayfee$. \userZethAccount{} also specifies the address $\outaddr$ to which the remaining balance $\vout - \relayfee$ (if any) should be sent. (In general, $\outaddr$ is expected to be a newly generated address $\addr_{new}$).

  \item[Step 3 (User sends parameters to the relay).] \userZethAccount{} sends a ``relay request'' $\relayrequest{req}$ to the chosen relay $\relayP{}$, containing $\mixparams$, $\prfrelayperm$ and other data such as $\outaddr$. Note that, as long as $\outaddr$ is a newly generated address (with no history on the blockchain), this request contains no information that identifies \userZethAccount{} as the originator. \userZethAccount{} is also expected to leverage anonymising mechanisms to avoid revealing any identifying information at the transport level.

  \item[Step 4 (Relayer verifies and broadcasts the received request).] Upon receipt of $\relayrequest{req}$, the relay performs a set of checks to gain confidence that $\mixparams$ and $\prfrelayperm$ are valid, and indeed grant the relay fee to \relayEthAccount{}. (Note that the relay has some scope to choose the extent of such checks, trading off the cost of checking against the risk of losing money by broadcasting an invalid transaction.) If the relay is satisfied that $\relayrequest{req}$ is valid, he signs (using the \relayEthAccount{} identity) and broadcasts a transaction that calls \relayexec{}.

  \item[Step 5 (The intermediary contract checks all parameters and executes the \zeth~mixer).] The \relayexec{} contract acts as an intermediary, trusted by both users and relays. It first checks $\prfrelayperm$ to ensure that the caller (\relayEthAccount{}) has indeed been granted permission to use the \mixparams~in exchange for $\relayfee$. \relayexec{} then calls \mixer{} with parameters $\mixparams$ and checks that the call succeeds. The transaction is aborted if any of these checks fail.

  \item[Step6 (The intermediary contract distributes the value $\vout$).] If the \mix{} call is successful, \relayexec{} now holds the $\vout$ from \mixer{}. From this, it pays $\relayfee$ to \relayEthAccount{}.\addr and the remainder $\vout - \relayfee$ to $\outaddr$.
\end{description}

\begin{remark}
  Using \relayexec{} to distribute the fee and fund the user-specified $\outaddr$ address, gives \userZethAccount{} confidence that $\outaddr$ will receive the correct output for the agreed fee. Further, \relayEthAccount{} can be sure that no other relay can use the same set of \mix{} parameters and forge a request from \userZethAccount{} to receive the relay fee.
\end{remark}

\section{Protocol}\label{relay-proof-permission:protocol}

Below we give further details of the protocol outlined above, enabling anonymous $\zeth$ transfers via relayed transactions. The protocol leverages specific characteristics of the $\zeth$ protocol design (some of which were described above). In particular, it builds on the fact that $\mix$ parameters can be generated without owning an $\ethereum$ account (see derivation of $\hsig$ and related discussion~\cite[Remark A.2.2]{zeth-protocol}), and that $\mix$ parameters are ``bound'' to the address of the $\ethereum$ account that must call the $\mixer$ contract (see derivation of $\datatobesigned$~\cite[Section 2.3]{zeth-protocol}).

\subsection{Relay Request and Permission Data}

We use $\mixInputDType$ and related datatypes from~\cite[Section 2.1]{zeth-protocol}, and define the following new data type to represent a relay request with proof of relay permission.

\begin{definition}\label{protocol:datatypes:relayrequest}
  The datatype $\relayRequestDType$ is defined as:
  \begin{table}[H]
    \centering
    \begin{tabular}{p{0.15\linewidth} | p{0.6\linewidth} | p{0.2\linewidth}}
      Field & Description & Value\\ \toprule

      $\mixparams$ &
      Parameters to the \mix{} call &
      $\mixInputDType$ \\ \midrule

      $\outaddr$ &
      $\ethereum$ address credited with the funds withdrawn from $\mixer$ &
      $\BB^{\addressLen}$ \\ \midrule

      $\relayaddr$ &
      $\ethereum$ address allowed to relay \mixparams{} (and perceive $\fee$) &
      $\BB^\addressLen$ \\ \midrule

      $\fee$ &
      Fee to pay (out of $\vout$) to the authorized relay &
      $\NN_{\ethWordLen}$ \\ \midrule

      $\permission$ &
      Signature proving the authenticity of the request &
      $\sigOtsDType$ \\ \bottomrule
    \end{tabular}
    \caption{\relayRequestDType~type}
  \end{table}
\end{definition}

The $\permission$ attribute is used to indicate that the user has given permission for the relay to forward the specific \mix{} call. We reuse the signature key $\mixparams.\otsvk$ for the scheme $\sigscheme_{\otsig}$, used to create $\mixparams.\otssig$ (see~\cite[Section 2.3]{zeth-protocol}), since only the author of the \mix{} call parameters can generate valid signatures.

\begin{todobox}
    Tighten the security requirements of the signature scheme used.
    For now, this proposal uses $\sigscheme_{\otsig}$ to create a second signature with the same private key.  However, $\sigscheme_{\otsig}$ is ``one-time''.
    Additionally, the security claim on the nested signature is only that it is UF-CMA (see specs), relying on the fact that the wrapping signature scheme (signing the transaction object) is SUF-CMA. Here however, since the relay request does not contain a signed blockchain transaction object, an adversary may be able to maul the signature and pass the UF-CMA game. More consideration of the threat model and the security requirements is required here.
\end{todobox}

As described below, the user must sign the attributes $\relayaddr$, $\fee$ and $\outaddr$ for the signature to be considered valid.

\subsection{User Operations}

We assume that user \userZethAccount{} has decided to use \relayP{} (controlling $\ethereum$ account \relayEthAccount{}) to relay her $\zeth$ transaction in order to either withdraw some value $\vout$ to $\ethereum$ address $\outaddr$ and/or carry out a private transfer. We further assume that \userZethAccount{} agrees to pay a fee $\relayfee$ to \relayEthAccount{} in order to achieve this. Here $\relayfee$ is the fee that $\relayP$ is willing to accept in exchange for relaying a single \mix{} call.\footnote{This fee should be strictly greater than the gas cost of \mix{} in order for the relay to be profitable (more refined profitability forecasts would internalize the infrastructure operational costs to adjust the fee). Additionally, the relay may adjust and republish $\relayfee$ in light of gas price fluctuations on the blockchain, other changes to cost and risk, or to compete with other relays.}.

\userZethAccount{} executes the following steps:

\begin{enumerate}
  \item Create a valid $\mixparams \in \mixInputDType$, where:
    \begin{itemize}
      \item $\mixparams$ spends previously unspent $\zethnotes$ owned by \userZethAccount{}, with a public output of $\vout$.
      \item $\mixparams.\otssig$ is created using the address of \relayexec{}, as described in \cite[Section 2.3]{zeth-protocol}
      \item The one-time signing key $\sk_\otsig$, used to create $\mixparams.\otssig$, can be (securely) extracted for use in the following step.
    \end{itemize}
  \item Use the signing key $\sk_\otsig$ to create a proof of relay permission:
    \begin{align*}
      & \relaydata \gets \encode{\relayEthAccount.\addr}{}\ \concat\ \encode{\relayfee}{}\ \concat\ \encode{\outaddr}{} & & \\
      & \prfrelayperm \gets \sigscheme_{\otsig}.\sig(\sk_\otsig, \crhots{\relaydata}) & &
    \end{align*}
  \item Create a relay request $\req \in \relayRequestDType$:
    \begin{align*}
      \req & \gets \{ & \\
      & \mixparams: \mixparams, & \\
      & \outaddr: \outaddr, & \\
      & \relayaddr: relayEthAccount.\addr, & \\
      & \fee: \relayfee, & \\
      & \permission: \prfrelayperm & \\
      \} &  &
    \end{align*}
\end{enumerate}

\begin{remark}
    If the user simply wants to leverage a relay to carry out a private $\zeth$ transfer (without withdrawing funds to a newly created address $\outaddr$), $\userZethAccount{}$ can set $\outaddr \gets \constantstyle{0x0}$ (see~\cref{relay-proof-permission:fig:relayexec-processrequest} for more details).
\end{remark}

The user then sends this request to the relay \relayP{}, via a secure (anonymous) communication channel.

\subsection{Relay Operations}

Let $\relayEthAccount.\sk$ be the secret key corresponding to the address $\relayEthAccount.\addr$ of $\ethereum$ account $\relayEthAccount$.
In what follows, we assume that $\relayEthAccount.\addr$ is funded with enough $\ether$ to pay the gas required to call \relayexec{} on-chain.

% Algo to check request on Relay

Let $\relayEthAccount{}.\addr$ be the $\ethereum$ address of $\relayP$ charging relay fee $\relayfee$. Further, assume that $\relayexec$ and $\mixer$ are deployed with addresses $\relayexec{}.\addr$ and $\mixer.\addr$ respectively.

Given the current \ethereum~state $\wstate$ (or a copy holding at least $\wstate[\relayexec{}.\addr]$ and $\wstate[\mixer.\addr]$) and a relay request $\req \in\ \relayRequestDType$, the algorithm $\relaycheckrequest$ (see~\cref{relay-proof-permission:fig:relay-request-check}) succeeds if the request $\req$ is valid and will result in \relayEthAccount{} receiving $\relayfee$.

%% The algorithm $\relaycheckrequest$ (see~\cref{relay-proof-permission:fig:relay-request-check}) checks an incoming request of type $\relayRequestDType$ to ensure that the relay will receive $\relayfee$ in return for relaying it to the blockchain.

Note that we assume the existence of an algorithm $\relaycheckmixparams$ which checks whether a given set of \mix{} parameters $\mixparams$ result in a successful \mix{} call in the context of the current blockchain state $\wstate[\mixer{}.\addr]$. That is, given the state $\wstate[\mixer{}.\addr]$ of the \mixer{} contract, an $\ethereum$ address $\addr_{caller} \in \BB^\addressLen$ and \mix{} parameters $\mixparams \in \mixInputDType$,
\[
  \relaycheckmixparams(\wstate[\mixer{}.\addr], \addr_{caller}, \mixparams)
\]
returns the result of $\zethVerifyTx(\tx)$ (see \cite[Section 2.5]{zeth-protocol}) where $\tx$ is a transaction that calls $\mix(\mixparams)$ from address $\addr_{caller}$, and $\zethVerifyTx$ executes in the context of \mixer{} with state $\wstate[\mixer{}.\addr]$.

\newcommand{\mixeraddr}{\ensuremath{\varstyle{mixerAddr}}\xspace}
\newcommand{\relayproxyaddr}{\ensuremath{\varstyle{relayProxyAddr}}\xspace}

\begin{figure}[!h]
  \centering
  \procedure[linenumbering]{$\relaycheckrequest(\wstate, \req)$}{%
    \pccomment{Check the fee and relay address in the request} \\
    \pcif (\req.\fee \ne \relayfee)\ \lor\ (\req.\relayaddr \ne \relayEthAccount{}.\addr) \pcthen \\
    \pcind \pcreturn \false \\
    \pcendif \\
    \pccomment{Check that $\vout$ can pay the fee and reject deposits} \\
    \pcif (\req.\mixparams.\vout < \relayfee)\ \lor\ (\req.\mixparams.\vin \ne 0) \pcthen \\
    \pcind \pcreturn \false \\
    \pcendif \\
    \pccomment{Check the proof-of-relay-permission} \\
    \relaydata \gets \encode{\relayEthAccount{}.\addr}\ \concat\ \encode{\relayfee}{}\ \concat\ \encode{\req.\outaddr}{} \\
    \pcif \neg \sigscheme_{\otsig}.\verify(\req.\mixparams.\otsvk, \crhots{\relaydata}, \req.\permission) \pcthen \\
    \pcind \pcreturn \false \\
    \pcendif \\
    \pcreturn\ \relaycheckmixparams(\wstate[\mixer{}.\addr], \relayEthAccount{}.\addr, \req.\mixparams)
  }
  \caption{\relaycheckrequest{} algorithm. The relay address $\relayEthAccount.\addr$, desired relay fee $\relayfee$ and contract addresses $\relayexec{}.\addr$ and $\mixer.\addr$ are implicitly available as variables.}
  \label{relay-proof-permission:fig:relay-request-check}
\end{figure}

\begin{remark}
  For this check to be meaningful, we assume here that \relayEthAccount{} has access to $\wstate[\mixer.\addr]$ for some recent block height\footnote{Implementations are not necessarily expected to track $\wstate[\mixer.\addr]$ by themselves, but rather to leverage an existing $\ethereum$ full node implementation. $\relaycheckmixparams$ can then be implemented as queries to the node (e.g.~via RPC).}.
\end{remark}

\medskip

Further, we introduce the relay logic in~\cref{relay-proof-permission:fig:relay-handle-request} which illustrates the $\relayhandlerequest$ algorithm that processes a received $\relayRequestDType$ request object $\req$ and relays it on the blockchain network. Here again, we assume that the tuple $(\relayEthAccount{}.\addr,\allowbreak \relayfee,\allowbreak \relayexec{}.\addr,\allowbreak \mixer{}.\addr)$ is implicitly available to the algorithm.
In $\relayhandlerequest$ a gas price $\gasp$ and gas limit $\gasl$ are (possibly dynamically) set at the relay's discretion (based on its ``relaying service and strategy'', defining the trade-off between relay fees, settlement latency, etc.) and passed as explicit parameters to $\relayhandlerequest$.

%% Relay algorithm
\begin{figure}[!h]
  \centering
  \procedure[linenumbering, syntaxhighlight=auto, addkeywords={abort}]{$\relayhandlerequest(\wstate, \req)$}{%
    \pccomment{Check the request}\\
    \pcif \neg \relaycheckrequest(\wstate, \req) \pcthen \\
    \pcind abort \\
    \pcendif \\
    \pccomment{Create and broadcast the relay Ethereum transaction} \\
    \relaytx \gets \genCallTx(\relayexec{}.\relayexecprocessrequest(\req), \relayEthAccount{}.\sk, \gasp, \gasl) \\
    \txid \gets \broadcastTx(\relaytx) \\
    \pccomment{Record the transaction id} \\
    \relayRecordTransaction(\txid) \\
    \pcreturn
  }
  \caption{Algorithm to process relay requests. $\relayRecordTransaction$ represents the relay-specific handling of the transaction id.}
  \label{relay-proof-permission:fig:relay-handle-request}
\end{figure}

%% \antoines{11.11}{Strictly speaking we may be missing an RLP step when concatenating the request $\req$ to fucntion descriptor to form the tx data. Not sure. Need to be checked in the yellow paper.}

\begin{remark}\label{relay-proof-permission:transaction-failure-cases}
  The checks in $\relaycheckrequest$ provide some level of assurance that $\relaytx$ will be successfully executed on chain. However, the relay cannot rule out the possibility that a conflicting transaction $\relaytx^*$ exists on the network, such that, if $\relaytx^*$ were mined first it would alter the blockchain state and affect the execution of $\relaytx$. For example, in the case of \zeth, if some transaction $\relaytx^*$ which spends the same notes as $\relaytx$ were mined first, $\relaytx$ would fail and no payment would be made to the relay.

Relays may perform further checks to increase their level of confidence that $\relaytx$ will execute as expected (such as examining their own mempool for conflicting transactions) but all such checks add to the cost of running a relay, which may be reflected in the relay fees. It is expected that relays will compete with each other on the basis of their fees. Consequently, relays are expected to adopt some strategy that trades off risk, validation costs and competitiveness, and thereby determine an appropriate price range for relaying.
\end{remark}

\medskip

\begin{notebox}
Most relay implementations are likely to be able to receive and process multiple requests simultaneously, and at times may receive requests at a faster rate than they can be processed.  In this case, the relay has scope to prioritise certain requests over others.  In the simplest case, requests may be handled in the order in which they are received (for example via a FIFO queue). More sophisticated relays may employ a strategy allowing them to accept ``relay bribes'', in which case $\relayfee$ is composed of a ``base fee'' covering the cost of relaying the request, complemented by a relaying premium/bribe to be processed ahead of other requests.
    Additionally, we note that, the strategy adopted by the relays for ordering and processing relay requests is likely to be impacted by the economics of the underlying platform (see e.g.~\cite{eip-1559,eip-1559-analysis}), and so, can be adjusted at the relay's discretion.
\end{notebox}

\subsection{\relayexec Contract}\label{relay-proof-permission:protocol:relayexec}

\relayexec{} is a smart contract, deployed to the blockchain, with knowledge of the address of the \mixer{} contract to which transactions are to be forwarded. Any participant (relay or potential user) can verify its byte code, and be sure that it cannot be modified by any other party. It executes the $\mix$ call on behalf of the relay, distributing fees as described in the user's request, thereby allowing user and relay to interact in a trustless way. That is, neither the user or the relay are required to trust the the other party to behave in a certain way -- \relayexec~constrains how a relay request will be handled once both parties have ``agreed'' to it.

Relays create transactions that call the $\relayexecprocessrequest$ method on \relayexec{}. This method carries out processing of relay requests, and distribution of $\vout$, as defined in~\cref{relay-proof-permission:fig:relayexec-processrequest}

\begin{figure}[H]
  \centering
  \procedure[linenumbering, syntaxhighlight=auto, addkeywords={abort}]{$\relayexec.\relayexecprocessrequest(\req)$}{%
    \pccomment{Check that the fee is redeemable} \\
    \pcif \req.\mixparams.\vout < \req.\fee \pcthen \\
    \pcind abort \\
    \pcendif \\
    \pccomment{Check the relay permission} \\
    \relaydata \gets \encode{\req.\relayaddr}{}\ \concat\ \encode{\req.\fee}{}\ \concat\ \encode{\req.\outaddr}{} \\
    \pcif \neg \sigscheme_{\otsig}.\verify(\req.\mixparams.\otsvk, \crhots{\relaydata}, \req.\permission) \pcthen \\
    \pcind abort \\
    \pcendif \\
    \pccomment{Cross-contract call to the zeth mixer} \\
    \mixsuccess \gets \mixer.\mix(\req.\mixparams) \\
    \pcif \neg \mixsuccess \pcthen \\
    \pcind abort \\
    \pcendif \\
    \pccomment{Distribute the fee and the withdrawn funds} \\
    \sendE(\req.\relayaddr, \req.\fee) \\
    \varstyle{funds} \gets \algostyle{safeSub}(\req.\mixparams.\vout, \req.\fee) \\
    \pcif \req.\outaddr \neq \constantstyle{0x0} \land \varstyle{funds} \neq 0 \pcthen \\
    \pcind \sendE(\req.\outaddr, \varstyle{funds}) \\
    \pcendif \\
    \pcreturn
  }
  \caption{\relayexecprocessrequest{} function, where $\sendE(addr, amt)$ refers to the $\ethereum$ operation that sends funds $amt$ to some address $addr$, and where $\algostyle{safeSub} \colon \NN_{\ethWordLen} \times \NN_{\ethWordLen} \to \NN_{\ethWordLen}$, such that $\algostyle{safeSub}(x, y) = x - y$ if $x > y$, $0$ otherwise.}
  \label{relay-proof-permission:fig:relayexec-processrequest}
\end{figure}

Note that the definition of $\relayexecprocessrequest$ does not check that the transaction has originated from $\req.\relayaddr$. Hence, it is technically possible for a third party to ``front-run'' this transaction. However any value returned from the transaction is always explicitly distributed to $\req.\relayaddr$ and $\req.\outaddr$, and so any other party signing this would essentially be paying the transaction fee on behalf of \relayEthAccount{}, while still allowing \relayEthAccount{} to keep the relay fee.

The system would still function if $\relayexecprocessrequest$ did perform such a check (i.e.~that $\req.\relayaddr$ signed the transaction). However, omitting such a check allows the relay some flexibility when sending the transaction. For example, the relay may wish to pay the transaction fee from another account, or may wish to call \relayexec{} from another contract of some kind.

Finally, we note that users create relay requests such that only \relayexec{} may successfully pass $\mixparams$ to \mixer{}, however there is no mechanism to ensure that $\mixparams$ may only be used by a specific \emph{method} of \relayexec{}. Thus, before entering into the protocol, the user should convince himself that \relayexec{} cannot be called in such a way as to violate the protocol (consider the case of a malicious deployer colluding with a relay to provide a second method on \relayexec{} which returns all value to $\relayaddr$). For simplicity, we stipulate that \relayexec{} \MUSTNOT~have methods other than $\relayexecprocessrequest$.

% !TEX root = zeth_relay

\chapter{Relays with Private Fees}\label{relay-private-fees}

\section{Introduction}

\cref{relay-proof-permission} describes a protocol that allows users to to carry out $\zeth$ transactions while remaining anonymous (i.e.~make \mix{} calls without controlling an \ethereum{} account in $\wstate$). Under this scheme, relays receive their fee as part of the public output $\vout$ withdrawn from the \zeth~mixer contract. It is clear that, under the assumptions described in \cref{preliminaries:assumptions}, fees paid publicly \emph{in real time} in this way can be detected by observers of the \ethereum~network. Such observers may then learn about the profits made by each relay service, which may not be desirable.

In this chapter we introduce a protocol in which relays can receive fees of hidden denominations. This setting is of particular interest in the context of a ``relaying market'' in which a set of competing relay services operate with the aim of capturing as much bandwidth\footnote{i.e.~``market share''} (on the ``relay network'') as possible, in order to maximize their revenue.

\section{Protocol overview}\label{relay-private-fees:protocol-overview}

As in \cref{relay-proof-permission}, we propose a protocol built on top of \zeth{} which allows holders of \zeth{} notes to securely spend their notes without needing to hold $\ether$. We assume that relays are willing to sign and broadcast \mix{} calls, and therefore pay for the gas, in exchange for fee payment in the form of \zeth{} notes. To do so, relays must publish their public \zeth{} address $\relayZethAccount.\pubaddr$ and their fee $\relayfee$, as well as additional network information such as endpoints which accept relay requests.

Users create parameters $\mixparams$ to the \mix{} call, such that one of the newly created notes corresponds to payment of $\relayfee$ to $\relayZethAccount.\pubaddr$. These parameters are then sent to the relay via an established communication channel. Users must create the signature $\mixparams.\otssig$ using the relay's $\ethereum$ address $\relayEthAccount.\addr$, to allow \relayEthAccount~to use \mixparams{}.
(In some simple scenarios, $\relayEthAccount.\addr$ may be made public alongside other relay information. Here, however, we assume that $\relayEthAccount.\addr$ is passed securely from the relay to the user client as part of the protocol, as this gives the relay more flexibility -- see \cref{relay-private-fees:extensions:ether-output} for discussion of how this may enable further relay privacy.)

When the relay receives \mixparams{} from the user, it checks to ensure that \mixparams{} is valid and indeed contains an output note that pays their fee. Relays can then sign and broadcast a transaction directly calling $\mix(\mixparams)$ on the \mixer{} contract. Note that $\mixparams.\otssig$ ensures that the transaction cannot be front-run.

\subsection{Relay-originated mix transactions}

One potential advantage of relays receiving their fees in the form of \zeth~notes is that they maintain a level of privacy with respect to their fees. Observers that know the relay's \ethereum~address can tell that a given transaction is likely to be on behalf of some user, and therefore that one of the output notes is likely to be addressed to the relay (although they will be unable to see any amounts). However, relays can generate their own \mix{} transactions (which increase privacy by mixing their notes). These \mix{} transactions are indistinguishable from regular relay transactions created on behalf of other users.

\section{Limitations and extensions to the protocol}\label{relay-private-fees:extensions}

\subsection{Limitation of output notes}\label{relay-private-fees:extensions:jsout-limitation}

The \mixer{} contract is deployed with a hard-coded number of inputs and outputs (denoted $\jsin$ and $\jsout$ respectively). In any \mix{} call that is anonymized using the relay system described above, one of the outputs must be used to pay the relay. For the case where $\jsout = 2$ (a reasonable default value when relays are not considered), the utility of the system is greatly reduced since users may only set the one remaining output freely. In this case, users are able to combine multiple of their notes into another, but are unable to ``split'' input notes into multiple output notes. In particular, they are unable to pay a specific amount to another $\zeth$ user and receive change.

\subsection{Increasing $\jsout$}\label{relay-private-fees:extensions:increasing-jsout}

To address the issues of limited output notes, \zeth~could be instantiated with different parameters (in particular $\jsout \gets 3$), in order to support ``note-splitting'' and relay fee payment in a single transaction. However, such a change to the configuration may have several consequences for the protocol.

To support more output notes, each transaction requires more data to be transferred and processed. This increases the storage and processing requirements of the \mixer~contract (increasing transaction gas costs), and in turn decreases the lifetime of a \zeth~deployment for a given Merkle tree size (note that the Merkle tree size is defined when \mixer~is deployed). Furthermore, the $\zeth$ statement must be made more complex (in order to handle more commitments, and possibly to accommodate a deeper Merkle tree), increasing the cost of generating zero-knowledge proofs.

Note also that if $\jsout$ is increased, there may be a tendency for each user's funds to become distributed over more notes. If $\jsin$ is not also increased, and the ability of users to recombine \zeth~notes is not balanced with this, users may more frequently be required to issue multiple transactions when spending their funds (to ``recombine'' their funds spread across many notes).

Hence, adjusting $\jsout$ may have important consequences which should be considered very carefully, especially if the extra output notes are unlikely to be used outside of the relay system.

\subsection{Support for \ether~output}\label{relay-private-fees:extensions:ether-output}

By default, the \mixer~contract will return any \ether~value $\vout$ to the calling address which, in the protocol described here, would be $\relayEthAccount.\addr$, belonging to the relay. Thus, the protocol as presented does not allow users to withdraw value as \ether~while using a relay, unless he is willing to trust the relay to forward the \ether~in a later transaction. Our aim is to remove any need for trust within the protocol, and a trustless way to withdraw \ether~could be valuable in several scenarios.

As in \cref{relay-proof-permission}, users could withdraw \ether~to previously unseen \ethereum~addresses. Such anonymous addresses could then be used to pay for \zeth~transactions, apparently disconnected from any other transactions in the blockchain history. Note that this provides a means for users to anonymously perform \zeth~transactions that utilize all $\jsout$ output notes, without changing the \zeth~configuration. Clearly, a user performing two transactions (one to withdraw and one to execute the original \zeth~transaction) must pay the relay fee \emph{and} the transaction fee for his subsequent transaction. This may have an impact on the economic model for relay fees.

It is clear that relays will be required to regularly withdraw \zeth~notes as \ether, in order to continue to pay transaction fees. They can, of course, simply issue \zeth~transactions to withdraw to $\relayEthAccount.\addr$. However, if the relay protocol supports output to \ether, relays could also use this mechanism to withdraw to new \ethereum~addresses. To an observer, such transactions would appear to be standard relay transactions on behalf of some user, but would provide the relay with anonymous \ether, further increasing their privacy.

We next identify two approaches to supporting withdrawals to \ether~using the protocol given here.

\subsubsection{Arbitrary $\vout$ address in \zeth~protocol}\label{relay-private-fees:extensions:ether-output:out-addr}

The \zeth~protocol could be slightly modified so that $\mixparams$ contains an explicit output address  $\mixparams.\outaddr$ to which $\vout$ should be sent by \mixer{}. This new field $\mixer.\outaddr$ must be included in the data signed by $\mixparams.\otssig$, ensuring that it cannot be altered by front-runners or malicious relays. This approach adds a small overhead to the generation of $\mixparams$, and to the cost of \mix{} calls, since this output address must be passed as an extra parameter and used in signature validation. However, supporting this would add versatility to the \zeth~protocol and may allow other applications to be built on top of it.

Note that this new address $\outaddr$ is distinct from the \emph{sender's address} already included in $\mixparams.\otssig$. $\mixparams$ must ensure that the encapsulating \ethereum~transaction originates from $\relayEthAccount.\addr$, and that $\vout$ \ether~are paid to $\mixparams.\outaddr$.

\subsubsection{Relay via intermediary contract}\label{relay-private-fees:extensions:ether-output:intermediary}

An alternative approach to support secure withdrawal of \ether~via relays is to use an intermediary contract, as described in \cref{relay-proof-permission}. This change to the relay protocol has the benefit that $\vout$ can be distributed to one or more parties in a trustless way. However, it does have a potential down-side in terms of privacy -- namely that it is trivial for observers to distinguish between transactions issued by relays, and regular transactions issued by users, even if the observer does not know any \ethereum~addresses owned by relays.

%% These mixing transactions, created by the relay itself to mix its own notes, can also be used to withdraw \ether. (Note that relays must periodically withdraw \ether~in order to pay for transactions.) By using an intermediary (as described in \cref{relay-proof-permission}) or an extension to this protocol, as described below, relays can withdraw to new \ethereum~addresses. By using such ``anonymous'' \ethere~to then pay for relay transactions, relays may also gain some privacy with respect to their \ethereum~addresses. (Note that we assume above that relays publicly advertise their \ethereum~addresses, limiting the privacy that can be achieved using ``new'' \ethereum~addresses for relays. If, however, the protocol is modified so that the relay privately specifies an \ethereum~address to the client, the relay can maintain a pool of anonymous \ethereum~addresses and it will be more difficult for observers to detect relay transactions from regular \zeth~transactions).

\subsection{Fees as \ether~or \zeth~notes}

In order to address the problem of limited output notes in \zeth~(see~\cref{relay-private-fees:extensions:jsout-limitation}), the protocol could allow the user to choose between 2 fee payment methods: as a \zeth~note (as described here) or as \ether~via $\mixparams.\vout$. In this case, users can use \emph{all} $\jsout$ outputs from the $\mix$ call for their own purposes, potentially avoiding the need to adjust $\jsout$ in the \zeth~configuration, and all the associated problems (as described in~\cref{relay-private-fees:extensions:increasing-jsout}).

A simple way to pay fees as \ether~is for users to set $\vout = \relayfee$ when creating $\mixparams$. Upon receiving $\mixparams$, relays then check for \emph{either} a \zeth~note \emph{or} $\mixparams.\vout$ that pays their fees. The \zeth~protocol extension described in \cref{relay-private-fees:extensions:ether-output:out-addr} to add $\mixparams.\outaddr$ would then be desirable, to prevent front-runners from claiming the relay fee.

An alternative approach would be to use an intermediary contract as described in \cref{relay-proof-permission} (already partially mentioned in \cref{relay-private-fees:extensions:ether-output:intermediary} to support \ether~withdrawals). The protocol would then require the extra complexity of a request structure and \emph{proof-of-relay-permission}, but would provide maximal flexibility for users. A single $\mix$ call could withdraw \ether~to a new user address, use all output \zeth~notes \emph{and} pay the relay fee (in \ether).

We expect that, given the choice, users would favour fee payment in \ether~more often than payment in \zeth~notes, since fee payment in \ether~allows them to control all $\jsout$ output notes from the \zeth~transaction. Further, it seems reasonable to assume that there will always be some relay operators willing to accept relay fees in \ether, and thereby users will have some element of choice in how fees are paid. Hence, we should expect some divergence between the relays fees paid in \ether~and those paid in \zeth~notes -- namely that fees paid in \zeth~notes will tend to be lower, in order to incentivise users to adopt this protocol.


\appendix

% !TEX root = zeth_relay

\chapter{Relays with Stake}\label{relay-stake}

%% Stake Contract
\newcommand{\stakecontract}{\ensuremath{\contractstyle{RelayStake}}\xspace}
%% Proof of Relay Stake
\newcommand{\prs}{\pi_{\constantstyle{\req}}}

\begin{notebox}
  This section is concerned with a high-level description of a work-in-progress proposal. There are several issues still to be addressed before it can be considered complete.  It is given here as a first step towards the goal of addressing possible relay DoS vectors, with the hope that it can be iterated and eventually turned into a practical solution.
\end{notebox}

\section{Introduction}\label{relay-stake:introduction}
In the above proposals (\cref{relay-proof-permission,relay-private-fees}), relays receive requests and perform ``offline'' checks to gain a high level of confidence that the transaction (which the relay must sign and therefore pay for) will ``succeed'' and the relay will receive the designated fee. Under these protocols, a relay is exposed to risk in two forms:
\begin{enumerate}
\item\label{relay-stake:introduction:risk-validation-cost} Users are free to make invalid relay requests with no consequences (we assume that they connect via anonymising networks). At the same time, relays are highly incentivised to filter out such invalid requests and avoid signing and broadcasting the corresponding transaction (which would result in the relay paying the invalid transaction's gas, without receiving their relay fee). In order to detect invalid relay requests, the relay must essentially simulate execution of the full transaction against the current state of the blockchain. Although must less costly than paying the corresponding gas, this may still require significant compute resources. If the relay request is judged to be invalid, the relay will necessarily receive no relay fee in exchange for these verification costs.
\item As mentioned in \cref{relay-proof-permission:transaction-failure-cases}, there is a chance that a transaction appears valid (i.e.~it passes all checks performed by the relay), but later fails due to a conflicting transaction (unseen by the relay) being mined ahead of it. While this risk can be reduced by more thorough checks, the relay can never rule out the possibility that a conflicting transaction exists somewhere on the network.
\end{enumerate}

Both of these risks represent possible DoS attack vectors, especially~\ref{relay-stake:introduction:risk-validation-cost}, since clients can very easily craft invalid transactions with maximal cost of validation.

We consider a potential approach to address these problems. Specifically, we outline a protocol involving collateral staked by users and associated with some specific relay request. This supports a very lightweight check that relays can perform \emph{before} they commit either \ether~as gas for transactions, or compute resources to fully verify the validity of the request. If this fast upfront check passes, the relay is effectively guaranteed income as a result of relaying the transaction - whether or not the transaction is valid at the time it is mined. In this way, relays can avoid DoS attacks that force them to waste resources for no return.

The user's stake is pre-deposited with a contract \stakecontract{} in such a way that it is bound to a specific relay request. On receipt of a relay request, relays need only confirm that an associated stake exists before creating the relay transaction and executing it via \stakecontract{}. The relay can be sure that \stakecontract{} will release the stake to the relay, even if the associated relay transaction fails for any reason. In this way, relays can very quickly gain assurance (up to some negligible probability) that they will not lose any operating costs by proceeding with the request.

Since they are exposed to reduced risk, relays should be able to charge users lower fees for their services, while still allowing users to perform \zeth~operations that are not directly connected to any of their \ethereum~addresses. However, we note that this reduction in risk for the relay comes with a trade-off for the user, who must deposit \ether~from some address in order to use the system. Although users do not reveal \emph{which} \zeth~operation they are performing, they reveal that the owner of the \ethereum~address paying for the deposit \emph{may} interact with \zeth~at some future time.

While this does not provide as much anonymity as the systems described in \cref{relay-proof-permission,relay-private-fees}, this does represent an improvement in anonymity over the plain \zeth~protocol. (Note that users interacting directly with \zeth~can employ strategies to obfuscate their actions, such as broadcasting ``dummy'' transactions which mix their commitments, however this may incur a relatively high cost and always reveal to observers the exact set of commitments they have created.) The nature of the improvement achieved by the stake system will depend on the details of any final design. We discuss possible trade-offs, and potential strategies to mitigate them, in~\cref{relay-stake:remarks}.

\section{Protocol overview}\label{relay-stake:protocol-overview}

We describe the components involved in the protocol, and their role in the full workflow.

\subsection{Stake contract}

We assume the existence of some contract \stakecontract{} which performs multiple functions:
\begin{itemize}
\item Accept and hold stake as collateral against a specific relay request (and in turn specific relay). Users depositing stake should be able to generate a proof $\prs$ that \stakecontract{} holds some stake for a specific request $\req$. Further, verifiers of $\prs$ should not learn which transaction caused the deposit (and thereby the \ethereum~address of the user who deposited it).
\item Act as a relay intermediary, accepting relay requests and proof-of-relay-stake objects. If a valid request and proof are received, the sender has permission to act as a relay for the given request, and the stake is still unspent, \stakecontract{} executes the relay request (calls $\mixer{}.\mix(\req.\mixparams)$, using the notation of \cref{relay-proof-permission}) and releases the stake to the relay, regardless of the outcome of the $\mix{}$ call.
\end{itemize}
Note that this may be implemented as multiple interacting contracts.

\subsection{Relay}

Relays receive pairs $(\req, \prs)$ of relays requests and associated proof-of-relay-stake objects. When a relay \relayP{} (with \ethereum~address $\relayEthAccount.\addr$) receives such a pair, the operations he performs are relatively straightforward:

\begin{description}
\item [Step 1.] Check the correctness of $\req$ and $\prs$, namely that:
  \begin{enumerate}
    \item $\req$ names the relay's \ethereum~address as the recipient of the relay fee
    \item a valid stake exists in \stakecontract{}, corresponding to $\req$ and $\prs$
  \end{enumerate}
  The relay should not learn which transaction deposited the stake that corresponds to $\req$ and $\prs$ (which would reveal an \ethereum~account of the user).
\item [Step 2.] Create a transaction $\relaytx$ which calls \stakecontract{} with parameters $\req$ and $\prs$.
\item [Step 3.] Broadcast $\relaytx$ to the network and asynchronously wait to receive the relay fee.
\end{description}

As described above, after the initial check of stake corresponding to $\req$, the relay can be sure he will receive his fee even if the $\mix$ call fails. Note also that the relay's transaction cannot be front-run, since \stakecontract{} will only release the stake to the relay mentioned in $\req$. A user may be able to spend the notes in $\req.\mixparams$ via another transaction, but he will not be able to prevent the relay from claiming the stake he previously deposited.

For these reasons, the relay need only perform the checks listed above. There is little to be gained by checking any further details of $\req$, including $\req.\mixparams$ or the state of the \zeth~mixer contract \mixer{}.

\subsection{User}

A user who wants to make use of a specific relay \relayP{} with \ethereum~address $\relayEthAccount.\addr$, using $\mix{}$ parameters $\mixparams$, performs the following actions:
\begin{description}
\item [Step 1.] Create the appropriate $\mixparams$ and corresponding relay request $\req$, bound to the relay's address $\relayEthAccount.\addr$.
\item [Step 2.] Stake some \ether~with \stakecontract{} against $\req$ and generate a corresponding proof-of-relay-stake $\prs$. Note that this collateral is ``bound'' to $\req$.
\item [Step 3.] Send $\req$, $\prs$ to the chosen relay (via an anonymous channel), and asynchronously wait for a corresponding transaction to be mined.
\end{description}

Note that the user must have a funded \ethereum~account $\userEthAccount$ in order to stake collateral, impacting their anonymity to some extent.

\subsection{Reclaiming stake}\label{relay-stake:protocol-overview:reclaiming-stake}

Under some circumstances, a user may wish to reclaim his stake after depositing it in \stakecontract{}. In fact, if no mechanism were available to accomplish this, the protocol would be vulnerable to withholding attacks. That is, malicious relays could accept relay requests but not relay them within a reasonable time, essentially locking up the user's stake. Eventually, the victims of withholding attacks would be forced to forfeit their stake and find another means to carry out their \zeth~operations (potentially via another relay). Once the operation has been completed via another transaction, the malicious relay (still holding a request with associated stake) can then claim the user's stake by broadcasting the relay transaction, which will now fail. The victim must pay for his transaction more than once, losing his stake (which, may be greater than the original relay fee - see~\cref{relay-stake:remarks}). This attack would also cause significant disruption to the relay network and associated market.

\begin{notebox}
  Note that the affect of such an attack may be mitigated by some kind of reputation system alongside a relay market fee, so the threat posed by this may not be considered severe. In fact, a reputation system may be a vital part of any relay market.
\end{notebox}

If the user were allowed to reclaim their stake at any time, the transaction to reclaim it could be unseen by the relay, yet mined ahead of $\relaytx$. Thus, the relay would no longer have any guarantee that he would receive the stake in exchange for broadcasting the relay transaction.

It may be possible to facilitate reclaim of the stake by setting a ``period of validity'' for the stake, if this can be implemented such that:
\begin{itemize}
\item the user cannot reclaim the stake until the period of validity elapses,
\item the relay can obtain some estimate regarding the period of validity (minimally, if the relay knows some lower bound on the remaining period of validity, it can gain a high degree of certainty that the relay transaction will be mined before the user is able to reclaim the stake),
\item for a relay transaction $\relaytx$, observers (and relays) should not be able to infer significant information about the transaction that deposited the collateral for $\relaytx$,
\item for a stake reclaim transaction, observers should not be able to infer significant information about the transaction that deposited the stake being reclaimed,
\item the transaction to reclaim a stake cannot be front-run.
\end{itemize}

\section{Remarks}\label{relay-stake:remarks}

In this section we give some remarks about the proposed stake system above. Note that details are highly dependent on the specific cryptographic primitives used. The steps above give an outline of how a staking system may work, with the caveat that they are very likely to be modified in any fully specified protocol.

\paragraph{Stake deposit leaks information.} The transaction to deposit the user's stake reveals an \ethereum~address of the user, and the value of the stake.  Similarly, the transaction to reclaim an unused stake is also associated with the user's \ethereum~address. In particular, observers may learn about \ethereum~addresses of users likely to interact with the relay system. Note that this could potentially be mitigated if users are able, via some other mechanism such as \cref{relay-proof-permission}, to anonymously receive funds at new ``unused'' \ethereum~addresses. However, in isolation, the stake system always requires an initial transaction from a funded address.

\paragraph{Relays with stake as part of a wider market.} Reliance on external mechanisms (such as the ability to withdraw to new \ethereum~addresses, as mentioned above), may not necessarily be a problem for a stake-based relay protocol. We note that there may be significant benefit to an ecosystem of various relay types (or relays supporting a variety of request types) including those discussed in this document. In particular, we may imagine a market in which users can choose between a range of relay request types - those supporting more privacy and not requiring \ether (with higher fees, as a consequence of the extra risk assumed by the relay), and relays with lower fees which place more requirements on users (able to mitigate much of their risk). Concretely, such a system may allow users to withdraw \ether~to an anonymous address via an expensive relay, and then use this \ether~as collateral for further relay transactions with lower relay fees. Obviously, in such a scenario, users (and their wallet software) must take a great deal of care not to reveal relationships between transactions.

\paragraph{Collateral value vs Relay Fee.} The above outline suggests a very simple scenario in which the stake is unconditionally used as the relay fee (that is, the stake must have the same value as the relay fee, which cannot be negotiated after the stake has been deposited. We note that any concrete instantiation would require somewhat more flexibility, and would likely extend fee payment mechanisms in one of several ways.
\begin{itemize}
\item The user could be required to stake some fixed value from which fees are paid, with the ``change'' being paid either to the user, or as $\vin$ to the \zeth~mixer. The fixed stake value would then define an upper bound on relay fees. This allows a lot more scope for relays to dynamically change their fees.
\item We may prefer to distinguish between successful and failed $\mix$ calls. On success, \stakecontract{} may refund the user with the difference between the stake and the relay fee (as described above), while users could be punished for invalid relay requests by forfeiting their entire stake to the relay.
\item Similarly, it may be preferable for relay fees to be paid directly from the $\mix$ call in the case of successful transactions. That is, the $\mix$ call parameters $\req.\mixparams$ must include an output paying the relay (either $\vout$ as in \cref{relay-proof-permission}, or a \zeth~note as in \cref{relay-private-fees}), and the stake is refunded to the user (either to an address of their choosing, or as $\vin$ to the $\mix$ call. This allows relays to maintain privacy with respect to their fee payments, while minimizing risk through the stake system.
\item It may be possible to allow users to deposit a single stake, bound to a specific relay, which can then be used for multiple relay requests. The relay could still claim the stake by presenting an invalid request from the user, but the user could make multiple relay requests (possibly within some time limit - see the following paragraph) without the need to redeposit. User anonymity would be improved because, while observers would still learn the \ethereum~address depositing each stake, they would not be able to determine how many relay requests were carried out on behalf of each depositor.
\end{itemize}

\paragraph{Period of validity.} The period of validity gives observers information about when the users relay transaction will be carried out. Depending on the frequency of relay transactions interacting with \stakecontract{}, users should ensure that the period of validity is sufficiently long, to avoid compromising their anonymity. Similarly, a predictable interval between a stake being deposited and the corresponding relay transaction that claims it, would also reveal information about the user originating each relay transaction. To avoid this, the period of validity should be sufficiently long to allow some ``noise'' in the interval between stake deposits and relay transactions. Note that this presents a trade-off, since the user's funds are potentially ``locked up'' for a longer period.

%%
%% ---------------------------------------- Saved Reference Text
%%

%% \newcommand{\csr}{\ensuremath{cm_{\constantstyle{S\textnormal{-}R}}}}
%% \newcommand{\psr}{\ensuremath{\pi_{\constantstyle{S\textnormal{-}R}}}}
%% \newcommand{\nfsr}{\ensuremath{nf_{\constantstyle{S\textnormal{-}R}}}}
%% \newcommand{\ps}{\ensuremath{\pi_{\constantstyle{S}}}}

%% %% Stake Contract and Methods
%% \newcommand{\stakecontract}{\ensuremath{\contractstyle{Stake}}\xspace}
%% \newcommand{\placestake}{\ensuremath{\algostyle{placeStake}}\xspace}
%% \newcommand{\registerrequest}{\ensuremath{\algostyle{registerRequest}}\xspace}
%% \newcommand{\claimstake}{\ensuremath{\algostyle{claimStake}}\xspace}
%% \newcommand{\revokestake}{\ensuremath{\algostyle{revokeStake}}\xspace}
%% \newcommand{\withdrawstake}{\ensuremath{\algostyle{revokeStake}}\xspace}


%% We assume the existence of some contract \stakecontract{} which is able to process requests as specified in the following description. Note that this functionality may be included within \relayexec{}, but is described here as a separate contract for simplicity (note that \stakecontract{} is expected to expose any of its state that may be required by \relayexec{}).

%% \paragraph{User Stakes Collateral.}

%% Before using the relay system, the user \userP{}, using some \ethereum{} address $\addr_U$ calls the \placestake{} method on \stakecontract{}, sending some agreed upon amount $\stakevalue$. \stakecontract{} records this event in its state, and holds the collateral.

%% \paragraph{User Submits Relay Request Against Stake.}

%% \userP{} creates the relay request $\req \in \relayRequestDType$, as specified above and creates a commitment $\csr$ to this request, which also binds it to \userP{}'s stake, without revealing any details of the request itself. \userP{} passes $\csr$ to the \registerrequest{} method of \stakecontract{}.

%% Upon receiving $\csr$, $\stakecontract$ ensures that no other commitment is associated with the stake, and records the commitment in its state. This ensures that a single stake cannot be used as collateral against more than one relay request.

%% Once $\csr$ is successfully registered with \stakecontract{}, \userP{} generates a spending token (or nullifier) $\nfsr$, and proof of commitment $\psr$ showing that:
%% \begin{itemize}
%% \item $\req$ is associated with a registered commitment held by $\stakecontract$, and
%% \item $\nfsr$ corresponding to the same commitment.
%% \end{itemize}
%% without revealing which commitment the request is associated with.

%% The user passes $(\req, \nfsr, \psr)$ to \relayP{}.

%% \paragraph{Relay Receives Request}

%% Upon receiving $(\req, \nfsr, \psr)$, the relay checks these against a recent snapshot of the \stakecontract{} state.  Namely, based on a recent snapshot of \stakecontract{}, the relay confirms that $\psr$ proves that the $\req$ is an opening of some commitment held by \stakecontract{}, whose nullifier is $\nfsr$.

%% If this check succeeds, the relay can have a high level of confidence that he will receive the cost of any further resources spent on $\req$. This is because:
%% \begin{itemize}
%% \item no other request (potentially tied to some other relay) is associated with the commitment nullified by $\nfsr$,
%% \item the user cannot revoke the commitment within some time window,
%% \item if $\req$ turns out to be \emph{invalid}, the relay can present $(\req, \nfsr, \psr)$ to \stakecontract{}, nullifying the users stake, and receive $\stakevalue$. (Note $\stakevalue$ is assumed to be set such that it covers the gas costs of this transactions along with the costs of any computation.  It should also be large enough to be a deterrent to malicious users.)
%% \item if $\req$ turns out to be \emph{valid}, the relay will receive the agreed upon fee.
%% \end{itemize}

%% \paragraph{Relay Submits Relay Transactions}

%% After verifying the $(\req, \nfsr, \psr)$ are valid (and that the checks on $\req$ specified in \cref{relay-proof-permission:protocol} all pass), relay creates and signs a transaction $\relaytx$ which passes $\req$ to \relayexec{}.

%% \paragraph{Relay Claims Stake for Invalid Relay Requests}

%% If $\req$ is an invalid request which does not result in \relayP{} receiving the promised fee, the relay submits $(\req, \nfsr, \psr)$ where $\psr$ is a valid proof for $\req$ and $\nfsr$ against the latest \stakecontract{} state. \stakecontract{} checks that $\psr$ is a valid proof showing that $\req$ is associated with a currently held commitment whose nullifier is $\nfsr$ (where $\nfsr$ is not held by \stakecontract{}). If this check passes, \stakecontract{} also checks the validity of $\req$.  If $\req$ is found to be invalid, \stakecontract{} records $\nfsr$ (invalidating the stake) and sends the stake value $\stakevalue$ to the \relayP{}.

%% \paragraph{User Withdraws Stake}

%% After \userP{} has used his stake, and some suitable time period has passed (in order to attempt to disguise the time at which the stake was used), he may wish to withdraw it from \stakecontract{}. \userP{} calls the \withdrawstake{} method on \stakecontract{}, passing $\csr$, $\nfsr$ and a proof $\hat{\ps}$ that $\nfsr$ is the nullifier for $\csr$. \stakecontract{} checks that the nullifier has not been used (i.e. no invalid request has been guaranteed using this nullifier).

%% \begin{todobox}
%%   \begin{itemize}
%%   \item If $\nfsr$ here is the same as that used to guarantee relay requests, a relay can compare the nullifiers in these withdraw requests to its records and identify the commitment and thereby the \ethereum address of the user who originated the request.
%%   \item Need to add some timing restrictions to stop users from posting commits and then front-running relay transactions with withdrawal requests.
%%   \end{itemize}
%% \end{todobox}

%% Note that \relayexec{} could include logic to automatically claim the users' stake in the case that any of the relay request check fail. Additionally, \relayexec{} and \stakecontract{} could be the same contract. Here we intentionally separate these two processes for clarity. This also solves the problem of front-runners on the stake claim transaction (i.e.~there is no need to bind $\csr$ to the $\relayEthAccount{}$.

%% \section{Protocol Specification}\label{relay-stake:protocol}

%% \subsection{Constants}

%% \begin{description}
%%   \item[\stakevalue{}] The fixed value (in $\wei$) that users must stake in order to use the relay system.
%%   \item[\stakewindow{}] A time window (specified in blocks). Stakes are posted with some (initially hidden) \emph{start time} from which they can be used. Users must make their requests (and relays must broadcast them) within $\stakewindow$ blocks of the hidden start time. After $\stakewindow$ blocks have elapsed from the start time, the client is permitted to claim a refund for his stake if it has not been used.
%% \end{description}

%% \subsection{Zero-Knowledge Proof of Relay Stake}

%% The proof-of-relay-stake is created by the user and passed to the relay. Using this, the relay and the \relayexec{} contract can check that the user has posted a stake which is bound to the request being handled, that the stake has not been used for any previous request and that the stake is currently active (the current block height is within the active time window of the stake).

%% \begin{definition}[Proof of Relay Stake]\label{relay-stake:proof-of-relay-stake}
%%   For public inputs:
%%   \begin{itemize}
%%   \item $\hreq$ --- hash of the request to be processed
%%   \item $\stakenullifier$ --- nullifier for the stake
%%   \item $\stakeheight$ --- height at which stake becomes valid
%%   \item $\mkroot$ --- Merkle root
%%   \end{itemize}
%%   and auxiliary inputs:
%%   \begin{itemize}
%%   \item $\rho$ --- secret used in stake commitment
%%   \item $\mkpath$ --- Merkle path of commitment to $\mkroot$
%%   \end{itemize}
%%   $\STAKEREL$ defines the following relations:
%%   \begin{enumerate}
%%   \item
%%     For $\cmstake = \stakecommit( \rho\ \concat\ \hreq\ \concat\ \stakeheight )$, $\mkpath$ is a valid path from $\cmstake$ to $\mkroot$.
%%   \item
%%     $\stakenullifier = \stakecommit( \rho )$
%%   \item
%%     $\stakeheight < h \leq \stakeheight + \stakewindow$
%%   \end{enumerate}
%% \end{definition}

%% \subsection{Zero-Knowledge Proof of Stake for Refund}

%% When $\stakewindow$ blocks have passed since the stakes activation height $\stakeheight$, the user is permitted to reclaim his stake if it has not been used. The following proof is used to demonstrate to \relayexec{} that the user is the owner of the stake.

%% \begin{definition}[Proof of Stake for Refund]\label{relay-stake:proof-of-stake-for-refund}
%%   For public inputs:
%%   \begin{itemize}
%%   \item $\hreq$ --- hash of the request to be processed
%%   \item $\stakenullifier$ --- nullifier for the stake
%%   \item $\stakeheight$ --- height at which stake became valid
%%   \item $\mkroot$ --- Merkle root
%%   \end{itemize}
%%   and auxiliary inputs:
%%   \begin{itemize}
%%   \item $\rho$ --- secret used in stake commitment
%%   \item $\mkpath$ --- Merkle path of commitment to $\mkroot$
%%   \end{itemize}
%%   $\STAKEREFUNDREL$ defines the following relations:
%%   \begin{enumerate}
%%   \item
%%     For $\cmstake = \stakecommit( \rho\ \concat\ \hreq\ \concat\ \stakeheight )$, $\mkpath$ is a valid path from $\cmstake$ to $\mkroot$.
%%   \item
%%     $\stakenullifier = \stakecommit( \rho )$
%%   \item
%%     $ \stakeheight + \stakewindow < h$
%%   \end{enumerate}
%% \end{definition}

%% \begin{todobox}
%%   zk-proof of $\stakeheight + \stakewindow < h$ may not be feasible in the way described here, since $h$ (the current block height) is not known upfront.
%%   Investigate ways to solve this.
%% \end{todobox}

%% \begin{todobox}
%%   The zk-proof here may not be required.  Investigate whether any important information is leaked if the user simply submits $\rho$, $\hreq$, $\stakeheight$, $\cmstake$ and $\stakenullifier$ to \relayexec{}. (On condition that the user does not attempt to reuse the same $\req$ or $\outaddr$ in future stakes or relay requests.)
%% \end{todobox}

%% \subsection{\relayexec{} Contract}

%% We augment the contract \relayexec{} in \cref{relay-proof-permission:protocol:relayexec} to have the following state.
%% \begin{description}
%%   \item[$\stakemtree$] Merkle tree of commitments.
%%   \item[$\stakenullifierlist$] List of nullifiers.
%%   \item[$\rootset$] The history set of Merkle roots of the Merkle tree $\stakemtree$.
%% \end{description}

%% Additionally, we denote by:
%% \begin{description}
%%   \item[$\insertleaf(T, l)$] the algorithm that inserts $l$ as a leaf in the Merkle tree $T$ and returns the index $i$ to which the leaf was added.
%%   \item[$\getpathtoleaf(T, i)$] the algorithm that returns the Merkle path (of length $\mkTreeDepth$) to leaf at index $i$.
%%   \item[$\getRoot(T)$] the algorithm that returns the Merkle root of tree $T$
%% \end{description}

%% Further, we distinguish between the entry points: $\registerstakemethod$, $\relaymethod$ and $\refundstakemethod$.

%% \begin{figure}[H]
%%   \centering
%%   \procedure[linenumbering, syntaxhighlight=auto, addkeywords={abort}]{$\registerstakemethod(\cmstake)$}{%
%%     \pcif (\ethvalue < \stakevalue) \pcthen \\
%%     \pcind abort \\
%%     \pcendif \\
%%     \pccomment{Record $\cmstake$ in $\stakemtree$} \\
%%     i \gets \insertleaf(\stakemtree, \cmstake) \\
%%     \pccomment{Record new Merkle root} \\
%%     \rootset \gets \rootset\ \cup\ \getRoot(\stakemtree) \\
%%     \pcreturn i
%%   }
%% \end{figure}

%% \begin{figure}[H]
%%   \centering
%%   \procedure[linenumbering, syntaxhighlight=auto, addkeywords={abort}]{$\relaymethod(\req, \stakenullifier, \stakeheight, \mkroot, \stakeproof)$}{%
%%     \pcif (\stakenullifier \in \stakenullifierlist) \pcthen \\
%%     \pcind abort \\
%%     \pcendif \\
%%     \pcif (\ethheight < \stakeheight) \lor (\ethheight \geq \stakeheight + \stakewindow) \pcthen \\
%%     \pcind abort \\
%%     \pcendif \\
%%     \pccomment{TODO: How can the fee be compared with a relation?? To clean.} \\
%%     \pcif (\req.\fee < \STAKEREL) \pcthen \\
%%     \pcind abort \\
%%     \pcendif \\
%%     \pccomment{Make sure that the merkle root corresponds to a valid merkle tree state} \\
%%     \pcif \mkroot \not\in \rootset \pcthen \\
%%     \pcind abort \\
%%     \pcendif \\
%%     \hreq \gets \crh(\req) \\
%%     \pccomment{TODO: Introduce proof system etc to call the verification algo here} \\
%%     \text{$\validstake \gets$ result of verifying $\stakeproof$ as proof of $\STAKEREL$ with public inputs $(\hreq, \stakenullifier, \stakeheight, \mkroot )$} \\
%%     \pcif \neg \validstake \pcthen \\
%%     \pcind abort \\
%%     \pcendif \\
%%     \stakenullifierlist.\algostyle{append}(\stakenullifier) \\
%%     \pccomment{Perform checks on $\req$. If an error is found, send stake to relay and return.} \\
%%     \pcif \req.\mixparams.\vout < \req.\fee \pcthen \\
%%     \pcind \sendE( \req.\relayaddr, \stakevalue ) \\
%%     \pcind \pcreturn \false \\
%%     \pcendif \\
%%     % Check signature
%%     \relaydata \gets \encode{\req.\relayaddr}{}\ \concat\ \encode{\req.\fee}{}\ \concat\ \encode{\req.\outaddr}{} \\
%%     \pcif \neg \left[\sigscheme_{\otsig}.\verify(\req.\mixparams.\otsvk, \crhots{\relaydata}, \req.\permission)\right] \pcthen \\
%%     \pcind \sendE( \req.\relayaddr, \stakevalue ) \\
%%     \pcind \pcreturn \false \\
%%     \pcendif \\
%%     % Mix call
%%     \mixsuccess \gets \algostyle{call}( \mix, \req.\mixparams ) \\
%%     \pcif \neg\mixsuccess \pcthen \\
%%     \pcind \sendE( \req.\relayaddr, \stakevalue ) \\
%%     \pcind \pcreturn \false \\
%%     \pcendif \\
%%     \pccomment{Pay the relay fee and send any remaining stake and vout to \req.\outaddr.} \\
%%     \sendE( \req.\relayaddr, \req.\fee ) \\
%%     \sendE( \req.\outaddr, \req.\mixparams.\vout + \stakevalue - \req.\fee ) \\
%%     \pcreturn \true
%%   }
%% \end{figure}

%% \begin{figure}[H]
%%   \centering
%%   \procedure[linenumbering, syntaxhighlight=auto, addkeywords={abort}]{$\refundstakemethod(\hreq, \stakenullifier, \stakeheight, \mkroot, \stakerefundproof)$}{%
%%     \pcif (\stakenullifier \in \stakenullifierlist) \pcthen \\
%%     \pcind abort \\
%%     \pcendif \\
%%     \pccomment{Make sure that the merkle root corresponds to a valid merkle tree state} \\
%%     \pcif \mkroot \not\in \rootset \pcthen \\
%%     \pcind abort \\
%%     \pcendif \\
%%     \pcif (\ethheight < \stakeheight + \stakewindow) \pcthen \\
%%     \pcind abort \\
%%     \pcendif \\
%%     \pccomment{TODO: Call the proof system verifiation routine} \\
%%     \text{$\validstake \gets$ result of verifying $\stakerefundproof$ as proof of $\STAKEREFUNDREL$ with public inputs $(\hreq, \stakenullifier, \stakeheight, \mkroot)$} \\
%%     \pcif \NOT\ \validstake \pcthen \\
%%     \pcind abort \\
%%     \pcendif \\
%%     \stakenullifierlist.\algostyle{append}(\stakenullifier) \\
%%     \sendE( \ethsender, \stakevalue ) \\
%%     \pcreturn \true
%%   }
%% \end{figure}

%% The intention of \refundstakemethod{} is to allow the user to reclaim his stake if it was not used.  Note that if a malicious relay receives a valid request associated with some stake, he knows the nullifier for that stake. If the user then requests a refund, he reveals this nullifier and associates it with one of his \ethereum{} addresses. However, assuming that the user does not reuse the same $\outaddr$ and $\vout$ values in future relay requests, the relay learns only that the user wanted to make use of the relay system (which is already known to all observers), and the quantity of Ether that the user wanted to withdraw.

%% We conclude that the user can safely attempt to use another relay anonymously after receiving a refund for his stake.

%% \subsection{User Actions}

%% Assume that user \userP{} has decided to use a relay \relayP{} to broadcast some \zeth{} call described by $\mixparams$, in exchange for a fee $\relayfee$. Below, we assume that \userP{} holds sufficient $\ether$ to pay for $\stakevalue$ and gas.

%% Given $\mixparams$, where $\mixparams.\otssig$ binds $\mixparams$ to the \ethereum{} address of \relayexec{} (see \cite[Section 2.3]{zeth-protocol}), we assume that the user has retained the one-time signing key $\sk_\otsig$ used to create $\mixparams.\otssig$. Let $\stakeheight$ be the block-height from which the user intends to submit the relay request (which should include some randomness).

%% The user $\userP{}$ executes the following steps to prepare the request and place the stake.

%% \begin{figure}[H]
%%   \centering
%%   \procedure[linenumbering]{$\USER_1(\mixparams, \sk_\otsig, \stakeheight)$}{%
%%     \relaydata \gets \relayEthAccount.\addr\ \concat\ \relayfee\ \concat\ \outaddr \\
%%     \permission \gets \sigscheme_{\otsig}.\sig( \sk_\otsig, \crhots{\relaydata} ) \\
%%     \req \gets \{  \\
%%     \pcind \pcind \mixparams: \mixparams, \\
%%     \pcind \pcind \outaddr: \outaddr, \\
%%     \pcind \pcind \relayaddr: \relayEthAccount.\addr, \\
%%     \pcind \pcind \fee: \relayfee, \\
%%     \pcind \pcind \permission: \permission \\
%%     \} \\
%%     \hreq \gets \crh( \encode{\req}{} ) \\
%%     \rho \sample \BB^\rholen \\
%%     \cmstake \gets \stakecommit( \rho\ \concat\ \hreq\ \concat\ \stakeheight ) \\
%%     \sendE( \stakevalue, \relayexec.\registerstakemethod, \cmstake ) \\
%%     \pcreturn (\req, \hreq, \rho, \cmstake)
%%   }
%% \end{figure}

%% \begin{todobox}a
%% Introduce a function to create a request object and use it in the algo. That'll be cleaner.
%% \end{todobox}

%% Here the $\sendE$ is assumed to create a transaction which sends $\ether$ to a contract, invoking a specific method on that contract with some arguments.

%% When height $\stakeheight$ is reached. \userP{} executes the following steps in order to send the relay request.

%% \begin{figure}[H]
%%   \centering
%%   \procedure[linenumbering]{$\USER_2(\req, \hreq, \rho, \cmstake)$}{%
%%     \stakenullifier \gets \stakecommit( \rho ) \\
%%     (\mkroot, \mkpath) \gets \text{Merkle root and path for $\cmstake$ in $\stakemtree$} \\
%%     \algostyle{in}_\algostyle{pub} \gets (\hreq, \stakenullifier, \stakeheight, \mkroot) \\
%%     \algostyle{in}_\algostyle{aux} \gets (\rho, \mkpath) \\
%%     \stakeproof \gets \text{Proof of $\STAKEREL$ with inputs $(\algostyle{in}_\algostyle{pub}, \algostyle{in}_\algostyle{aux})$} \\
%%     \text{Send $(\req, \stakenullifier, \stakeproof, \stakeheight, \mkroot)$ to \relayEthAccount{}.\addr}
%%   }
%% \end{figure}

%% \begin{todobox}
%% \item Rename these algos
%% \item Use macros instead of algostyle directly
%% \end{todobox}

%% \subsection{Relay Actions}

%% \begin{figure}[H]
%%   \centering
%%   \procedure[linenumbering, syntaxhighlight=auto, addkeywords={abort}]{$\RELAY(\req, \stakenullifier, \stakeproof, \stakeheight, \mkroot)$}{%
%%     \pcif (\req.\fee \ne \relayfee)\ \lor\ (\req.\relayaddr \ne \relayEthAccount.\addr) \pcthen \\
%%     \pcind \pcreturn \false \\
%%     \pcendif \\
%%     \pcif (\ethheight < \stakeheight)\ \lor\ (\ethheight \geq \stakeheight + \stakewindow) \pcthen \\
%%     \pcind abort \\
%%     \pcendif \\
%%     \pcif \text{ $\mkroot$ not seen by \relayexec{} } \pcthen \\
%%     \pcind abort \\
%%     \pcendif \\
%%     \hreq \gets \crh( \req ) \\
%%     \pccomment{TODO: Add verifier algo}\\
%%     \varstyle{valid} \gets \text{ Check $\stakeproof$ with public inputs $(\hreq, \stakenullifier, \stakeheight, \mkroot)$ } \\
%%     \pcif \varstyle{valid} = \false \pcthen
%%     \pcind abort \\
%%     \pcendif \\
%%     \pccomment{At this stage, the relay can be sure he will receive the fee.} \\
%%     %
%%     \relaytx \gets \text{Transaction calling $\relaymethod(\req, \stakenullifier, \stakeproof, \stakeheight, \mkroot)$ on \relayexec{}.} \\
%%     \sendE(\relaytx)
%%   }
%% \end{figure}

%% \begin{todobox}
%%   Add labels and captions to figures in this chapter and add references to images labels to avoid forcing inclusion of floats (via [H])
%% \end{todobox}

%%
%% ---------------------------------------- SAVED TEXT END
%%

% !TEX root = layer2

\chapter{Network structure}\label{chap:unicast-vs-broadcast}

\section{Binding requests to relays}

\subsection{Background}

The protocols presented in this document require users to create relay requests that can only be successfully processed by a specific relay. This serves as a mechanism to prevent other network participants from ``stealing'' the relay requests or front-running relay transactions. The alternative to this would be to support ``free'' relay requests, not bound to specific relays, which could therefore be processed by any participant. If these ``free'' requests are made available (or ``broadcast'') to multiple relays, those relays must then ``race'' to process the request and broadcast a corresponding relay transaction. When the first relay transaction is accepted by the blockchain, the ``winning'' relay will receive the relay fee and later transactions from other relays will be rendered invalid (as a consequence of the nullifiers declared in the $\zeth$ relay request being marked as used).

While such an approach is entirely feasible, it increases the risk for relays, making it much harder for them to hedge against lost fees and wasted compute resources. As a consequence, it becomes much more difficult for relays to assess the risk associated with a given request, which in turn is likely to result in an increase in relay fees. All resources used by ``losers'' of the race are wasted.  In contrast, in the case where requests are bound to specific relays, these resources can be used to process multiple requests in parallel, increasing the efficiency of the system.

\subsection{Emulating ``free'' relay requests}\label{unicast-vs-broadcast:requests:emulate-free-requests}

Despite the mechanisms to prevent front-running, the protocols presented in this document could be leveraged by some user (say $\userP$) to force relays to ``compete'' for relay requests.
Specifically, in order to call the \zeth~mixer \mixer~with parameters $\mixparams$, $\userP$ can run multiple instances of a protocol in parallel, generating $N$ relay requests for $\mixparams$, each targeting a different relay. If the user then sends each of these $N$ requests to the targeted relay, the desired state transition will be carried out by the first relay transaction to be accepted into the blockchain, rendering the remaining $N-1$ requests invalid (by the nullifier mechanism cited previously).

As in the case of ``free'' requests, relays are exposed to extra risk for the reasons given above. However, this ``emulating'' approach does provide partial mitigation of this risk, due to the extra cost that the user $\userP$ must incur. Under the protocols given in this document, in order to create $N$ requests targeting different relays, the user must generate $N$ zk-SNARKs, which is computationally demanding and therefore represents a cost to the user. Therefore, request generation may act as a user-side proof-of-work, preventing malicious messages from flooding the network (as originally designed for by Dwork et al.~\cite{DBLP:conf/crypto/DworkN92,Jakobsson1999ProofsOW}). This naturally leads to the following process by which relays can partially protect against some DoS vectors, by performing the following checks on relay requests:
\begin{enumerate}
    \item Verify that none of the nullifiers in the request has been seen in previously received requests (inspect the mempool and the blockchain state). If one or more nullifiers is a duplicate then reject the request, else proceed.
    \item Verify the zk-SNARK proof in the request. If the proof is invalid, reject the request. Otherwise the request can be considered for processing.
\end{enumerate}

In this way, the cost of generating $N$ proofs imposes some upper bound on the message output rate of a potential attacker.

\section{Unicast vs broadcast networks}\label{unicast-vs-broadcast:unicast-vs-broadcast}

The discussions in this document assume only that some transport mechanism exists for users to send relay requests to specific relays. As noted in \cref{preliminaries:introduction}, users can achieve further anonymity if this transport mechanism does not require the user to reveal any identifying information at the network level. We now discuss some specific implementations of the transport layer (namely ``unicast'' vs ``broadcast'' networks), and their respective properties.

By design, relay requests are bound to specific relays, which intuitively implies a ``unicast'' style transport mechanism. That is, relays publish a network address of some form, and users send requests to this address. Observers of the physical network may determine that a message has been sent from the user to the relay, but the content of the message (i.e.~the details of the relay request) is not visible to other participants. This is a natural choice given that relay request data cannot be used by parties other than the targeted relay. While not a requirement of any of these protocols, unicast channels (in particular point-to-point communication channels, which we assume to be encrypted by default) only reveal the relay request content to the relay itself. Adversaries able to gain control a physical network node along the route between user and relay (in general a limited set of nodes, which varies depending on user and relay) may learn that a message was sent from the user to the relay, but they will not learn anything about the message content. Such communication channels also allow for interaction between relay and user (for example, the relay could dynamically select an \ethereum~or \zeth~address to receive payment, or privately negotiate fees with the user).

Clearly it would be entirely possible to implement these protocols using a ``broadcast'' system, such as those employed by blockchains to propagate transactions and blocks. Requests could be broadcast unencrypted without impacting the reliability of the system, as long as messages were eventually seen by the target relay. Broadcast networks provide some inherent receiver anonymity, in the sense that it is more difficult to identify which \emph{network node} is the recipient of a given message, however in this setting, all participants in the system would be able to see the content of relay requests and potentially determine the number of requests received by each relay identity (and, in turn, infer information about their profit). Despite relay requests being visible to other participants, the protocol would still prevent other relays from profiting from these requests, since they are bound to the target relay. Further, the content of requests could be hidden by encryption so that only the intended relay may read them. Instead of publishing a network ``address'' of some form, relays could publish an encryption key, with which users must encrypt requests before broadcasting them (although care must be taken to use a key-private encryption scheme to avoid leaking information about the recipient).

While broadcast networks could theoretically be used in these relay protocols, unicast networks are more bandwidth efficient (i.e.~a given message needs only to find a path through the network in order to flow from the sender to the recipient). In contrast, broadcast networks may provide a level of sender anonymity in the face of network observers, although even in broadcast networks methods exist to infer the message originator (e.g.~nodes with high degree\footnote{In the graph theoretical sense.} -- also referred to as ``supernodes'' -- can be used to infer the sender of a message on a broadcast channel by using timing information~\cite{DBLP:conf/fc/KoshyKM14}). Broadcast communication channels are also of great interest to achieve ``recipient anonymity''(see~\cite{DBLP:journals/compsec/PfitzmannW87} for more details on ``anonymity'').

At first sight, the use of a unicast transport may appear to increase the centralization of the system. However, this is demonstrably not the case for the protocols discussed here, which can (as described above) be implemented using a transparent broadcast network and do not inherently rely on any centralization.

Finally, we note that, although broadcast networks could theoretically be used, we suggest that unicast networks are likely to be more suitable, given their lower bandwidth and complexity.

\section{Network anonymity}\label{unicast-vs-broadcast:network-anonymity}

Relay protocol designers may choose to transmit requests via the method that best fits their needs, taking into consideration the tradeoffs mentioned in \cref{unicast-vs-broadcast:unicast-vs-broadcast} above. As well as overhead, network topology also has a strong influence on anonymity~\cite{DBLP:conf/pet/DiazMT10}). However, in order to achieve strong privacy guarantees, anonymisation techniques (e.g.~cover traffic, message padding etc.) must be used, to minimize communication leakages.

%% While unicast networks are more bandwidth efficient than broadcast channels (i.e.~a given message needs only to find a path through the network in order to flow from the sender to the recipient), such networks leak the sender and recipient of the communication. Conversely, while using broadcast channels increases the number of messages exchanged on the network, using such communication channels is of great interest to achieve ``recipient anonymity''(see~\cite{DBLP:journals/compsec/PfitzmannW87} for more details on ``anonymity''). However, even in broadcast networks, methods exist to infer the message originator (e.g.~nodes with high degree\footnote{In the graph theoretical sense.} -- also referred to as ``supernodes'' -- can be used to infer the sender of a message on a broadcast channel by using timing information~\cite{DBLP:conf/fc/KoshyKM14}).

%% Thus, as well as appropriate network topology (which as we saw has a strong influence on both overhead and anonymity~\cite{DBLP:conf/pet/DiazMT10}), achieving strong privacy guarantees on a network requires the use of anonymisation techniques (e.g.~cover traffic, message padding etc.) to minimize communication leakages.

While protocols like Dandelion~\cite{DBLP:journals/pomacs/Venkatakrishnan17,DBLP:journals/pomacs/FantiVBDBMV18} were initially introduced to improve \emph{diffusion} mechanisms and improve network anonymity on Bitcoin, they could also be of interest in the context of relay request broadcasts (as alluded to in~\cref{unicast-vs-broadcast:unicast-vs-broadcast}).
However, other techniques (providing different properties) may also be of interest for relay network. Some of these are given below:
\begin{itemize}
    \item DC-nets~\cite{DBLP:journals/joc/Chaum88} provide strong guarantees with respect to the sender anonymity (but generally incur a big overhead and require large amounts of randomness).
    \item Crowds~\cite{DBLP:journals/tissec/ReiterR98} follow a ``blending into a crowd'' approach (i.e.~hiding one's actions among the actions of many others), in which a user's request is randomly circulated in a ``crowd'' (set of users) before being submitted -- by a random member of the crowd -- and sent to the destination\footnote{Crowd members \emph{cannot} identify the initiator of the request. The initiator is indistinguishable from a member that forwards a request from another user.}. Note that such approaches generally do not provide strong guarantees with respect to recipient anonymity\footnote{While relay anonymity is not our principal focus, it is worth keeping in mind the impact of side channel leakages which can be used to infer information about the sender. For instance, a powerful adversary -- monitoring a big part of the Internet -- may notice a client access the relay's public information (such as the relay's website) followed by a message to the relay from a crowd to which the client belongs. The adversary may then infer that the client was the relay user. Hence, additional care needs to be allocated to the relay discovery mechanism itself, and the right trade-offs must be made depending on the application and associated threat model.}.
    \item Mix networks (or \emph{mixnets})~\cite{DBLP:series/ais/Chaum03}, in which nodes (``mix nodes'') are routers that perform cryptographic operations (providing bit-wise unlinkability), and modify the order in which output messages are emitted. This hides any correspondence between input and output messages.
    \item Onion routing~\cite{DBLP:journals/cacm/GoldschlagRS99} (which also underlies ``garlic routing''~\cite{garlic-nets}) consists of multiple layers of encryption (one per ``hop'' on the network). Requests are sent through a chosen set of routers (forming a ``circuit'') in order to obfuscate the link between sender and recipient, as seen by non-global adversaries (i.e.~those that do not control all nodes on the circuit\footnote{In some cases, controlling the ``entry'' and ``exit'' nodes (i.e.~first and last nodes of the chain/circuit) is sufficient to carry out so-called ``correlation attacks''. See~\url{https://github.com/Attacks-on-Tor/Attacks-on-Tor} for a list of attacks on Tor~\cite{DBLP:conf/uss/DingledineMS04}}). This generally achieves low-latency relative to other approaches.
\end{itemize}

Importantly, modern protocols building on these techniques use additional mechanisms for enhanced robustness (e.g.~``cover traffic'' to prevent timing attacks etc.).

\begin{remark}
    We note that accountable anonymous communication networks~\cite{DBLP:books/sp/07/DiazP07} are also of great interest in the context of transaction relay protocols as a way to further prevent DoS attacks.
\end{remark}


\bibliographystyle{alpha}
\bibliography{references}

\end{document}
