% !TEX root = ../zeth-protocol-specification.tex

\section{Processing $\zethTx$}\label{zeth-protocol:process-tx}

When a $\zethTx$ is mined (hence assuming that $\ethVerifyTx(\zethTx)$ returns $\true$), the state transition specified by the $\mix$ function of $\mixer$ is executed.

To preserve the soundness of \zeth, and make sure that no $\zparty{U}$ is able to create value by double spending $\zethnotes$, various checks need to be satisfied. The function $\zethVerifyTx$ is defined as the function that returns $\true$ if all the checks are satisfied, and $\false$ otherwise.

If $\zethVerifyTx(\zethTx)$ returns $\true$, then $\mix$ modifies the ``World state'' $\wstate$ to account for the spent $\zethnotes$ and the newly generated ones. However, if $\zethVerifyTx(\zethTx)$ returns $\false$, then the state transition ends.

\begin{notebox}
    Even if $\zethVerifyTx(\zethTx)$ returns $\false$, $\wstate$ is modified since the \ethereum~balances of the transaction originator is decremented by the sum of $\txDefaultGas$ and the gas consumed by the $\zethVerifyTx$ function, and the balance of the \ethereum~account of the miner gets incremented by the same amount.
\end{notebox}

Thus, $\mix$ proceeds as follows:
\begin{enumerate}
    \item Check that all the values of the primary inputs' ($\inp{\mix}.\primInp$) entries are elements of the scalar field over which the zk-proof is generated:
        \[
            \inp{\mix}.\primInp \in \FFx{\rCURVE}^*
        \]
    \item Unpack the nullifiers, signature hash and public values (see~\cref{instantiation:statement:pack} for the definitions of the $\unpack{}{}$ functions):
        \begin{align*}
            \nf{i} &= \unpack{\inp{\mix}.\primInp.\nfs{i}, \inp{\mix}.\primInp.\resbits}{\nf{}}\ \forall i \in [\jsin] \\
            \vin &= \decode{\unpack{(), \inp{\mix}.\primInp.\resbits}{\vin}}{\NN} \\
            \vout &= \decode{\unpack{(), \inp{\mix}.\primInp.\resbits}{\vout}}{\NN} \\
            \hsig &= \unpack{\inp{\mix}.\primInp.\hsig, \inp{\mix}.\primInp.\resbits}{\hsig}
        \end{align*}
    \item Check the validity of the $\zethTx$ object ($\zethVerifyTx$):
        \begin{enumerate}
            \item Check that $\inp{\mix}.\primInp.\hsig$ is correctly computed, i.e.~check that the following equation holds (to prevent transaction malleability, see~\cref{appendix:trnm}):
                \[
                    \hsig = \crhhsig{\inp{\mix}.\primInp.\nfs{}, \inp{\mix}.\otsvk}
                \]
            \item\label{zeth-protocol:process-tx:verify-zkp} Check that $\zkp$ is a valid zk-SNARK proof for $\inp{\mix}.\primInp$, i.e.~check that:
                \[
                    \zksnark.\verifier(\pparams_{\zksnark}, \zkp, \inp{\mix}.\primInp) = \true
                \]
            \item Check that none of the nullifiers in $\inp{\mix}.\primInp.\nfs{}$ have already been used, i.e.~check that:
                \[
                    \nf{i} \not\in \nullifierset, \forall i \in [\jsin]
                \]
                where $\nullifierset$ is the set of all nullifiers that are ``declared'' on $\mixer$.
            \item\label{zeth-protocol:process-tx:check-otsig} Check that $\inp{\mix}.\otssig$ is a valid signature of the $\ethereum{}$ sender's address $\addr$ (see~\cref{zeth-protocol:create-tx}) and the attributes of $\inp{\mix}$, to prevent transaction malleability (see~\cref{appendix:trnm}), i.e.~check that:
                \begin{align*}
                    & \otsigscheme.\verify(\inp{\mix}.\otsvk, \msg, \inp{\mix}.\otssig) = \true \\
                    & \text{where} \ \datatobesigned = \addr \concat \inp{\mix}.\primInp \concat \inp{\mix}.\zkp \concat \inp{\mix}.\ciphers, \\
                    & \text{and} \ \msg = \crhots{\datatobesigned}
                \end{align*}
            \item Check that $\inp{\mix}.\primInp.\mkroot$ corresponds to a valid state of the Merkle tree held on $\mixer$, i.e.~check that:
                \[
                    \inp{\mix}.\primInp.\mkroot \in \rootset'
                \]
                where $\rootset'$ is the set of all Merkle roots corresponding to one of the states of the Merkle tree.
            \item Check that $\vin$ corresponds to the value $\val$ of the transaction object, i.e.~check that:
                \[
                    \vin = \zethTx.\val
                \]
        \end{enumerate}
    \item If all checks above pass, i.e.~if $\zethVerifyTx(\zethTx)$ returns $\true$, then the following additional modifications are made in $\wstate$:
        \begin{enumerate}
            \item Add the commitments $\inp{\mix}.\primInp.\cms{}$ to the Merkle tree held on $\mixer$.
            \item $\rootset' \gets \rootset' \cup \smallset{\mkroot'}$, where $\mkroot'$ is the Merkle root of the Merkle tree after insertion of the commitments $\inp{\mix}.\primInp.\cms{}$ in the Merkle tree.
            \item $\nullifierset \gets \nullifierset \cup \indexedset{\nf{i}}{i \in [\jsin]}$, i.e.~the nullifiers $\nfs{}$ become ``declared''.
            \item Modify the \ethereum~balances according to the public values:
                \begin{itemize}
                    \item $\wstate[\eparty{S}.\addr].\balance = \wstate[\eparty{S}.\addr].\balance - \vin$
                    \item $\wstate[\eparty{S}.\addr].\balance = \wstate[\eparty{S}.\addr].\balance + \vout$
                    \item $\mixer.\balance = \mixer.\balance + \vin$
                    \item $\mixer.\balance = \mixer.\balance - \vout$
                \end{itemize}
            \item Emit an event (\cref{sssec:ethereum-events}) $\evMixOut$ of type \mixEventDType, containing the new root $\mkroot'$ of the Merkle tree of commitments, the nullifiers $\indexedset{\nf{i}}{i \in [\jsin]}$, commitments to the newly created \zethnotes $\inp{\mix}.\primInp.\cms{}$, and the corresponding ciphertexts $\inp{\mix}.\primInp.\ciphers$.
        \end{enumerate}
\end{enumerate}

\begin{remark}\label{zeth-protocol:process-tx:dispatch-call}
  In some deployments, verification of the zk-SNARK proof may be delegated to some trusted external mechanism, and the proof $\zkp$ may not be available to the mixer contract.  (As an example, consider the case where multiple \zeth~transactions are aggreagted by a system such as that described in \cite{rondelet2020zecale}. The multiple zk-SNARK proofs become auxiliary inputs to a ``wrapping'' zk-SNARK which checks their validity via a single proof verification. A modified version of the \mix~function receives \mix~parameters from a trusted contract without $\inp{\mix}.\zkp$.)

  In this case, the value of $\datatobesigned$ in \cref{zeth-protocol:process-tx:check-otsig} may be replaced by:
  \[
  \datatobesigned = \addr \concat \inp{\mix}.\primInp \concat \inp{\mix}.\ciphers, \\
  \]
  and the equivalent change must be made when generating the \mix~parameters, as described in \cref{zeth-protocol:mix-inp:otsig-for-dispatch-call}. The scheme used to verify the zk-SNARK proof must specify exactly how the contract should be modified, including any further checks that must be carried out. In the proof-of-concept \zeth~implementation, a modified entry point is included (alongside the regular \mix~entry point), which:
  \begin{itemize}
  \item Checks that the \mixer~contract has been deployed with the (immutable) address of a trusted contract, permitted to call this entry point. Otherwise the contract aborts.
  \item Check that the caller $msg.sender$ matches the permitted caller, set at deployment time, and aborts otherwise.
  \item Performs all checks related to the \mix~parameters, except \cref{zeth-protocol:process-tx:verify-zkp}, and the modification to \cref{zeth-protocol:process-tx:check-otsig} described in this remark.
  \end{itemize}

  Further, the \zeth~client implementation includes a flag to enable the corresponding change described in \cref{zeth-protocol:mix-inp:otsig-for-dispatch-call} (to generate a signature on the modified $\datatobesigned$). Naturally, the \emph{dispatch} entry point can only be used with parameters generated using this flag (otherwise the signature check will fail).

  Although this modification is not part of the core protocol, we mention it here for completeness and to justify the associated code in the proof-of-concept implementation. Such a modification \MUSTNOT{} be implemented except as described by the secure external scheme for delegating proof verification.
\end{remark}
