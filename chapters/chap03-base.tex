% !TEX root = ../zeth-protocol-specification.tex

\chapter{Instantiation of the cryptographic primitives}\label{chap:instantiation}

In this chapter, we start by instantiating the cryptographic building blocks used in previous sections to describe the \zeth~\dapscheme~design. Finally, we proceed by providing security proofs justifying that our instantiation complies with the security requirements listed in previous sections.

Note that, in several cases, it is necessary to specify details in terms of concrete properties of the curve \Curve~and associated scalar field $\FFx{\rCURVE}$. In these cases, we focus on two curves of interest: \BNCurve~and \BLSCurve. We note, however, that other suitable curves could be used.

\BNCurve~\cite{bn-prime} has several properties that make it implementation-friendly. Elements of both the base field and scalar field can be represented in \ethWordLen~bits (the native word size of the \evm), allowing efficient encoding and manipulation of such elements. Moreover, a subset of operations on \BNCurve are supported by the \evm~through precompiled contracts. These precompiled contracts enable verification of signatures (\cref{instantiation:otsig}) and zero-knowledge proofs (\cref{instantiation:zksnark}), required by this protocol, with minimal gas overhead.

\BLSCurve~\cite{bowe18zexe}, like \BNCurve, has the advantage that scalar field elements can be represented within \ethWordLen-bit words (although the same is not true of base field elements). However, the \evm~provides no native support for \BLSCurve, which increases the complexity of the \mixer~implementation (see \cref{zeth-protocol:process-tx}~for details of the operations to be performed). An advantage that \BLSCurve does provide, is that is it the ``inner'' curve of a one-layer chain (as described in \cite{bowe18zexe,housni2020onelayer}). Therefore zero-knowledge proofs using \BLSCurve can be efficiently verified by statements in other zero-knowledge proofs using an approporiate ``outer'' pairing. Support for \BLSCurve in \zeth~therefore admits several applications (no explicitly covered by this document), such as aggregation of proofs over multiple \zeth~transactions (e.g.~\cite{rondelet2020zecale}).

Further details related to implementation and optimization are given in \cref{chap:implementation}.

% !TEX root = ../zeth-protocol-specification.tex

\section{Instantiating the \prf{}s, \comm~and \crh{}s}\label{instantiation:prf-comm-crh}

%We use a one-time Schnorr-based signature scheme by Bellare and Shoup~\cite{bellare2007two} (see:~\cref{ss:ot-schnorr}) with \sha{256} as collision resistant hash function and the subgroup of prime order $\rBN$ of \BNCurve~(see:~\cref{ssec:notations}). As such, we have $\sk \in \FFx{\rBN}^2$ (written over $2 \cdot 32$ bytes), $\vk \in \gset^2$ (uncompressed, written over $4 \cdot 32$ bytes) and $\sigma \in \FFx{\rBN}$. This signature scheme is \sufcma~which is stronger than \ufcma, hence the security requirement is fulfilled.

The functions \crhhsig{} and \crhots{} are instantiated with \sha{256}~\cite{fips1804} which we assume to be collision resistant. Furthermore, $\comm$, $\prfpk{}{x}$, $\prfrho{}{x}$, $\prfaddr{}{x}$, and $\prfnf{}{x}$ are all instantiated with \blake{2}{}'s hash function optimized for 32-bit platforms, \blake{2s}{}, which we prove in the Weakly Ideal Cipher Model~\cite{luykx2016security} to be from a family of PRF and collision resistant functions. The Weakly Ideal Cipher model assumes that \blake{2}{}'s underlying block cipher is ideal, has a distinguisher but no structural weaknesses (see:~\ref{appendix:blake:proofs}). In addition to that, and to ensure that the functions $\prfpk{}{x}$, $\prfrho{}{x}$, $\prfaddr{}{x}$, and $\prfnf{}{x}$ compute images lying in different domains, we use different message prefixes (or ``domain separators'') for the $\prf{}s$ inputs. This approach ensures that the $\apk_{i}$'s, $\nf{i}$'s, $\rrho_{i}$'s, and $\htag{i}$'s have independent distributions from a \ppt{} adversary point of view.

\begin{notebox}
    It is important to note that, for this approach to be secure, the hash function used needs to be secure against \emph{chosen-prefix collisions attacks}~\cite{md5-collision}.
\end{notebox}

Furthermore, we take:
\begin{itemize}
    \item $\noterLen, \askLen, \phiLen = \blakeCompLen$
\end{itemize}

\subsection{Blake2 primitive}\label{instantiation:prf-comm-crh:blake}

\blake{}{}~\cite{aumasson2008sha} is a hash family that was presented as a candidate at the \sha{3} competition. \blake{2}{} is the next iteration of the family which has been further optimized to achieve higher throughput thanks to some optimizations and by being less conservative on its security\footnote{The authors increased the number of rounds of \blake{}{} for the \sha{3} competition to be more conservative on security. They however showed afterwards that this change was not ``meaningfully more secure'' and thus reverted it for \blake{2}{} (see:~\cite[Section 2.1]{aumasson2013blake2}).}. \blake{}{} and \blake{2}{} are based on the \chacha{} stream cipher~\cite{bernstein2008chacha} composed with the \haifa{} framework~\cite{biham2007framework}. \chacha{} defined over 20 rounds, as used in \blake{2}{}, is deemed secure and a PRF based on today's cryptanalysis~\cite{procter2014security,choudhuri2016differential}. \blake{2}{} is specified in RFC-7693~\cite{blakecompietf} and licensed under CC0. \blake{2s}{} is an instantiation of \blake{2}{} optimized for 32-bit platforms. As such, to reason about the security of \blake{2s}{} we prove the security of \blake{2}{}.

\paragraph*{Blake Security}

\blake{}{} security has been heavily scrutinized through the \sha{3} competition~\cite{vidali2010collisions, ming2010security, andreeva2010security, alshaikhli2012comparison, andreeva2012security, andreeva2012provable, homsirikamol2012security}. \blake{2}{} has also been thoroughly cryptanalyzed independently~\cite{guo2014analysis, hao2014boomerang, espitau2015higher, neves2019observation}. For $n$-bit long digests/outputs, the hash and compression functions present $n/2$-bit of collision resistance and $n$-bit of pre-image resistance, immunity to length extension, and indifferentiability from a random oracle~\cite{aumasson2013blake2}. They have furthermore been demonstrated secure in the Weakly Ideal Cipher Model~\cite{luykx2016security} (WICM, see~\cref{appendix:blake:secmod:WICM}). More particularly, Luykx et al.~show that \blake{2}{}, is indifferentiable from a random oracle in this model and is a PRF.

\begin{notebox}
	We assume that the encryption scheme used in the \blake{2}{} underlying compression function --- which is derived from \chacha{20} --- has no exploitable structural behavior. More precisely, that this encryption scheme behaves like a weak ideal cipher. We provide proofs in this model.
\end{notebox}

We use that result in~\cref{appendix:blake:proofs} to show the collision resistance. We also prove that, given that \blake{2}{} is collision resistant and a PRF, $\blake{2}{r \concat x}$ is computationally binding and computationally hiding commitment scheme for input $x$ and randomness $r$.

\subsection{Commitment scheme}\label{instantiation:prf-comm-crh:comm}

We define our commitment scheme as follows,
\begin{align*}
	& \comm.\setup &&: \smallset{\secparam\ s.t\ \secpar \in \NN} \to \BB^{*} \\
	& \comm.\commit{}{} &&: \left( \BB^\prfAddrOutLen \times \BB^\prfRhoOutLen \times \BB^\zvalueLen \right ) \times \BB^\noterLen \to \FFx{\rBN}
\end{align*}

We instantiate the commitment scheme with $\blake{2s}{}$ as follows,
\begin{align*}
	\pparams &= \comm.\setup(\secparam)\ (\text{corresponds to \blake{2s}{}'s constant } \blakePB{}\ \text{and}\ \rBN) \\
	\cm{} &= \comm.\commit{\msg = (\apk, \rrho, \notev)}{\noter} \\
	&= \decode{\blake{2s}{\noter \concat \apk \concat \rrho \concat \notev}}{\NN} \pmod{\rBN}
\end{align*}

\begin{remark}
We set the commitment digest length in the parameter block $\blakePB{}$~\cite{blakecompietf}.
\end{remark}

\subsubsection{Security proof.}\label{instantiation:prf-comm-crh:comm:sec-proof}

The commitment scheme defined above is computationally hiding and binding in the WICM, see~\cref{appendix:blake:full-comm}. However, because of the modulo $\rBN$ operation, the scheme is only $(\bnFieldBitLen / 2)$-bit binding.

\subsection{PRFs}\label{instantiation:prf-comm-crh:prf}

We show in this section how we instantiate the \prf{}s with Blake primitives. As a reminder, the \prf{}s are defined as follows,
\begin{align*}
 	& \prfaddr{}{} : \BB^\askLen \times \smallset{0} \to \BB^\prfAddrOutLen \\
	& \prfpk{}{} : \left ( \BB^\askLen \times  [\jsin] \right ) \times \BB^\crhhsigOutLen \to \BB^\prfPkOutLen \\
	& \prfnf{}{} : \BB^\askLen  \times \BB^\prfRhoOutLen \to \BB^\prfNfOutLen \\
	& \prfrho{}{} :\left ( \BB^\phiLen \times  [\jsout] \right ) \times \BB^\crhhsigOutLen  \to \BB^\prfRhoOutLen
\end{align*}

As we instantiate the \prf{s} with \blake{2s}{}, we have,
\begin{align*}
	\prfAddrOutLen, \prfNfOutLen, \prfPkOutLen,  \prfRhoOutLen = \blakeCompLen
\end{align*}

To ensure that the \prf{}s have independent distributions, we first introduce tagging functions $\tagfunction^x$ which truncate and preppend with a distinct tag the \prf{}s key. We have,
\begin{align*}
	& \tagaddr{} : \BB^\askLen \to \BB^\blakeCompLen \\
	& \tagpk{}{} : \BB^\askLen \times [\jsin] \to \BB^\blakeCompLen  \\
	& \tagnf{} : \BB^\askLen \to \BB^\blakeCompLen \\
	& \tagrho{}{} : \BB^\phiLen \times [\jsout] \to \BB^\blakeCompLen
\end{align*}

The tagging functions are instantiated as follows,
\begin{align*}
	\tagaddr{\auxinputs.\jsins{i}.\ask} &= \taggedaddr \\
	&= (1) \concat {(1)}^{\ceil{\frac{\jsmax}{2}}} \concat (0,0) \concat \trunc{\auxinputs.\jsins{i}.\ask}{\blakeCompLen-3-\ceil{\frac{\jsmax}{2}}} \\
	%
	\tagnf{\auxinputs.\jsins{i}.\ask} &= \taggednf \\
	&= (1) \concat {(1)}^{\ceil{\frac{\jsmax}{2}}} \concat (1,0) \concat \trunc{\auxinputs.\jsins{i}.\ask}{\blakeCompLen-3-\ceil{\frac{\jsmax}{2}}} \\
	%
	\tagpk{i}{\auxinputs.\jsins{i}.\ask} &= \taggedpk \\
	&= (0) \concat \pad{\encode{i}{\NN}}{\ceil{\frac{\jsmax}{2}}} \concat (0,0) \concat \trunc{\auxinputs.\jsins{i}.\ask}{\blakeCompLen-3-\ceil{\frac{\jsmax}{2}}} \\
	%
	\tagrho{j}{\auxinputs.\pphi} &= \taggedrho \\
	&= (0) \concat \pad{\encode{j}{\NN}}{\ceil{\frac{\jsmax}{2}}} \concat (1,0) \concat \trunc{\auxinputs.\pphi}{\blakeCompLen-3-\ceil{\frac{\jsmax}{2}}}
\end{align*}

We now present how the \prf{}s are instantiated,
\begin{align*}
	\prfaddr{\auxinputs.\jsins{i}.\ask}{0} &= \auxinputs.\jsins{i}.\znote.\apk \\
	&= \blake{2s}{\tagaddr{\auxinputs.\jsins{i}.\ask} \concat \pad{0}{\blakeCompLen}} \\
	%
	\prfnf{\auxinputs.\jsins{i}.\ask}{\auxinputs.\jsins{i}.\rho} &= \priminputs.\nfs{i} \\
	&= \blake{2s}{\tagnf{\auxinputs.\jsins{i}.\ask} \concat \auxinputs.\jsins{i}.\znote.\rho} \\
	%
	\prfpk{\auxinputs.\jsins{i}.\ask}{i, \priminputs.\hsig} &= \priminputs.\htags{i} \\
	&= \blake{2s}{\tagpk{i}{\auxinputs.\jsins{i}.\ask} \concat \priminputs.\hsig} \\
	%
	\prfrho{\auxinputs.\pphi}{j, \priminputs.\hsig} &= \auxinputs.\znotes{j}.\rrho \\
	&= \blake{2s}{\tagrho{j}{\auxinputs.\pphi} \concat \priminputs.\hsig}
\end{align*}

\begin{remark}
	We set the \prf{}s' output length in the \blake{2s}{}'s parameter block $\blakePB{}$.
\end{remark}

\subsubsection{Security proof.}\label{instantiation:prf-comm-crh:prf:sec-proof}

The functions defined above are collision resistant and PRFs in the WICM, see~\cref{appendix:blake:proofs}.
Because of the tagging functions, the security parameter of the \prf{}s becomes $\secpar = \blakeCompLen/2 - \jsmax/4 - 3/2$.

\subsection{Collision Resistant Hashes}\label{instantiation:prf-comm-crh:crh}
We instantiate in this section the collision resistant hash functions $\crhhsig{}$ and $\crhots{}$ with \sha{256}. As a consequence, we have,
\[
    \crhhsigOutLen = \crhotsOutLen = \shaTwoDigestLen
\]

\paragraph*{\sha{256} Security}
SHA-256 (Secure Hash Algorithm 256) is a hash function designed by the National Security Agency (NSA) in 2001. It is based on the Merkle–Damgård structure, the Davies–Meyer compression function construct~\cite[Function f5 in Figure 3]{black2002black} and the classified SHACAL-2 block cipher.

Collision attacks have been thoroughly studied by the research community~\cite{sanadhya2008new,mendel2011finding}. The best attacks at this day, are second-order differential attack by Lamberger et al.~\cite{lamberger2011higher} on the SHA-256 compression function reduced to 46 out of 64 rounds.

Many researchers~\cite{isobe2009preimage,aoki2009preimages} have also studied preimage attacks on SHA-256 with reduced rounds. Guo et al.~\cite{guo2010advanced} in particular were among the first to use the meet in the middle strategy~\cite{aoki2009meet} and achieved more efficient ones on 42-step SHA-256. Khovratovich et al.~in 2012~\cite{khovratovich2012bicliques} have so far presented the best preimage attacks, on 45-round and 52-round SHA-256 as well as a 52-round attack on the SHA-256 compression function.

Li et al.~have published in 2012~\cite{li2012converting} a noteworthy paper on converting meet in the middle preimage attack into pseudo collision attack. Using preimage attacks by bicliques, they found pseudo collisions  attacks on 52 steps of SHA-256.

\begin{lemma}
	\sha{256} is $128$-bit collision resistant.
\end{lemma}
 % PRFs, COMM and CRHs (may need to be split if we decide to instantiate these with different functions)
% !TEX root = ../zeth-protocol-specification.tex

\section{Instantiating $\mkhash$}\label{instantiation:mkhash}

In this section we describe the instantiation of $\mkhash$ with a compression function based on \mimc{}~\cite{albrecht2016mimc}. We firstly show how the compression function is constructed, and prove that this instantiation complies with the security requirements mentioned in~\cref{zeth-protocol:sec-req}

\subsection{\mimc{} Encryption}\label{instantiation:mkhash:mimc-encryption}

\mimc{} is a block cipher with a simple design, consisting of a number of rounds (denoted $\rounds$). During the $i$-th round, the message $\msg$ is mixed with the encryption key $\key{}$ and a randomly chosen constant $c[i]$, and a permutation function is applied to generate a new value of $\msg$. The permutation function consists of exponentiation with a carefully chosen exponent $\exponent{}$ (see~\cref{instantiation:mkhash:mimc-encryption:security}). Note that \rounds{} depends on the desired security level $\secpar$. We denote the encryption function by \mimcEnc{} and illustrate it in~\cref{instantiation:fig:mimc}.

\begin{figure*}[ht]
    \centering
    \procedure[linenumbering]{$\mimcEnc(\key, \msg, c, \exponent{}, \rounds)$}{
        \pcforeach\ i \in [\rounds]:\\
        \t \msg \gets {(\key\ \mathsf{OP}\ c[i]\ \mathsf{OP}\ \msg)}^\exponent{}\\
        \pcreturn (\msg\ \mathsf{OP}\ \key)
    }
    \caption{\mimc{} Encryption function.}\label{instantiation:fig:mimc}
\end{figure*}

\mimcEnc{} can be defined on both binary and prime fields, and as such the $\mathsf{OP}$ operation corresponds to either $\oplus$ or $+ \pmod{p}$~\cite{albrecht2016mimc, grassi2016mpc}.
For general prime $p$ (resp.~ positive integer $n$), we denote by $\mimcPrime{p}$ (resp.~$\mimc{}_{2^n}$) the \mimcEnc{} function defined over \FFx{p} (resp.~\FFx{2^n}). In this document, we only consider \mimc{} defined over prime fields (in particular, the field $\FFx{\rCURVE}$ with elements written over $\secpar$ bits, over which \zksnark~operates).

Since block ciphers are usually defined over the product space of keys and messages, we consider the variables $c$, \rounds{} and \exponent{} as fixed. We thereby consider an instantiation of $\mimc{}$ with signature
\begin{align*}
    \mimcPrime{\rCURVE} &: \FFx{\rCURVE} \times \FFx{\rCURVE} \to \FFx{\rCURVE}
\end{align*}

\subsubsection{Security parameters and analysis}\label{instantiation:mkhash:mimc-encryption:security}

To ensure that the exponentiation leads to a permutation in $\FFx{\rCURVE}$, we
consider $\exponent{}$ such that $\gcd(\exponent{}, \rCURVE-1) = 1$. To achieve
a security of $\secpar$, we require that $\rounds = \left\lceil \frac{\log_2
    \rCURVE}{\log_2 \exponent{}} \right\rceil$. For the efficiency reasons we
consider exponents of form $2^t \pm 1$.

We refer to the $\mimc{}$ paper~\cite[Section 4.2 and 5.1]{albrecht2016mimc} for more details on the security analysis and attacks on the scheme. Note that $\mimcPrime{\rCURVE}$ does not suffer from \emph{inversion subfield attacks} as there are no proper subfields of $\FFx{\rCURVE}$.

We note that Albrecht et al.~\cite[Section 5.3]{albrecht2016mimc} advised agains
using exponents of form $2^t + 1$ for MiMC over $\FFx{2^n}$ as in that case
polynomials $(x + y)^{2^t + 1} \in \FFx{2^n}$ are sparse (i.e.~most of their
coefficients are zero, here, only 4 coefficients are nonzero). However in case
of $\FFx{\rCURVE}$, where $\rCURVE > \binom{\exponent}{\lfloor \exponent / 2
  \rfloor}$, polynomials $(x + y)^{\exponent}$ are not sparse. This comes from
the following observation.

\[
  (x + y)^\exponent = \sum_{i = 0}^{\exponent} \binom{\exponent}{i} x^{i}y^{\exponent - i}
\]
and $\binom{\exponent}{i} \bmod \rCURVE = \binom{\exponent}{i}$ as $\binom{\exponent}{i} < \rCURVE$, thus all the coefficients are greater than $0$.

\subsection{\mimc{}-based compression function}\label{instantiation:mkhash:mimc-compressionf}

There exist two main techniques to construct a hash function from a block-cipher (or permutation): sponge functions~\cite{bertoni2007sponge} and iterated compression functions~\cite{black2002black}.

A Merkle tree is a binary tree of values of fixed size, where the values in each ``layer'' are generated by hashing pairs of values from the previous ``layer''. That is, we require a compression function $\mkhash$, which we construct via the Miyaguchi-Preneel scheme. (Miyaguchi-Preneel is more secure~\cite[$f_5$ function]{black2002black} than the more flexible Davies-Meyer construct~\cite[Section 3]{gazzoni2006maelstrom}, but this flexibility is not required in our case).

\subsubsection{Miyaguchi-Preneel compression construct}

Miyaguchi-Preneel (MP)~\cite[$f_3$ function]{black2002black} is a general scheme for constructing compression functions from block ciphers (see~\cref{preliminaries:definitions:hashcomp}). Given a block cipher \Enc, the corresponding compression function by \fMP{} is given in~\cref{instantiation:fig:mp-constructions}. The original construction is defined over binary fields, however \zeth~operates over prime fields. Hence, in the general discussion here we replace the bitwise addition operator $\oplus$ by modular addition in $\FFx{\rCURVE}$ (see~\cite{mp-security-ethsnarks}).

We denote by \mimcMP{} the compression function defined by the application of the Miyaguchi-Preneel construct over \mimc{}. Similarly, for general prime $p$ we denote by $\mimcMPPrime{p}$ (see~\cref{instantiation:fig:mimc-mp-constructions}) the compression function defined by application of the Miyaguchi-Preneel construct over $\mimcPrime{p}$.

\begin{figure*}[ht]
    \begin{minipage}[t]{0.50\textwidth}
        \procedure[linenumbering]{$\fMP{} (\key, \msg)$}{
            res \gets \Enc_\key(\msg) \\
            \pcreturn (res + \msg + \key) \pmod{\rCURVE}
        }
        \caption{\MP{} construct in $\FFx{\rCURVE}$.}\label{instantiation:fig:mp-constructions}
    \end{minipage}%
    \begin{minipage}[t]{0.50\textwidth}
        \procedure[linenumbering]{$\mimcMPPrime{\rCURVE}(\key, \msg)$}{
            res \gets \mimcPrime{\rCURVE}(\key, \msg) \\
            \pcreturn (res + \key + \msg) \pmod{\rCURVE}
        }
    \caption{$\mimcMPPrime{\rCURVE}$ construction.}\label{instantiation:fig:mimc-mp-constructions}
    \end{minipage}%
\end{figure*}

\subsection{An efficient instantiation of \mimc{} primitives}\label{instantiation:mkhash:efficient-instance}

To select appropriate instances of $\mimcPrime{\rCURVE}$ and $\mimcMPPrime{\rCURVE}$, we consider the cost (in terms of gas consumption and prover efficiency). For given $\exponent{}$ and $\rounds{}$, the final definition of $\mimcMPPrime{\rCURVE}$ is given in \cref{instantiation:fig:mimcp-construction} and \cref{instantiation:fig:mimcp-mp-construction}.

\newcommand{\initRoundConstants}{\algostyle{InitRoundConstants}}

\begin{figure*}[ht]
    \begin{minipage}[t]{0.5\textwidth}
        \centering
        \procedure[linenumbering]{$\mimcPrime{\rCURVE}(\key, \msg)$}{
            c \gets \initRoundConstants() \\
            \pcforeach i \in [\rounds]:\\
            \t \msg \gets {(\key + c[i] + \msg)}^\exponent{} \pmod{\rCURVE}\\
            \pcreturn (\msg + \key) \pmod{\rCURVE}
        }
    \end{minipage}%
    \begin{minipage}[t]{0.5\textwidth}
        \centering
        \procedure{$\initRoundConstants()$}{
            iv \gets \keccak{256} (\text{``clearmatics\_mt\_seed''}) \\
            c[0] \gets 0 \\
            c[1] \gets \keccak{256} (iv) \\
            \pcforeach i \in \range{2}{\rounds}:\\
            \t  c[i] \gets \keccak{256} (c[i-1])\\
            \pcreturn c = (c[0], \ldots, c[\rounds-1])
        }
    \end{minipage}%
    \caption{$\mimcPrime{\rCURVE}$ full construction}\label{instantiation:fig:mimcp-construction}
\end{figure*}

\begin{figure*}[ht]
    \centering
    \begin{minipage}[t]{0.5\textwidth}
        \procedure{$\mimcMPPrime{\rCURVE}(\key, \msg)$}{
            \pcreturn \mimcPrime{\rCURVE}(\key, \msg) + \msg + \key \pmod{\rCURVE}
        }
    \caption{\mimcMPPrime{\rCURVE} full construction}\label{instantiation:fig:mimcp-mp-construction}
    \end{minipage}%
\end{figure*}

\begin{remark}
    Note that \keccak{256} is the $256$-bit digest instance of the \keccak{} family that won the NIST SHA-3 competition~\cite{keccak-submission}. It is supported by the \evm via an opcode (see~\cite[Appendix G]{wood2014ethereum}), making it convenient for use in smart contracts.
\end{remark}

\begin{remark}
    To increase the security of the $\mathsf{\mkhash}$, different round constants for each level of the Merkle tree could be used.
\end{remark}

We define $\mkhash$ to be $\mimcMPPrime{}$ over $\FFx{\rCURVE}$. Thereby, for input values $m_0$ and $m_1$, $\mkhash : \FFx{\rCURVE} \times \FFx{\rCURVE} \to \FFx{\rCURVE}$ is defined by
\begin{equation}\label{instantiation:eq:mkhash-instantiation}
    \mkhash(\msg_0, \msg_1) = \mimcMPPrime{\rCURVE}(m_0, m_1)
\end{equation}

For specific values of $\rCURVE$ (such as $\rBN$ for $\BNCurve$ or $\rBLS$ for $\BLSCurve$), it remains to choose concrete values of $\exponent{}$ and $\rounds{}$.

\newcommand{\constraints}{\varstyle{constraints}}

Note that small exponents $\exponent{}$ result in fewer constraints in the arithmetic circuit (see~\cref{zeth-protocol:statement}), while larger exponents can reduce the cost of Merkle tree operations on the contract (see~\cref{zeth-protocol:process-tx}). This is due to two factors, namely that exponentiation is cheaper to execute on a contract than in an arithmetic circuit, and that the number of rounds decreases with higher $\exponent{}$. For instance, choosing $\exponent{} = 7$ results in $365$ constraints and $\approx 20k$ gas while $\exponent{} = 31$ corresponds to $417$ constraints ($+15\%$) and $\approx 17k$ ($-10\%$) in gas consumption. Repeating the same process for different exponents, we observe roughly the same order of magnitude gain on the gas consumption and loss on the number of constraints.

The number of constraints of \mimcMPPrime{} for several exponents $\exponent{}$ is given by the formula
\[
    \constraints = \rounds \cdot \mults + 1
\]
where $\rounds = \lceil \frac{\log_2 \rCURVE}{\log_2 \exponent{}} \rceil$, $\mults$ is the number of multiplications required for exponentiation and the additional constraint (corresponding to $+1$ in the above formula) is a result of the final message and key addition. Note that for $\exponent = 2^t - 1$ we have $\mults = 2 \cdot t - 2$, using the \emph{square-and-multiply} algorithm~\cite{menezes1996handbook}, and for $\exponent = 2^t + 1$ we have $mults = t + 1$.
% TODO: Check if this final constraint can be removed.

For several concrete values of $\exponent{}$, the number of $\rounds$ required to attain the desired security level, along with the number of constraints, are shown in~\cref{table:mimc-exp-analysis}.

\begin{table}
  \centering
    \begin{minipage}[t]{0.50\textwidth}
        \centering
        \begin{tabular}{r c c c c}
            \toprule
            \multirow{2}{*}{$\exponent{}$} & \multicolumn{2}{c}{\BNCurve} & \multicolumn{2}{c}{\BLSCurve} \\ [0.5ex]
            & $\rounds$ & $\constraints$ & $\rounds$ & $\constraints$ \\ [0.5ex]
            \midrule
            5 & 110 & 331 & & \\
            7 & 91 & 365 & & \\
            17 & 65 & 316 & 62 & 311 \\
            31 & 52 & 417 & 51 & 409 \\
            127 & 37 & 445 & 37 & 445 \\
            257 & 32 & 289 & 32 289 & \\
            511 & 29 & 465 & & \\
            2047 & 24 & 481 & 23 & 461 \\
            8191 & 20 & 481 & 20 & 481 \\
            32676 & 17 & 477 & & \\
            65537 & 16 & 273 & 16 & 273 \\
            131071 & 15 & 481 & 15 & 481 \\
            524287 & 14 & 505 & 14 & 505 \\
            1048577 & 13 & 274 & 13 & 274 \\
            2097151 & 13 & 521 & & \\
            \bottomrule
        \end{tabular}
    \end{minipage}%
    \caption{Arithmetic constraints required to represent \mimcMP{} as an R1CS program, for different exponents $\exponent{}$ and curves. Missing entries where $\gcd(\exponent{},\rCURVE - 1) \neq 1$}\label{table:mimc-exp-analysis}
\end{table}

For the case of \BNCurve~we set $\exponent{} = 7$ and $\rounds{} = 91$, targetting a $254$-bit security level. For \BLSCurve~we set $\exponent{} = 31$ and $\rounds{} = 51$, targetting a $253$-bit security level. These values are chosen such that they satisfy the requirement that $\gcd(\exponent{},\rCURVE - 1) = 1$ and give a good balance between the number of constraints in the arithmetic circuit and the gas cost of hashing on the contract.

\subsection{Security requirements satisfaction}\label{instantiation:mkhash:security}

After presenting the state of the art of MiMC cryptanalysis, we present the security proof of \mimcMPPrime{} collision resistance.

\subsubsection{Cryptanalysis of $\mimc{}$ block cipher and primitives}\label{instantiation:mkhash:security:cryptanalysis}

\mimc{}'s security is increasingly being analysed since the primitive has gained traction in zero-knowledge and cryptocurrency communities for its succinct algebraic constraint representation. As of today, we do not know of any attacks breaking \mimc{} on prime fields on full rounds.

The first attack on \mimc{} was an interpolation attack~\cite{li2019improved} which targets a reduced-round version for a scenario in which the attacker has only limited memory.
An attack on Feistel-based \mimc{}~\cite{bonnetain2019collisions} was discovered shortly after, by using generic properties of the used Feistel construction (instead of exploiting properties of the primitive itself).
Additionally,~\cite{albrecht2019algebraic} proposes an attack based on Gr\"{o}bner basis. The authors state that by introducing a new intermediate variable in each round, the resulting multivariate system of equations is a Gr\"{o}bner basis. As such, the first step of a Gr\"{o}bner basis attack can be obtained for free. However, the following steps of the attack are so computationally demanding that the attack becomes infeasible in practice.
A recent work~\cite{cryptoeprint:2020:182} targets \mimc{} on binary fields, and achieves a full-round break of the scheme. While, the attack presented does not apply to prime fields, the authors note that it ``can be generalized to include ciphers over $\FFx{p}$'', and that only the lack of efficient distinguishers over prime fields precludes this.
Another attack from Beyne et al~\cite{cryptoeprint:2020:188} uses a low complexity distinguisher against full \mimc{} permutation leading to a practical collision attack on reduced round sponge-based \mimc{} hash defined with security of 128 bits.

\subsubsection{Security proof of \mimcMPPrime{} collision resistance}\label{instantiation:mkhash:security:colres-proof}
We now prove that this compression scheme satisfies all the security requirements listed in~\cref{zeth-protocol:sec-req}. To do so, we first assume that the round constants are pseudo-random, i.e.~that $\keccak{256}$ is a \prf{}.

\begin{lemma}
	\keccak{256} is a $\prf$ with $\lambda=128$.
\end{lemma}

The security of \mimcMPPrime{} derives from a more general result, i.e.~from modelling \mimcPrime{} as an ideal cipher (see~\cref{preliminaries:def:ICM}). More specifically, we show a security result for the \MP{} construction on \FFx{\rCURVE} by proving that, in the Ideal Cipher Model, the collision resistance advantage of any adversary is bounded by $\frac{q(q+1)}{\rCURVE}$, where $q$ is the number of different queries that the attacker makes to the oracle. This means that, assuming a maximum $q$ number of possible encryption/decryption queries, parameter $\rCURVE$ can be chosen to make the advantage small as needed and $\fMP$ considered collision resistant. Similar result applies to the ${2^n}$ case.

The instance of \mimc{} we use is modelled as an ideal cipher defined on field elements, for this reason we consider a variant of the ICM model where the keys, inputs and outputs are field elements in $\FFx{\rCURVE}$ and the block cipher scheme, with key $\key$, correspond to a family of $\rCURVE$ independent random permutations $f_{\key}: \FFx{\rCURVE} \times \FFx{\rCURVE} \to \FFx{\rCURVE}$.

In the proof, without loss of generality, we assume the following conventions for an adversary \adv{}:
\begin{itemize}
    \item the adversary asks distinct queries: i.e.~if \adv{} asks a query $\oracleEnc(\key,\msg)$ and this returns $y$, then \adv{} does not ask a subsequent query of $\oracleEnc(\key,\msg)$ or $\oracleDec(\key,y)$, and inversely;
    \item the adversary necessarily obtained the candidate collision from the oracle. This property follows suite from modelling \mimc{} as an ideal cipher.
\end{itemize}

\begin{lemma}\label{lemma:colrescomp}
    Let \fMP{} be the \MP{} compression function built on an ideal block-cipher \Enc{} on \FFx{\rCURVE}, the probability for an adversary \adv{} to find a collision is not greater than $q(q+1)/\rCURVE$ where $q$ is a (positive) number of distinct oracle queries.
\end{lemma}

The following proof has been adapted from~\cite[Lemma 3.3]{black2002black}\footnote{It states the collision resistance of a set of compression functions $f_1,,\ldots,,f_{12}$, denoted as \emph{group-1 compression functions} and showed in~\cite[Figure 3]{black2002black}. As mentioned above, Miyaguchi-Preneel corresponds to $f_3$ of that group. Since the proof of~\cite[Lemma 3.3]{black2002black} shows collision resistance of $f_1$, we slightly modified it to work for $f_3$.}.

\begin{proof}
    Fix $h_0\in\FFx{\rCURVE}$. Let \adv{} be an adversary attacking the compression function \fMP{}.
    Assume that \adv{} asks the oracles \oracleEnc{} and \oracleDec{} a total of \emph{distinct} $q$ queries. Let us denote the result of the $q$ queries and output of the attacker (candidate collision) as $\left ( (\key_1, \msg_1, y_1), \ldots , (\key_q, \msg_q, y_q), \text{out} \right )$.
    If \adv{} is successful it means that it outputs $(\key, \msg)$, $(\key', \msg')$ such that either $(\key, \msg) \neq (\key', \msg')$ and $\fMP(\key, \msg) = \fMP(\key', \msg')$ or $\fMP(\key, \msg) = h_0$.
    By the definition of \fMP, we have that $\Enc_\key(\msg) + \msg + \key = \Enc_{\key'}(\msg') + \msg' + \key'$ for the first case, or $\Enc_\key(\msg) + \msg + \key = h_0$ for the second.
    So either there are distinct $r, s \in [1,\ldots, q]$ such that $(\key_r, \msg_r, y_r) = (\key, \msg, \Enc_\key(\msg))$ and $(\key_s,\msg_s, y_s) = (\key',\msg', \Enc_{\key'}(\msg'))$ and $\Enc_{\key_r}(\msg_r) + \msg_r + \key_r = \Enc_{\key_s}(\msg_s) + \msg_s + \key_s$ or else there is an $r \in [1,\ldots, q]\ \suchthat\ (\key_r, \msg_r, y_r) = (\key, \msg, h_0)$ and $\Enc_{\key_r}(\msg_r) + \msg_r + \key_r = h_0$. We show that this event is unlikely.

    In fact, for each $i \in [1,\ldots, q]$, let $C_i$ be the event that either $y_i + \msg_i + \key_i = h_0$ or does exist $j \in [1,\ldots, i-1]\ \suchthat\ y_i + \msg_i + \key_i = y_j + \msg_j + \key_j$. When carrying out the simulation $y_i$ or $\msg_i$ was randomly selected from a set of at least $\rCURVE - (i-1)$ elements, so $\prob{C_i}\leq i / (\rCURVE-i)$. This means that for the collision advantage of \adv{}, \advCollMP it holds that $\advCollMP \leq \prob{C_1 \lor \cdots \lor C_q} \leq \sum_{i=1}^{q} \prob{C_i}$. For $q \leq \frac{\rCURVE}{2}$ this probability is bounded by $l \cdot \frac{q(q+1)}{\rCURVE}$. However, we allow only a polynomial number of queries, thus for $q = \poly$ this probability becomes $\frac{\poly}{\rCURVE}$, where $\rCURVE \approx 2^\secpar$.
\end{proof}

\begin{notebox}
   \cref{lemma:colrescomp} is applicable to our case by the strong assumption of \mimcPrime{\rCURVE} being an ideal cipher. In other words, the proof does not take into account any structural weakness or knowledge that an attacker is aware of. Any such additional information could make~\cref{lemma:colrescomp} invalid, and consequently could be used to break the collision resistance.
\end{notebox}

\begin{remark}
    Note that from~\cref{lemma:colrescomp} follows that the collision resistance security of the \zeth{} Merkle tree is $\log_2(\rCURVE/2)$ (around $127$ bits for $\rCURVE = \rBN$ or $\rBLS$).
\end{remark}

\begin{notebox}
    $\mimc{}$ has \emph{not} received as much cryptanalytic scrutiny as other ``older'' and more established hash functions. This is important to note since, for these type of primitives which are not provably secure, the amount of attacks received by a scheme is a great indicator of its security and robustness.
    A natural alternative to $\mimc{}$ here consists in using Pedersen hash which is provably collision resistant under the discrete-logarithm assumption.
\end{notebox}
 % MKHASH
% !TEX root = ../zeth-protocol-specification.tex

\section{\zeth~statement after primitive instantiation}\label{instantiation:statement}

After instantiating the various primitives and providing security proofs to justify that they comply with the security requirements listed in previous sections, $\RELCIRC$, now becomes:

\begin{itemize}
    \item For each $i \in [\jsin]$:
    \begin{enumerate}
        \item $ \auxinputs.\jsins{i}.\znote.\apk = \blake{2s}{\taggedaddr \concat \pad{0}{\blakeCompLen}}$ \\ with $\taggedaddr$ defined in~\cref{instantiation:prf-comm-crh:prf}
        \item $\auxinputs.\jsins{i}.\nf{} = \blake{2s}{\taggednf \concat \auxinputs.\jsins{i}.\znote.\rho}$ \\ with $\taggednf$ defined in~\cref{instantiation:prf-comm-crh:prf}
        \item $\auxinputs.\jsins{i}.\cm{} = \blake{2s}{\auxinputs.\jsins{i}.\znote.\noter{} \concat \msg}$ \\ with $\msg = \auxinputs.\jsins{i}.\znote.\apk \concat \auxinputs.\jsins{i}.\znote.\rrho \concat \auxinputs.\jsins{i}.\znote.\notev$
        \item \secfix{$\auxinputs.\htags{i} = \blake{2s}{ \taggedpk \concat \priminputs.\hsig}$ (malleability fix, see:~\cref{appendix:trnm})} with $\taggedpk$ defined in~\cref{instantiation:prf-comm-crh:prf}
        \item $(\auxinputs.\jsins{i}.\znote.\notev) \cdot (1 - e) = 0$ is satisfied for the boolean value $e$ set such that if $\auxinputs.\jsins{i}.\znote.\notev > 0$ then $e = 1$.
        \item The Merkle root $\mkroot'$ obtained after checking the Merkle authentication path $\auxinputs.\jsins{i}.\mkpath$ of commitment $\auxinputs.\jsins{i}.\cm{}$, with $\mimcSevenMPPrime{}$, equals to $\priminputs.\mkroot$ if $e = 1$.
        \item $\priminputs.\nfs{i}$ \\ $= \indexedset{\pack{\slice{\auxinputs.\jsins{i}.\nf{}}{k \cdot \bnFieldBitCap}{(k+1) \cdot \bnFieldBitCap}}{\FFx{\rBN}}}{k \in [\floor{\prfNfOutLen/\bnFieldBitCap}]}$
        \item $\priminputs.\htags{i}$ \\ $= \indexedset{\pack{\slice{\auxinputs.\htags{i}}{k \cdot \bnFieldBitCap}{(k+1) \cdot \bnFieldBitCap}}{\FFx{\rBN}}}{k \in [\floor{\prfPkOutLen/\bnFieldBitCap}]}$
    \end{enumerate}
    \item For each $j \in [\jsout]$:
    \begin{enumerate}
        \item \secfix{$\auxinputs.\znotes{j}.\rrho = \blake{2s}{ \taggedrho \concat \priminputs.\hsig}$ (malleability fix, see:~\cref{appendix:trnm})} with $\taggedrho$ defined in~\cref{instantiation:prf-comm-crh:prf}
        \item $\priminputs.\cms{j} = \blake{2s}{\auxinputs.\znotes{j}.\noter{} \concat \msg }$ \\ with $\msg = \auxinputs.\znotes{j}.\apk \concat \auxinputs.\znotes{j}.\rrho \concat \auxinputs.\znotes{j}.\notev$
    \end{enumerate}
    \item $\priminputs.\hsig = \indexedset{\pack{\slice{\auxinputs.\hsig}{k \cdot \bnFieldBitCap}{(k+1) \cdot \bnFieldBitCap}}{\FFx{\rBN}}}{k \in [\floor{\crhhsigOutLen/\bnFieldBitCap}]}$
    \item $\priminputs.\resbits = \packResBits{\indexedset{\auxinputs.\jsins{i}.\nf{}}{i \in [\jsin]}, \auxinputs.\vin, \auxinputs.\vout, \auxinputs.\hsig, \indexedset{\auxinputs.\htags{i}}{i \in [\jsin]}}$
    \item Check that the ``\gls{joinsplit} is balanced'', i.e.~check that the \gls{joinsplit-eq} holds:
    \begin{align*}
        &\pack{\auxinputs.\vin}{\FFx{\rBN}} + \sum_{i \in [\jsin]} \pack{\auxinputs.\jsins{i}.\znote.\notev}{\FFx{\rBN}} \\
        & = \sum_{j \in [\jsout]} \pack{\auxinputs.\znotes{j}.\notev}{\FFx{\rBN}} + \pack{\auxinputs.\vout}{\FFx{\rBN}}
    \end{align*}
\end{itemize}

\begin{remark}
    For higher security, we could use \blake{2b}{} with 32-byte output instead of \sha{256}. In fact, since a precompiled contract computing the \blake{2}{}~compression function~\cite{blakecompietf} has been added to the Istanbul release of \ethereum~(EIP 152~\cite{blake-eip}), it could be possible to write a small wrapper on the smart contracts, in order to hash with \blake{2b}{} with any parameter.
\end{remark}

\subsection{Instantiating the Packing functions}\label{instantiation:statement:pack}
As we consider SNARKs based on arithmetic circuits based over a prime finite field, our statement's variables are interpreted as field elements. As such, we can take advantage of the packing strategy to reduce the length of the primary inputs and diminish the cost of the on-chain verification. Indeed, the cost of Groth proof verification is linear in the number of primary inputs, each variable is the input of a costly scalar multiplication of an element in $\gset_1$. Hence, while packing puts more ``pressure'' on the prover --- by adding constraints in the circuit --- it simplifies the verifier's work. 

We detail in this section how we pack the primary inputs to minimize their number and present efficient packing and unpacking functions.

The original primary inputs (see:~\cite[Section 3.4.3]{zethpaper}) were the input nullifiers, the output commitments, the public values \secfix{to which we added the signature hash and the authentication tags for security} (malleability fix, see:~\cref{appendix:trnm}).
\[
    (\indexedset{\priminputs.\nf{i}}{i \in [\jsin]}, \indexedset{\priminputs.\cms{j}}{j \in [\jsout]}, \vin, \vout, \hsig, \indexedset{\priminputs.\htags{i}}{i \in [\jsin]})
\]

Consider the binary variables, that is the nullifiers $\nfs{}$, the public values $\vin$ and $\vout$, the signature hash $\hsig$ and the authentication tags $\htags{}$. 
For each of these variables $x$, let $\alpha_x=\ceil {\len{x} / \bnFieldBitCap}$, represent the number of field elements to encode all of the variable's bits. Let $\beta_x= \floor{\len{x} / \bnFieldBitCap}$ represent the number of field elements which are fully ``filled'' and $\gamma_x = \len{x} \pmod{\bnFieldBitCap}$ the remaining ones. For simplicity, as we assume $\zvalueLen<\bnFieldBitCap$, we only define the notation $\gamma_\notev = \zvalueLen$ for the public values $\vin$ and $\vout$.

We denote by $\resBitsBLen$ the total number of bits which could not fully fill a field element, i.e. the weighted sum of the $\gamma_x$,
\[
    \resBitsBLen = \gamma_\hsig + 2 \cdot \gamma_\notev + \jsin \cdot (\gamma_{\nf{}} + \gamma_{\htag{}} )
\]

A simple packing strategy would have been to encode each primary input's field $x$ as $\alpha_x$ field elements. However, this leads to a potentially large waste of space when the PRFs and hash digests are slightly longer than $\bnFieldBitCap$ bits. Another strategy would have been to encode the binary string which results from the concatenation of all binary variables. This would have made the necessary unpacking (see:~\cref{zeth-protocol:process-tx}) costly. We thus decided to keep the fully filled $\sum_x \beta_x$ field elements which are ``fully'' filled and aggregate the remaining $\resBitsBLen$ ``residual'' bits in a new variable $\resbits$:
\begin{align*}
    \nfFLen &= \floor{\prfNfOutLen / \bnFieldBitCap} \\
    \hsigFLen &= \floor{\crhhsigOutLen / \bnFieldBitCap} \\
    \htagFLen &= \floor{\prfPkOutLen / \bnFieldBitCap} \\
    \resBitsFLen &= \ceil{\resBitsBLen / \bnFieldBitCap}
\end{align*}

To facilitate the unpacking of the primary inputs, we chose to first aggregate the primary inputs' singular elements. More precisely, regardless of the values $\jsin$ and $\jsout$, the public values $\vin$ and $\vout$, and the hash signature $\hsig$ will always be at the same location in the $\resbits$ string. The residual bits $\resbits$ are thus formatted as follows,
\[
    \vin \concat \vout \concat \hsig \concat \nfs{} \concat \htags{}.
\]
    
To format the unpacked primary inputs into field elements, we define the following functions. 
The algorithm \pack{}{} (see:~\cref{zeth-protocol:fig:packing-alg}), given a bit string of length less than $\bnFieldBitCap$, returns a field element. The algorithm \packResBits{}{} (see:~\cref{zeth-protocol:fig:packing-resbits-alg}) given the nullifiers, public values and authentication tags outputs the residual bits. For a given field, the algorithm \unpack{}{}, given the associated packed field elements and the residual bits returns the variables reassembled in binary strings. For instance, we have that $\unpack{\priminputs.\nfs{}, \resbits}{\nf{}} = \indexedset{\auxinputs.\jsins{i}.\nf{}}{i \in [\jsin]}$.
\begin{align*}
    &\pack{}{\FFx{\rBN}} : \BB^{\leq \bnFieldBitCap} \to \FFx{\rBN} \\
    &\packResBits{} : {(\BB^\prfNfOutLen)}^{\jsin} \times {\BB^\zvalueLen}^2 \times \BB^\crhhsigOutLen \times {(\BB^\prfPkOutLen)}^{\jsin} \to \FFx{\rBN}^\resBitsFLen \\
    &\unpack{}{} : \FFx{\rBN}^* \times \FFx{\rBN}^{\resBitsFLen} \to \BB^* \\
\end{align*}

More particularly, we use the function $\unpack{}{}$ for the nullifiers, public values and signature hash. As such, we have,
\begin{align*}
    &\unpack{}{}_{\hsig} : \FFx{\rBN}^\hsigFLen \times \FFx{\rBN}^{\resBitsFLen} \to \BB^\crhhsigOutLen \\
    &\unpack{}{}_{\nf{}} : \FFx{\rBN}^\nfFLen \times \FFx{\rBN}^{\resBitsFLen} \to \BB^\prfNfOutLen \\
    &\unpack{}{}_{\vin} : \FFx{\rBN}^0 \times \FFx{\rBN}^{\resBitsFLen} \to \BB^\zvalueLen \\
    &\unpack{}{}_{\vout} : \FFx{\rBN}^0 \times \FFx{\rBN}^{\resBitsFLen} \to \BB^\zvalueLen
\end{align*}

\begin{figure*}
    \begin{minipage}[t]{.4\textwidth}
        \centering
        \procedure{\pack{x}{\FFx{\rBN}}}{%
            out \gets 0_{\FFx{\rBN}}; \\
            \pcforeach i \in [\len{x}] \pcdo: \\
            \t \pcif x[i] = 1 \pcdo: \\
            \t \t out \gets out +_{\FFx{\rBN}} 2^{\len{x}-1-i} \\
            \pcreturn out;
        }
        \caption{Packing algorithm in Big Endian.}\label{zeth-protocol:fig:packing-alg}
    \end{minipage}%
    \begin{minipage}[t]{.6\textwidth}
        \centering
        \procedure{\packResBits{\nfs{}, \vin, \vout, \hsig, \htags{}}}{%
            out \gets []; r \gets \epsilon; \\
            r \gets \slice{\vin}{\floor{\zvalueLen / \bnFieldBitCap} \cdot \bnFieldBitCap}{}; \\
            r \gets r \concat \slice{\vout}{\floor{\zvalueLen / \bnFieldBitCap} \cdot \bnFieldBitCap}{}; \\
            r \gets r \concat \slice{\hsig}{\floor{\crhhsigOutLen / \bnFieldBitCap} \cdot \bnFieldBitCap}{}; \\
            %
            \pcfor i \in [\jsin] \pcdo:\\
            \t r \gets r \concat \slice{\nfs{i}}{\floor{\prfNfOutLen / \bnFieldBitCap} \cdot \bnFieldBitCap}{}; \\
            %
            \pcfor i \in [\jsin] \pcdo:\\
            \t r \gets r \concat \slice{\htags{i}}{\floor{\prfPkOutLen / \bnFieldBitCap} \cdot \bnFieldBitCap}{}; \\
            %
            \pcfor i \in [ \ceil{\len{r}/ \bnFieldBitCap} ] \pcdo: \\
            \t out \gets \pack{\slice{r}{i \cdot \bnFieldBitCap}{ (i+1) \cdot \bnFieldBitCap}}{\FFx{\rBN}}; \\
            \pcreturn out;
        }
        \caption{Packing residual bits algorithm.}\label{zeth-protocol:fig:packing-resbits-alg}
    \end{minipage}
\end{figure*}

\subsubsection{Packing Policy Security}
\begin{proposition}[Packing security]
    The encoding (resp. decoding) of a variable via $\pack{}{}$ and $\packResBits{}$ (resp. $\unpack{}{}$) is bijective.
\end{proposition}

\subsubsection{Packing Policy Example}
    
In the case where $\jsin=\jsout=2$, $\bnFieldBitCap$ and all PRFs, and $\crhhsig{}$ output 256 bits, the unpacked primary inputs are 2167-bit long. The packing parameters thus are, 
\begin{align*}
    & \resBitsBLen = 143 \\
    & \nfFLen = \hsigFLen = \htagFLen = \resBitsFLen = 1 
\end{align*}
The packed primary inputs are 2277 bits long which corresponds to a small space overhead ($\approx 5\%$ unused bits). Moreover, as the residual bits are 143 bit long, they can be written over a single field element. As such, the primary inputs are 9 field element long.
Finally, the residual bits are formatted as follows,
\[
    \underbrace{padding}_{\text{113 bits}} \concat \underbrace{\vin}_{64\ bits} \concat \underbrace{\vout}_{64\ bits} \concat \underbrace{\hsig}_{3\ bits} \concat \underbrace{\nf{0}}_{3\ bits} \concat \underbrace{\nf{1}}_{3\ bits} \concat \underbrace{\htag{0}}_{3\ bits} \concat \underbrace{\htag{1}}_{3\ bits}
\]
 % Zeth statement after instantiation
% !TEX root = ../zeth-protocol-specification.tex

\section{Instantiate $\sigscheme_{\otsig}$}\label{instantiation:otsig}

In \zeth, we chose to use the one-time Schnorr-based signature scheme introduced by Bellare and Shoup~\cite{bellare2007two}, over $\BNCurve$, for its long proven security, simplicity, speed and size. Its security relies on the one-more discrete log problem (see:~\cref{preliminaries:def:om-dlog}) and the collision resistance of the underlying hash function $\crh$ (see:~\cref{preliminaries:def:collision-resistance}) that we instantiate with \sha{256}.

This one-time signature scheme (see:~\cref{preliminaries:def:ot-sig}) is defined by the two-tier signature scheme over a cyclic group $(p, \gset, \langle \ggen \rangle, \otimes)$.
In the two-tier signature scheme, the hash function $\crh$ only needs to be collision resistant (the random oracle model is not used). Similarly, the variable $\hk$ represents the key of the hash function (a particular instance).

To turn this two-tier signature scheme into a one-time signature scheme, one simply has to define the one-time signature key generation \kgen~as the combination of both primary and secondary key generations of the two-tier (see:~\cite[Section 6]{bellare2007two}). The one-time signing key (respectively verification key) of the one time signature scheme is defined as both the primary and secondary signing key (respectively verification key) of the two-tier scheme,~\cref{instantiation:fig:ots-from-two-tier-sig}

\begin{figure*}[ht]
    \begin{minipage}[t]{0.33\textwidth}
        \begin{align*}
            & \underline{\kgen():} \\
            & \hk \sample \BB^{\kl} \\
            & \ggen \sample \gset^* \\
            & x \sample \FFx{p} \\
            & \pk{}1 = (\hk, \ggen, \groupenc{x}) \\
            & \sk{}1 = (\hk, \ggen, x) \\
            & y \sample \FFx{p} \\
            & \pk{}2 =  \groupenc{y} \\
            & \sk{}2 = (y, \groupenc{y}) \\
            & \pk = (\pk{}1, \pk{}2) \\
            & \sk = (\sk{}1, \sk{}2) \\
        \end{align*}
    \end{minipage}%
    \begin{minipage}[t]{0.33\textwidth}
        \begin{align*}
            & \underline{\sig(\sk, \msg):}\\
            & \hk, \ggen, x = \sk.\sk{}1 \\
            & y, \groupenc{y} = \sk.\sk{}2 \\
            & c = \crh(\hk, \groupenc{y} \concat \msg) \\
            & \sigma = y \bmod p \\
            & \sigma\ \text{+=}\ c \cdot x \bmod p \\
            & \textbf{return}\ \sigma
        \end{align*}
    \end{minipage}%
    \begin{minipage}[t]{0.33\textwidth}
        \begin{align*}
        & \underline{\verify(\pk, \msg, \sigma):}\\
        &  \hk, \ggen, \groupenc{x} = \pk.\pk{}1 \\
        &  \groupenc{y} = \pk.\pk{}2 \\
        &  c = \crh(\hk, \groupenc{y} \concat \msg)\\
        &  \pcif \sigma \iseq \groupenc{y} \otimes c \cdot \groupenc{x} \pcthen: \\
        &  \pcind \pcreturn 1\\
        &  \pcendif \\
        &  \pcreturn 0
        \end{align*}
    \end{minipage}
\caption{One-time signature scheme from two tier Schnorr based signature scheme by Bellare and Shoup~\cite{bellare2007two}}\label{instantiation:fig:ots-from-two-tier-sig}
\end{figure*}

\subsection{Security requirements satisfaction}

We now prove that this signature scheme satisfies all the security requirements listed in~\cref{zeth-protocol:sec-req}.

\begin{theorem}
    The One-Time Schnorr signature is strongly unforgeable under chosen-message attacks (\sufcma) assuming that the \omdlog~problem is hard in \gset~and that the hash function is collision resistant.

\begin{proof}
    See:~\cite[Theorems 5.1, 5.2 and 6.1]{bellare2007two}.
\end{proof}

\end{theorem}

\subsection{Data types}\label{instantiation:otsig:data-types}

We now describe the data types associated with this signature scheme defined over $\BNCurve$.

\begin{description}
    \item[\vkOtsDType] Denotes the verification key associated with the one-time signature scheme.
        \begin{table}[H]
        \centering
        \begin{tabular}{cp{20em}c}
            Field & Description & Value\\ \toprule
            $pk1$ & Encoding of the scalar $x$ in the group & ${(\FFx{\rBN})}^{2}$ \\ \midrule
            $pk2$ & Encoding of the scalar $y$ in the group & ${(\FFx{\rBN})}^{2}$ \\ \bottomrule
        \end{tabular}
        \caption{\vkOtsDType~data type}\label{instantiation:tab:vk-ots-dtype}
        \end{table}
    \item[\skOtsDType] Denotes the signing key associated with the one-time signature scheme.
        \begin{table}[H]
        \centering
        \begin{tabular}{cp{20em}c}
            Field & Description & Value\\ \toprule
            $sk1$ & Scalar element $x$ & $\FFx{\rBN}$ \\ \midrule
            $sk21$ & Scalar element $y$ & $\FFx{\rBN}$ \\ \midrule
            $sk22$ & Encoding of the scalar $y$ in the group & ${(\FFx{\rBN})}^{2}$ \\ \bottomrule
        \end{tabular}
        \caption{\skOtsDType~data type}\label{instantiation:tab:sk-ots-dtype}
        \end{table}
    \item[\sigOtsDType] Denotes the signature data type associated with the one-time signature scheme. $\sigOtsDType$ is an alias for ${(\FFx{\rBN})}^{2}$.
\end{description}
 % OT-SIG
% !TEX root = ../zeth-protocol-specification.tex

\section{Instantiate $\encscheme$}\label{instantiation:enc}

In this section we describe the instantiation of $\encscheme$ primitive introduced in \cref{zeth-protocol:mix-inp}. First, we present a general asymmetric encryption scheme called \dhaes{} (Diffie-Hellman Asymmetric Encryption Scheme~\cite{abdalla1999dhaes}), which satisfies all the required security properties for the in-band encryption scheme $\encscheme$ (see~\cref{preliminaries:sec-assumptions}). Then, we give details of the concrete algorithms used for the implementation.

\subsection{\dhaes{} encryption scheme}\label{instantiation:enc:dhaes}

Given a symmetric encryption scheme $\sym{}$, a group defined by $\groupSetup{}$, a family of hash function $\hashSet{}$\footnote{Here, we only consider fixed-length hash functions with $\hashInpLen(\secpar) = 2\groupLen$ and $\hashLen(\secpar) = \symKeyLen(\secpar) + \macKeyLen(\secpar)$ (see~\cref{preliminaries:definitions}).} and a message authentication scheme $\mac{}$ as defined in~\cref{preliminaries:definitions}, we define a \dhaes{} scheme as the following public-key encryption scheme:

\begin{itemize}
    \item \setup, setup algorithm, takes as input a security parameter \secparam{}. It runs $\hashSet.\hashSetup{}$, \groupSetup{} and returns public parameters \pparams{} = $(\hk,(q, \gset, \ggen, +)$).
    \item \kgen, key generation algorithm, takes as input public parameters $\pparams$. It samples at random $v \sample{} [q]$ and returns a keypair $(\sk, \pk) = (v, \groupenc{v})$.
    \item \enc, encryption algorithm, takes as input public parameters $\pparams$, a message \msg{} and a public key $\pk$. It runs \kgen{} that returns an ephemeral keypair $(\esk, \epk) = (u, \groupenc{u})$. Then, it computes a shared secret $\sharedSecret = \hash_{\hk} (\epk \concat{} \esk{} \cdot \pk) = \hash_{\hk} (\epk \concat{} \sk{} \cdot \epk)$, parsed as $\ek{} \concat{} \mk{}$\footnote{Note that \ek{} and \mk{} must have the same length.}. It computes $\ct_{\sym} = \sym.\enc(\ek, \msg)$ and $\tau = \mac.\tagg(\mk, \ct_{\sym})$ and finally outputs the ciphertext $\epk \concat \ct_{\sym} \concat \tau$.
    \item \dec, decryption algorithm, takes as input public parameters $\pparams$, a private key $\sk$ a ciphertext $\epk \concat{} \ct_{\sym} \concat \tau$. It computes $\sharedSecret = \hash_{\hk} (\epk \concat \sk \cdot \epk)$ and parses it, as above, as $\ek \concat{} \mk$. If \mac{} verification passes, i.e.~$\mac.\verify(\mk, \tau) = 1$, the algorithm returns $\sym.\dec(\ek, \ct_{\sym})$ and $\bot{}$ otherwise.
\end{itemize}

The \dhaes{} definition given above is an asymptotic adaptation of~\cite[Section 1.3]{abdalla1999dhaes}.

\subsubsection{Inclusion of ephemeral key in hash input}\label{instantiation:enc:dhaes:eph-key}

Given an ephemeral keypair $(u_0, \groupenc{u_0})$, If the group $\langle \ggen \rangle$, generated by \groupSetup{}, has composite order, then $\groupenc{u_0}$ is required to be part of the hash input because $\groupenc{u_0 v}$ and $\groupenc{v}$ together may not uniquely determine $\groupenc{u_0}$. Equivalently, there may exist two values $u_0$ and $u_1$ such that $u_0 \neq u_1$ and $\groupenc{u_0 v} = \groupenc{u_1 v}$. As a result, both $u_0$ and $u_1$ can be used to produce two different \emph{valid} ciphertexts of the same plaintext \msg, under different ephemeral keys ($\groupenc{u_0}, \groupenc{u_1}$). It is easy to show this, for example, in the multiplicative group $\mathbb{Z}_p \setminus \{0\}$, where $p$ is a prime (see~\cite[Section 3.1]{abdalla1999dhaes}). A scheme having such malleability property clearly cannot be proven \indccaii{} secure: an attacker could easily win the related security game by altering the challenged ciphertext and query the decryption oracle that would not recognize that as a not allowed query. If the group has prime order this problem does not arise so only $\groupenc{u_0v}$ is required as input of the $\hash$ function~\cite[Section 3]{abdalla2001dhies}.

\subsection{A \dhaes{} instance}\label{instantiation:enc:algos}

\subsubsection{Curve25519}\label{instantiation:enc:algos:curve25519}

For a cyclic group we propose the use of a subgroup of \curve{25519} described in~\cite{bernstein2006curve25519} and in~\cite{rfc7748}. \curve{25519} is a Montgomery elliptic curve~\cite{montgomery1987speeding} defined by the equation $y^2 = x^3 + 486662x^2 + x$ and coordinates on $\mathbb{F}_p$, where $p$ is the prime number $2^{255}-19$. It has a prime order subgroup of order $2^{252} + \seqsplit{27742317777372353535851937790883648493}$ and cofactor $8$.
\curve{25519} comes with an efficient scalar multiplication denoted as \xscalarmult{25519}\footnote{\xscalarmult{25519} is actually introduced in~\cite{rfc7748} in order to avoid notation issues due to the use \curve{25519} to indicate both curve and scalar multiplication as done in~\cite{bernstein2006curve25519}}. In a Diffie-Hellman-based scheme it allows to have 32-byte long public and private keys (given a point $P = (x,y)$ only the $x$ coordinate is actually used) and the $32$-byte sequence representing $9$ is specified as base point.

\subsubsection{Efficiency and security of Curve25519}\label{instantiation:enc:algos:curve25519sec}
% Security of the X25519
High-speed and timing-attack resistant implementations of \xscalarmult{25519} are available and its security level is conjectured to be $128$ bits~\cite[Section 1]{bernstein2006curve25519}. However, combined attacks can lead to $124$ bits of security (see~\cite[Section ``Twist Security'']{safecurves2017}). By design, \curve{25519} is resistant to state-of-the-art attacks and satisfies all security criteria and principles listed in \emph{Safecurves}~\cite{safecurves2017}\footnote{In this work, the authors take into account both Elliptic Curve Discrete Logarithm Problem (ECDLP) and Elliptic Curve Cryptosystems (ECC) security, that allows to have an overall evaluation of the security guarantees.}.

% Security: public key validation
Interestingly, \curve{25519} does not require \emph{public key validation}\footnote{Informally, it is a set of security checks that a user performs before using a not trusted public key (e.g.~see~\cite{barker2018recommendation})}, while we know that, on other curves, active attacks -- consisting of sending malformed public keys -- could be carried out by adversaries, to violate the confidentiality of private keys, e.g.~\cite{antipa2003validation}. However, \curve{25519} specification mandates the \emph{clamping} of private keys: that is, after the random sampling of $32$ bytes, the user clears bits $0$, $1$ and $2$ of the first byte, clears bit $7$ and sets bit $6$ of the last byte. The resulting $32$ bytes are then used as private key. This particular structure for private keys prevents various types of attacks (see~\cite[Section 3]{bernstein2006curve25519} for more details).

\begin{notebox}
    Note that the \emph{clamping} procedure is vital to ensure the security guarantees of the \curve{25519} specification, and implementations \MUST{} perform this exactly as described.
\end{notebox}

\subsubsection{Chacha20}\label{instantiation:enc:algos:chacha20}

\chacha{20} is an ARX-based\footnote{Addition-Rotation-XOR} stream cipher introduced in~\cite{bernstein2008chacha}. It is an improved version of \salsa{20}~\cite{Bernstein:2008:SFS:1423346.1423354} that won the \emph{eSTREAM} challenge~\cite{estreamchallenge}. Compared with \salsa{20}, it has been designed to improve diffusion per round, conjecturally increasing resistance to cryptanalysis, while preserving time efficiency per round. It is considerably faster than \aes{} in software-only implementations and can be easily implemented to be timing-attacks resistant. Several versions of the cipher can be used. The original paper presents \chacha{20} with a $128$-bit key and $64$-bit nonce/block count. However, the length of the key, nonce and block count -- which indicates how many chunks can be processed by using the same key and nonce -- can be modified depending on the application. In~\cite{langley2018chacha20}[Section 2.3], for instance, the key is a $256$-bit string, the nonce is a string of $96$ bits and the block count is encoded on a $32$-bit word. This configuration allows to process around $2^{32}$ blocks, corresponding to roughly $256$\,GB of data. We propose to use the same parameters in \zeth{}.

\begin{align*}
    \chacha{20}&: \BB^{256} \times \BB^{32} \times \BB^{96} \times \BB^{*} \to \BB^{*}
\end{align*}

\subsubsection{Security of Chacha}

Recent cryptanalysis results for \chacha{} are available in~\cite{aumasson2008new,ishiguro2012modified, shi2012improved, maitra2016chosen, choudhuri2016differential, choudhuri2017maitra}: all of them make use of advanced cryptanalysis techniques able to perform key-recovery attacks only on reduced versions (6 and 7 rounds) of \chacha{}.

\begin{notebox}
    Importantly, the security properties of \chacha{} rely on the fact that, for a given key, all blocks are processed with distinct values in the state words 12 to 15 (storing the counter and the nonce)~\cite[Section 2.3]{langley2018chacha20}.
\end{notebox}

\subsubsection{Poly1305}\label{instantiation:enc:algos:poly1305}

\polymac{1305}~\cite{bernstein2005state} is a high-speed message authentication code, easy to implement and make side-channel attack resistant. It takes a $32$-byte one-time key $\mk$ and a message $\msg$ and produces a $16$-byte tag $\tau$ that authenticates the message. $\mk$ must be unpredictable and it is represented as a couple $(r,s)$, where both components are given as a sequence of $16$ bytes each. It can be generated by using pseudorandom algorithms: in~\cite[Section 2]{bernstein2005state}, for example, AES and a nonce are used to generate $s$. The second part of the key, $r$, is expected to have a given form~\cite[Section 2]{bernstein2005state}, and must be ``clamped'' as follows: top four bits of $r[3]$, $r[7]$, $r[11]$, $r[15]$ and bottom two bits of $r[4]$, $r[8]$, $r[12]$ are cleared (see also~\cref{instantiation:enc:enc-sch}).
\begin{notebox}
    Similarly to \curve{25519}, the \emph{clamping} procedure here is essential to the security of the \polymac{1305} scheme. Implementations \MUST{} ensure that this is performed correctly in order for all security guarantees to hold.
\end{notebox}
We refer to~\cite[Section 2.5, Section 3]{langley2018chacha20} for \tagg{} and \verify{} implementations of \polymac{1305}.

\begin{align*}
    \polymac{1305}.\tagg &: \BBy{32} \times \BBy{*} \to \BBy{16}\\
    \polymac{1305}.\verify &: \BBy{32} \times \BBy{16} \times \BBy{*} \to \BB
\end{align*}

% security
\subsubsection{Security of Poly1305}

Citing~\polymac{1305}~\cite[Section 4]{langley2018chacha20},``the \polymac{1305} authenticator is designed to ensure that forged messages are rejected with a probability of $1-(n/(2^{102}))$ for a $16n$-byte message, even after sending $2^{64}$ legitimate messages, so it is SUF-CMA (strong unforgeability against chosen-message attacks)''.

\subsubsection{Blake2b-512}\label{instantiation:enc:algos:blake2b256512}

Since we need a total of $64$ bytes for the key material ($32$ for \chacha{20} and $32$ for \polymac{1305}) \blake{2b512}{} can be used. ZCash protocol~\cite[Section 5.4.3]{zcashprotocol}, instead, makes use of \blake{2b256}{} since a \dhaes{} variant, denoted as \chacha{20}-\polymac{1305}, is adopted (see~\cite[Section 2.8]{langley2018chacha20}).

\begin{align*}
    \blake{2b512}{}&: \BB^{*} \to \BBy{32}
\end{align*}

\subsection{\encscheme{} instantiation}\label{instantiation:enc:enc-sch}

In the following we instantiate \encscheme{} as a \dhaes{} scheme, detailing the \kgen{}, \enc{} and \dec{} components. First, we introduce some required constant values:
\begin{align*}
    \eskByteLen &= 32\\
    \epkByteLen &= 32\\
    \noteByteLen &= (\prfAddrOutLen + \noterLen + \zvalueLen + \prfRhoOutLen) / \byteLen\\
    \symKeyByteLen &= 32\\
    \macKeyByteLen &= 32\\
    \kdfDigestByteLen &= \symKeyByteLen + \macKeyByteLen\\
    \ctByteLen &= \epkByteLen + \noteByteLen + \tagByteLen\\
    \tagByteLen &= 16\\
    \chachaNonceValue &= 0^{32}\\
    \chachaBlockCounterValue &= 0^{96}\\
\end{align*}

\subsubsection{\encscheme.\kgen{}}

The keypair $(\sk, \pk)$ generation is defined as:

\begin{itemize}
    \item Randomly sample a sequence of $\eskByteLen$ bytes and assign to $\sk$.
    \item Clamp \sk{} as follows:
    \begin{align*}
        \sk[0] &\gets \sk[0]\ \&\ \bytestyle{F8}\\
        \sk[31] &\gets \sk[31]\ \&\ \bytestyle{7F}\\
        \sk[31] &\gets \sk[31]\ |\ \bytestyle{40}
    \end{align*}
    where $|$ and $\&$ denotes, respectively, OR and AND binary operators between bit strings of same the length.\footnote{E.g Given two bytes \bytestyle{15} and \bytestyle{03} then $\bytestyle{15} | \bytestyle{03} = \bytestyle{17}$ and $\bytestyle{15} \& \bytestyle{03} = \bytestyle{01}$.}
    \item Compute $\pk{} = \xscalarmult{25519}(\sk, \bytestyle{09})$.
    \item Return $(\sk, \pk) \in \BBy{\eskByteLen} \times \BBy{\epkByteLen}$
\end{itemize}

% Enc
\subsubsection{\encscheme.\enc}

The encryption, on inputs $(\pk, \msg) \in \BBy{\epkByteLen} \times \BBy{\noteByteLen}$, is defined as follows:

\begin{enumerate}
    \item Generate an ephemeral \curve{25519} keypair $(\esk, \epk) \in \BBy{\eskByteLen} \times \BBy{\epkByteLen}$ (as above).
    \item Compute the shared secret\footnote{We assume here that \esk{} has been clamped as discussed in~\cref{instantiation:enc:algos:curve25519}} $\sharedSecret \in \BBy{\epkByteLen}$:
    \[
        \sharedSecret = \xscalarmult{25519}(\esk, \pk) \in \BBy{\epkByteLen}
    \]
    \item Generate a session key:
    \[
        \blake{2b512}{\encTag \concat \epk \concat \sharedSecret} \in \BBy{\kdfDigestByteLen}
    \]
    where $\encTag = \bytestyle{5A} \concat \bytestyle{65} \concat \bytestyle{74} \concat \bytestyle{68} \concat \bytestyle{45} \concat \bytestyle{6E} \concat \bytestyle{63}$, that is the UTF-8 encoding of ``$ZethEnc$'' string (used for domain separation purposes). The result, then, is parsed as follows:
    \begin{align*}
        \ek &= \blake{2b512}{\encTag \concat\epk \concat\sharedSecret}[: \symKeyByteLen-1]\\
        \mk &= \blake{2b512}{\encTag \concat\epk \concat\sharedSecret}[\symKeyByteLen: \symKeyByteLen + \macKeyByteLen-1].
    \end{align*}
    \item Encrypt the confidential data:
    \[
        \ct_{\sym} = \chacha{20}(\ek, \chachaBlockCounterValue, \chachaNonceValue, \msg) \in \BB^{\noteByteLen * \byteLen}
    \]

    \begin{remark}
        Formally speaking we should have written $\ct_{\sym} \in \BB^{n}$, where $n$ is the length of binary representation of the encrypted message $m$. In \zeth{} however, the only data encrypted are the notes. As such, the size of the plaintexts is $\noteByteLen * \byteLen$ bits.
    \end{remark}
    \begin{remark}
        In the following, we omit the explicit conversion from $\BB^{n}$ to $\BBy{\ceil{n/\byteLen}}$ when passing the output of \chacha{20} to the \polymac{1305} algorithms.
    \end{remark}
    \item Randomly generate $(r,s) \in \BBy{\macKeyByteLen/2} \times \BBy{\macKeyByteLen/2}$ and clamp it:
    \begin{align*}
        r[3] &\gets r[3]\ \&\ \bytestyle{0F}\\
        r[7] &\gets r[7]\ \&\ \bytestyle{0F}\\
        r[11] &\gets r[11]\ \&\ \bytestyle{0F}\\
        r[15] &\gets r[15]\ \&\ \bytestyle{0F}\\
        r[4] &\gets r[4]\ \&\ \bytestyle{FC}\\
        r[8] &\gets r[8]\ \&\ \bytestyle{FC}\\
        r[12] &\gets r[12]\ \&\ \bytestyle{FC}
    \end{align*}
    \item Generate the related tag:
    \[
        \tau = \polymac{1305}.\tagg(\mk, \ct_{\sym}) \in \BBy{\tagByteLen}.
    \]
    \item Create the asymmetric ciphertext as:
    \[
        \ct = \epk \concat \ct_{\sym} \concat \tau \in \BBy{\ctByteLen}.
    \]
    \item Return \ct.
    As consequence $\encZethNoteLen{} = \ctByteLen*\byteLen $ bits.
\end{enumerate}

\subsubsection{\encscheme.\dec}

The decryption, on inputs $(\sk, \ct) \in \BBy{\eskByteLen} \times \BBy{\ctByteLen}$, is defined as follows:
\begin{enumerate}
    \item Parse the ciphertext \ct{} as:
    \begin{align*}
        \epk &\gets \ct[: \epkByteLen-1]\\
        \ct_{\sym} &\gets \ct[\epkByteLen: \epkByteLen + \noteByteLen-1]\\
        \tau &\gets \ct[\epkByteLen + \noteByteLen: \epkByteLen + \noteByteLen + \tagByteLen -1]
    \end{align*}
    \item Recover the shared secret
    \[
        \sharedSecret = \xscalarmult{25519}(\sk, \epk).
    \]
    \item Compute the $\ek{}\concat\mk$
    \begin{align*}
        \ek &= \blake{2b512}{\encTag \concat \epk \concat\sharedSecret}[: \symKeyByteLen - 1]\\
        \mk &= \blake{2b512}{\encTag \concat \epk \concat\sharedSecret}[\symKeyByteLen: \symKeyByteLen + \macKeyByteLen - 1].
    \end{align*}
    \item Verify that the ciphertext has not been forged:
    \[
        \polymac{1305}.\verify(\mk, \tau, \ct_{\sym})
    \]
    \item (If the \mac{} verifies) decrypt:
    \[
        \msg = \chacha{20}.\dec(\ek, \chachaBlockCounterValue, \chachaNonceValue, \ct_{\sym})
    \]
    \item Return \msg.
\end{enumerate}

\subsection{Security requirements satisfaction}\label{instantiation:enc:security}

\dhaes{} has already been proved to be \indccaii{} secure (see~\cite[Section 3.5, Theorem 6]{abdalla1999dhaes})\footnote{Specifically, if \sym{} is \indcpa{} secure, it holds that \hash{} is \hdhi{} secure and \mac{} is \sufcma{} secure.} and to the best of our knowledge there is no paper showing \ikcca{} security. The only proof we have found is related to \dhies{} scheme~\cite{abdalla2010robust}, that is a prime order group version of \dhaes{}. In the following, we provide a proof for \ikcca{} security of \dhaes{} by adapting that proof to our case.

\begin{theorem}[\ikcca{} of \dhaes]\label{th:ik-cca}
    Let \dhaes{} be the asymmetric encryption scheme as defined above. Let \adv{} be an adversary for the \ikcca{} game, then there exists a \hdhi{} adversary \bdv{} of $(\hashSet,\groupSetup)$ and a \sufcma{} adversary \cdv{} of \mac{} such that
    \begin{align*}
        \advantage{\ikcca}{\dhaes, \adv} \leq 2\cdot\advantage{\hdhi}{\hashSet, \groupSetup, \bdv} + \advantage{\sufcma}{\mac, \cdv}.
    \end{align*}
    The adversaries \bdv{} and \cdv{} have the same running time as \adv{}\footnote{In order to give an asymptotic version of the theorem, the number of queries $q$ has been substituted by the fact of considering \ppt{} adversaries.}.
\end{theorem}

\begin{proof}[Informal proof.]
    As already mentioned, \dhaes{} is similar to \dhies{} scheme, except for the underlying group and the way the symmetric keys are constructed. As consequence, \ikcca{} property for \dhaes{} can be shown similarly to the approach in~\cite[Theorem 6.2]{abdalla2010robust}. More precisely, they show that one can construct from an attacker \adv{} for the \ikcca{} game two attackers \bdv{} and \cdv{} for the \odh{} and \sufcma{} games. Actually, they make use of a \bdvii{} attacker for the \odhii{} game~\cite[Figure 20]{abdalla2010robust} and then apply~\cite[Lemma 6.1]{abdalla2010robust} to obtain an attacker \bdv{}\footnote{Note that in~\cite{abdalla2010robust} the \ikcca{} game is a particular case of the \aicca{} game that requires two input messages in the LR query. In order to reason only about the key-privacy, the two messages $\msg_0$ and $\msg_1$ are constrained to be equal.} in the \odh{} game. We adopt a similar strategy, working with \hdhi{}, \hdhii{} and~\cref{preliminaries:lemma:hdhi_hdhii}.

    Let \adv{} be an attacker for the \ikcca{} game, and let \bdvii{} be an attacker for the \hdhii{} game described in~\cref{fig:hdhii_adv}. We show that,
    \[
        \advantage{\hdhii}{\hashSet, \groupSetup, \bdvii{}} = \abs{\prob{\ikcca^{\adv}(\secpar)=1} + \prob{\gamestyle{G_0}^{\adv{}} (\secpar)= 1}-1}
    \]
    where $\gamestyle{G_0}$ is the security game described in~\cref{fig:g0_game}.

    Given an \hdhii{} challenge $(\groupenc{u}, \groupenc{v_{0}}, \groupenc{v_{1}}, w_{b_2, 0}, w_{b_2, 1})$, an adversary \bdvii{} samples $b \sample{} \set{0,1}$ and runs \adv{} on $\groupenc{v_{0}}$, $\groupenc{v_{1}}$ (note that $b_2$ is the random bit chosen by the \bdvii{} challenger in the \hdhii{} game). $\bdvii$ constructs oracles $\oracle{\dec_{\sk_i}}$ where the queries $(\gel{r} \concat \ct_{\sym} \concat \tau)$ are processed as follows: if $\gel{r} \neq \groupenc{u}$, then \bdvii{} queries related \hdhii{} oracle to obtain $\ek{} \concat{} \mk\gets{} \oracleHdhi{v_i}(\gel{r})$ (see~\cref{fig:hdhii_adv}). If $\gel{r}=\groupenc{u}$, $w_{b_2, i}$ is parsed as $\ek \concat \mk$. In both cases, it checks that $\mac.\verify(\mk, \ct_{\sym},\tau) = 1$ and, if so, returns $\msg \gets{} \sym.\dec(\ek,\ct_{\sym})$. We note that \adv{} cannot query the challenged ciphertext. \bdvii{} returns $0$ if and only if $b=\widetilde{b}$. It easy to see that if $b_2$ is equal to $0$, then all symmetric encryption and MAC keys used for the challenge ciphertext $(\gel{r}^{*} \concat \ct_{\sym}^{*} \concat \tau^{*})$ and decryption responses are exactly as in a \dhaes{} game.

    If $b_2=1$, then $w_{1, 0}$ and $w_{1, 1}$ are random strings and the challenge ciphertext and decryption responses are given as in the $\gamestyle{G_0}$ game described in \cref{fig:g0_game}.
%
    So we get,
        \[
            \prob{\hdhii^{\bdvii{}}(\secpar) = 1} = \frac{1}{2}\cdot\prob{\ikcca^{\adv}(\secpar) = 1} + \frac{1}{2}\cdot\prob{\gamestyle{G_0}^\adv{}(\secpar) = 1}\,.
        \]
    And from the definition of \hdhii{} advantage we have

        \[
            \advantage{\hdhii}{\hashSet,\groupSetup,\bdvii{}} = \abs{\prob{\ikcca^{\adv}(\secpar)=1} + \prob{\gamestyle{G_0}^{\adv}(\secpar)=1} - 1}\,.
        \]

    At this point, we can conclude as in~\cite[Theorem 6.2]{abdalla2010robust}, with the only difference of applying~\cref{preliminaries:lemma:hdhi_hdhii} instead of~\cite[Lemma 6.1]{abdalla2010robust} and by defining a game $\gamestyle{G_1}$ that is \emph{identical until} \badvar\footnote{Games $\gamestyle{G_i}$ and $\gamestyle{G_j}$ are said to be \emph{identical until} \badvar if they differ only in statements that follow the setting of the $\badvar{}$ variable to $True$. $\badvar$ is initialized with $False$} $\gamestyle{G_0}$ defined in \cref{fig:g0_game}.
\end{proof}


% Adversary B bar for odhii
\begin{figure}
    \begin{minipage}[t]{0.5\textwidth}
        \centering
        \procedure[syntaxhighlight=auto, space=auto]{Adversary $\bdvii (\groupenc{u}, \groupenc{v_{0}}, \groupenc{v_{1}}, w_{b_2,0}, w_{b_2,1})$}{
            b \sample{} \bin{}\\
            % Running adv
            (\msg, \state) \gets \adv^{\oracle{\dec_{\sk_0}}, \oracle{\dec_{\sk_1}}} (\groupenc{v_{0}}, \groupenc{v_{1}})\\
            \ek{} \concat{} \mk \gets{} w_{b_2, b}\\
            \gel{r}^{*} \gets{} u\\
            \ct_{\sym}^{*} \gets{} \sym.\enc{(\ek, \msg)}\\
            \tau^{*}\gets{} \mac.\tagg{(\mk, \ct_{\sym}^{*})}\\
            \widetilde{b} \gets \adv^{\oracle{\dec_{\sk_0}}, \oracle{\dec_{\sk_1}}} (\gel{r}^{*} \concat \ct_{\sym}^{*} \concat \tau^{*}, \state)\\
            % bdvii guess
            \pcreturn{} \widetilde{b} = b
        }
    \end{minipage}%
    \begin{minipage}[t]{0.5\textwidth}
    \centering
        % Adv decryption simulation
        \procedure[syntaxhighlight=auto, space=auto]{\bdvii{} simulation of $\oracle{\dec_{\sk_i}} (\gel{r} \concat \ct_{\sym} \concat \tau)$}{
            \pcif{} \gel{r} \neq{} \groupenc{u}\\
            \ek{} \concat{} \mk \gets \oracleHdhi{v_i} (\gel{r})\\
            \pcelse{}\\
            \ek{} \concat{} \mk \gets{} w_{b_2, i}\\
            \pcfi{}\\
            \pcif{} \mac.\verify(\mk, \ct_{\sym}, \tau) = 1\\
            \pcreturn{} \sym.\dec(\ek, \ct_{\sym})\\
            \pcelse{}\\
            \pcreturn{\bot}\\
            \pcfi{}
        }
    \end{minipage}%
    \caption{Description of the adversary \bdvii{} for \hdhii{}, simulating \dhaes{} game for \adv{}.}\label{fig:hdhii_adv}
\end{figure}

% G_0 game
\begin{figure}
    \begin{minipage}[t]{0.5\textwidth}
    \centering
        \procedure[syntaxhighlight=auto, space=auto]{$\gamestyle{G_0} (\secpar)$}{
            (q, \gset, \ggen, +) \gets{} \groupSetup(\secparam{})\\
            (\sk_0, \pk_0), (\sk_1, \pk_1) \sample{} \kgen{(\secparam{})}\\
            \gel{r}^{*} \sample{} \gset{}\\
            \ek^{*} \sample{}\bin^{\symKeyLen}\\
            \mk^{*} \sample{}\bin^{\macKeyLen}\\
            (\msg, \state) \gets{} \adv^{\oracle{\overline{\dec}_{\sk_0}}, \oracle{\overline{\dec}_{\sk_1}}} (\pk_0, \pk_1)\\
            b \sample{} \bin{} \\
            \ct_{\sym}^{*} \gets{} {} \sym.\enc(\ek^{*}, \msg)\\
            \tau^{*} \gets{} {} \mac.\tagg(\mk^{*}, \ct_{\sym}^{*})\\
            \widetilde{b} \gets{} \adv^{\oracle{\overline{\dec}_{\sk_0}}, \oracle{\overline{\dec}_{\sk_1}}} (\gel{r}^{*} \concat \ct_{\sym}^{*} \concat \tau^{*}, \state)\\
            \pcreturn{} \widetilde{b} = b
        }
    \end{minipage}%
    \begin{minipage}[t]{0.5\textwidth}
        \centering
        \procedure[syntaxhighlight=auto, space=auto]{Oracle $\oracle{\overline{\dec}_{\sk_i}} (\gel{r} \concat \ct_{\sym} \concat \tau)$}{
            \pcif{} \gel{r} = \gel{r}^{*}\\
            \msg \gets \bot{}\\
            \pcif{} \mac.\verify(\mk^{*}, \ct_{\sym}, \tau) = 1\\
            \badvar \gets{} true\\
            \msg \gets \sym.\dec(\ek^{*}, \ct_{\sym})\\
            \pcfi{}\\
            \pcelse{}\\
            \msg \gets \dec(\sk_i, \gel{r} \concat \ct_{\sym} \concat \tau)\\
            \pcfi{}\\
            \pcreturn{} \msg
        }
    \end{minipage}%
    \caption{$\gamestyle{G_0}$ game and related decryption oracles for \cref{th:ik-cca}.}\label{fig:g0_game}
\end{figure}

\subsection{Final notes and observations}\label{instantiation:enc:final-notes}

In this section we list some notes regarding the approach taken in \zcash{} (see~\cite[Section 8.7]{zcashprotocol}), and other observations:
\begin{itemize}
    \item \emph{Key derivation parameters}: in \dhaes{} construction, the only required input variables are the shared secret \sharedSecret{} and \epk. In the Sprout release of $\zcash{}$, additional parameters were added (i.e.~$h_{sig}$, $\pk_{enc}$ and a counter $i$) (see~\cite[5.4.4.2]{zcashprotocol}): they state that $h_{sig}$ was used in order to get a different randomness extractor for each joinsplit transfer in order to limit the degradation of the security and weaken assumption on the hash. The authors believed, about the use of long-standing public key $\pk_{enc}$, that it might be necessary for \indccaii{} security and for post-quantum privacy (in the case where the quantum attacker does not have the public key)~\cite{zcashforum2019encsec}. None of these additional components are used any longer starting from the Sapling release (see~\cite[5.4.4.4]{zcashprotocol}). To the best of our knowledge there is no formal reason to use the note counter $i$ as an input to the \kdf{}: an explanation could be to avoid the same session key being reused for multiple notes, but this should not be a problem since a different nonce or block counter is used for the symmetric cipher (actually this is already mandated in the case where \epk{} is reused, as described below).
    \item \emph{Reuse of ephemeral keys} \epk: $\zcash{}$ reuses the same ephemeral keys \epk{} (and different nonces) for two ciphertexts in a joinsplit description, claiming that this does not affect the security of the scheme as soon as the \hdhi{} assumption of the \dhaes{} security proof is adapted. Note that the proof they refer to is related to the \indccaii{} notion.
    \item Note that in \zcash{} Sprout and Sapling, being able to break the Elliptic Curve Diffie-Hellman Problem on \curve{25519} or \jubjub{} would not help to decrypt the transmitted notes ciphertext unless the receiver $\pk_{enc}$ is known or guessed. On the other hand, having $\pk_{enc}$ into the hash (as used in Sprout) may violate in principle the key-privacy of the encryption scheme. For these reasons, we underline that the protocol should enforce a mechanism that does not reveal users public keys to increase the security.
    \item In~\cite{abdalla2010robust}, the concept of \emph{robustness} for an asymmetric encryption scheme is introduced: it formalizes the infeasibility of producing a ciphertext valid under two different public encryption keys. We note that this is particularly useful for \zeth{} since only the intended receiver will be able to decrypt the encrypted note. In fact, the definition is more general since it also covers the case in which a decryption is successful but returns an incorrect plaintext. This prevents situations where a user, scanning the \mixer{} logs for incoming transactions, gets a false positive decryption and stores garbage notes.
    \begin{notebox}
        We note however, that the ``false-positive'' situation above can be prevented by relying on a weaker notion of robustness called \emph{collision-freeness}~\cite{asiacrypt-2010-23840}. In fact, as described in~\cref{zeth-protocol:zeth-receive}, the procedure to receive a $\zethnote$ requires to decrypt the ciphertext emitted by the $\mixer$, and then to verify that the recovered plaintext is the opening of a commitment in the Merkle tree.
        As such, since the \emph{collision-freeness} of the encryption ensures that plaintexts recovered under different keys are different (i.e.~``do not produce a collision''), then we know that plaintexts recovered by parties who are not the intended recipient will fail the ``commitment opening verification'', leading the payment to be rejected, and solving the aforementioned false-positive issue.
    \end{notebox}
In~\cite[Section 6]{abdalla2010robust}, the authors prove that \dhies{} can be made strongly robust. The proof can be easily adapted to work with \dhaes{}.
    \item \emph{No public key validation for} \xscalarmult{25519}: cryptographers have been discussing the absence of any mandated public key validation or checks on the result of \xscalarmult{25519}. For example, in~\cite[Section 6.1]{rfc7748}, an optional zero check is introduced in order to assure that the result of \xscalarmult{25519} is not $0$: this avoids a situation in which one of the two parties can force the result of the key-exchange by using a small order point as public key. This property is generally defined as \emph{contributory behaviour}, that is, none of the parties is able to force the output of a key exchange. However, protocols do not have all the same security requirements and adding default checks in the \curve{25519} specifications would be superfluous in most cases and would add complexities that Bernstein has deliberately chosen to avoid (\emph{simple implementation principle}). More importantly, Diffie-Hellman does not require \emph{contributory behaviour} property~\cite{trevorzerocheckcritique}: modern view is that the only requirements are key indistinguishability and, in case of an active attacker, that the output of the key exchange should not produce a low-entropy function of the honest party's private key (e.g.~small-subgroup and invalid-curve attacks). Since these two properties are considered satisfied by \curve{25519}, there is no need to add extra checks to the \curve{25519} specification. We conclude by observing that in the Sprout release, the $\zcash{}$ protocol does not specify any point validation and makes use only of the private key clamping to keep Diffie-Hellman key exchange secure.
    \item \emph{Fuzzy message detection}:
\end{itemize}
 % Encryption scheme
% !TEX root = ../zeth-protocol-specification.tex

\section{$\zksnark$ instantiation}\label{instantiation:zksnark}

Groth's proof system $\groth$ \cite{groth2016size} is the most efficient known zk-SNARK (in terms of the proof size and proof and verification cost) for QAPs, and thus one of the most efficient $\nizk$ for proving statements on arithmetic circuits. Below we present $\groth$'s key generation, prover, verifier, and simulator algorithms, adjusted as described in \cite{bowe2017mpc} to further reduce the size of $\srs$ and proofs, and to make the $\kgen$ algorithm more amenable to implementation as a multi-party computation.

In what follows, let the number $\constno$ of constraints in the relation $\REL$ be fixed. Without loss of generality we consider $\constno$ to be an \emph{upper bound} on the number of constraints in the $\REL$ parameter, and assume that there exists some $\constno$-th root of unity $\omega \in \FFx{\rCURVE}$. Define $\ell_i(\xchi)$ to be the $i$-th Lagrange polynomial of degree $(\constno - 1)$ over the set $\smallset{\omega^i}_{i \in [\constno]}$, and let $\ell(\xchi)$ be the unique non-zero polynomial of degree $\constno$ that satisfies $\ell(\omega^i) = 0$ for all $i \in [\constno]$.

We note that the requirement that there exists a $\constno$-th root of unity $\omega$ imposes a restriction on the maximum number of constraints in $\REL$ that the scheme can support. In the particular case of $\omega \in \FFx{\rBN}$, the restriction becomes $\constno \leq 2^{28}$. For $\FFx{\rBLS}$ this becomes $\constno \leq 2^{47}$.

%% For future reference (if it becomes necessary to explain MPC details):
%%
%% ... where $\constno$ is a power of 2 such that there exists some $\constno$-th root of unity $\omega \in \FFx{\rBN}$.
%%
%% The use of the $\constno$-th root of unity supports FFT transforms for fast polynomial manipulation. The requirement that $\constno$ is a power-of-2 dividing $\rBN$ allows FFT transforms to be used for fast evaluation of Langrange polynomials at $\tau$, in terms of a series of encoded powers $\smallset{ \groupenci{\tau}{1} }_{i = 0}^{\constno - 1}$ (as described in Section 3 of "A multi-party protocol for constructing the public parameters of the Pinocchio zk-SNARK" - https://eprint.iacr.org/2017/602.pdf)

\begin{description}

%% KeyGen
\item[$\kgen(\REL, \secparam)$:]\hfill
  \begin{compactenum}[i.]
  \item Pick trapdoor $\td = (\tau, \alpha, \beta, \delta) \sample (\ZZ^*_p \setminus \smallset{\omega^{i - 1}}_{i = 1}^\constno) \times (\ZZ^*_p)^3$;
  \item For $j \in \range{1}{\inpno}$, let
    \begin{align*}
      u_j (\tau) & = \sum_{i = 1}^\constno U_{i j} \ell_i (\tau), \\
      v_j (\tau) & = \sum_{i = 1}^\constno V_{i j} \ell_i (\tau), \\
      w_j (\tau) & = \sum_{i = 1}^\constno W_{i j} \ell_i (\tau);
    \end{align*}
  \item Set
    \begin{align*}
      \srs_\prover \gets &
      \left(
      \begin{aligned}
        & \groupenci{\alpha}{1}
        , \groupenc{\beta}
        , \groupenc{\delta}
        , \smallset{ \groupenci{ u_j(\tau) }{1} }_{j = 1}^{\inpno}
        , \smallset{ \groupenc{ v_j(\tau) } }_{j = 0}^{\inpno}
        , \\
        & \smallset{
          \groupenci{(u_j (\tau) \beta + v_j (\tau) \alpha + w_j (\tau)) /\delta}{1}
        }_{j = \inpnoprim + 1}^\inpno
        , \\
        & \smallset{
          \groupenci{\tau^i \ell (\tau) / \delta}{1}
        }_{i = 0}^{\constno - 2}
      \end{aligned}
      \right) \\
      \srs_\verifier \gets &
      \left(
      \begin{aligned}
        & \groupenci{\alpha}{1}
        , \groupenci{\beta}{2}
        , \groupenci{\delta}{2}
        , \smallset{
          \groupenci{ \beta u_j(\tau) + \alpha v_j(\tau) +w_j }{1}
        }_{j = 0}^{\inpnoprim}
      \end{aligned}
      \right) \\
      \srs \gets & (\srs_\algostyle{P}, \srs_\algostyle{V})
    \end{align*}
  \end{compactenum}
  \pcreturn $\srs, \td$

%% Prover
\item[$\prover(\REL, \srs_\prover, \priminputs  = (\inputs_j)_{j = 1}^{\inpnoprim}, \auxinputs = (\inputs_j)_{j = \inpnoprim + 1}^{\inpno})$:]\hfill
  \begin{compactenum}[i.]
  \item
    Define
    \begin{align*}
      a^\dagger (\xchi) = \sum_{j = 1}^\inpno \inputs_j u_j (\xchi),\ \
      b^\dagger (\xchi) = \sum_{j = 1}^\inpno \inputs_j v_j (\xchi),\ \
      c^\dagger (\xchi) = \sum_{j = 1}^\inpno \inputs_j w_j (\xchi);
    \end{align*}
  \item
    Define the polynomial $\HHH (\xchi) = (a^\dagger (\xchi) b^\dagger (\xchi) - c^\dagger (\xchi)) / \ell (\xchi)$ and compute the coefficients $\smallset{ \HHH_i }_{i = 0}^{\constno - 2}$ of $\HHH$, such that $\HHH(X) = \sum_{i = 0}^{\constno - 2} \HHH_i \xchi^i$.
  \item
    $r_a \sample \ZZ_p$;
  \item
    $r_b \sample \ZZ_p$;
  \item Compute proof elements:
    \begin{align*}
      \AAA \gets & \sum_{j = 1}^\inpno \inputs_j \groupenci{u_j (\tau)}{1} + \groupenci{\alpha}{1} + r_a \groupenci{\delta}{1} \\
      \BBB \gets & \sum_{j = 1}^\inpno \inputs_j \groupenci{v_j (\tau)}{2} + \groupenci{\beta}{2} + r_b \groupenci{\delta}{2} \\
      \CCC \gets & r_b \AAA + r_a \brak{\sum_{j = 1}^\inpno \inputs_j \groupenci{v_j (\tau)}{1} + \groupenci{\beta}{1}} + \\
      & \sum_{j = \inpnoprim + 1}^\inpno \inputs_j \groupenci{\frac{u_j (\tau) \beta + v_j (\tau) \alpha + w_j (\tau)}{\delta}}{1} + \\
      & \sum_{i =0}^{\constno - 2} \HHH_i \groupenci{\tau^i \ell (\tau) / \delta}{1}
    \end{align*}
  \end{compactenum}
  \pcreturn $\pi \gets (\AAA, \BBB, \CCC)$;

%% Verifier
\item[$\verifier (\REL, \srs_\verifier, \priminputs = (\inputs_j)_{j = 1}^{\inpnoprim}, \pi)$:]\hfill
  \begin{compactenum}[i.]
  \item Check that:
    \begin{align*}
      \AAA \pair \BBB = & \CCC \pair \groupenci{\delta}{2} \\
      & + \brak{\sum_{j = 1}^{\inpnoprim} \inputs_j  \groupenci{u_j (\tau) \beta + v_j (\tau) \alpha + w_j (\tau)}{1}} \pair \groupenci{1}{2} \\
      & + \groupenci{\alpha}{1} \pair \groupenci{\beta}{2}
    \end{align*}
  \end{compactenum}
  Note that $\groupenci{\alpha}{1}$ and $\groupenci{\beta}{2}$ are stored individually and used by the prover to recompute $\groupenci{\alpha \beta}{T}$ seemingly redundantly. This is required in order to leverage the pairing check functionality built in to \ethereum, which accepts a sequence of tuples in $\gset_1 \times \gset_2$ and returns $\true$ if and only if the product of the resulting pairings equals $\groupenci{1}{T}$.

%% Simulator
\item[$\simulator(\REL, \srs, \td, \priminputs)$:]\hfill
  \begin{compactenum}[i.]
  \item Sample $\AAA^* \sample \ZZ_p$; $\BBB^* \sample \ZZ_p$;
  \item Compute proof elements:
    \begin{align*}
      & \AAA \gets \groupenci{\AAA^*}{1} + \groupenci{\alpha}{1} \\
      & \BBB \gets \groupenci{\BBB^*}{1} + \groupenci{\beta}{2} \\
      & \CCC \gets \frac{1}{\delta} \cdot
      \begin{aligned}[t]
        & \bigg[ \AAA^* \BBB^* \groupenci{1}{1} + \AAA^* \groupenci{\beta}{1} + \BBB^* \groupenci{\alpha}{1} \\
        & \ \ - \sum_{j = 1}^{\inpnoprim} \inputs_j \groupenci{ u_j (\tau) \beta + v_j (\tau) \alpha + w_j (\tau)}{1} \bigg]
      \end{aligned}
    \end{align*}
  \end{compactenum}
  \pcreturn $\pi \gets (\AAA, \BBB, \CCC)$;
\end{description}
 % ZkSNARK
