% !TEX root = ../zeth-protocol-specification.tex

\section{Generating the inputs of the $\mix$ function (\inp{\mix})}\label{zeth-protocol:mix-inp}

In the following section, we assume that the system is initialized. In other words, we assume that a ledger $\ledger$ is available (i.e.~an \ethereum~network is operated by a set of miners), the $\mixer$ contract is deployed on $\ledger$. Likewise, we assume that the public parameters $\pparams_{\zksnark} \gets \zksnark.\kgen(\secparam, \RELCIRC)$ are available to $\mixer$ and to all parties willing to call the $\mix$ function of $\mixer$. Furthermore, we assume that there exists as set of \ethereum~and \zeth~users, and that the \emph{payment address} of each \zeth~user is easily discoverable. In the rest of this section, the set of \emph{payment addresses} discovered by a zeth user $\zparty{U}$ is represented as a list attribute $\zparty{U}.\keystore$ indexed by usernames.

\medskip

In order for $\zparty{U}$ to transact via \zeth, $\zparty{U}$ needs to create an object $\inp{\mix}$ of type $\mixInputDType$ in order to call the $\mix$ function of $\mixer$:
\begin{enumerate}
    \item Create an object $\priminputs$ of type $\primInputDType$ to represent the primary input, and an object $\auxinputs$ of type $\auxInputDType$ to represent the auxiliary input, where:
    \begin{enumerate}
        \item $\priminputs.\mkroot \in \rootset$, where $\rootset$ is the set of \emph{all} Merkle roots corresponding to one of the state of the Merkle tree on $\mixer$ containing \emph{all} the commitments to the input notes, in $\auxinputs.\jsins{}$, in its set of leaves.
        \item $\auxinputs.\znotes{j}.\noter \sample \BB^{\noterLen}, \forall j \in [\jsout]$, and $\auxinputs.\pphi \sample \BB^{\phiLen}$
        \item The public values $(\auxinputs.\vin, \auxinputs.\vout) \in {(\BB^{\zvalueLen})}^{2}$, $\auxinputs.\znotes{j}.\notev$ and $\auxinputs.\znotes{j}.\apk\ \forall j \in [\jsout]$ are all set by the sender, $\zparty{U}$, as desired as long as they satisfy the \gls{joinsplit-eq}.
        \item All attributes of the $\priminputs$ and $\auxinputs$ objects should be derived as specified in the statement (see:~\cref{zeth-protocol:statement}) \secfix{alongside a signature hash ($\auxinputs.\hsig$) that is generated as the hash of the nullifiers and a one-time signing verification key (malleability fix, see:~\cref{appendix:trnm}), using the desired signature scheme $\otsigscheme$ (see:~\cref{instantiation:otsig}):}
            \begin{align}
                & (\sk_{\otsig}, \vk_{\otsig}) = \otsigscheme.\kgen(\secparam) \label{zeth-protocol:otsig-generation} \\
                & \auxinputs.\hsig = \crhhsig{\indexedset{\auxinputs.\jsins{i}.\nf{}}{i \in [\jsin]}, \vk_{\otsig}}
            \end{align}
        \item $\inp{\mix}.\primInp \gets \priminputs$
    \end{enumerate}
    \begin{notebox}
        If one of the attributes of $\priminputs$ and $\auxinputs$ is not correctly generated, then the proof of computational integrity generated in the next step will be rejected on $\mixer$, and the state of $\mixer$ will not be modified.
    \end{notebox}
\item Generate a zk-snark \zkp~to prove, in zero-knowledge, that the relation $\RELCIRC$ (\cref{zeth-protocol:statement}) holds on the primary and auxiliary inputs, using the desired zk-snark scheme $\zksnark$ (see:~\cref{instantiation:zksnark}):
    \begin{enumerate}
        \item
            \[
                \zkp \gets \zksnark.\prover(\pparams_{\zksnark}, \priminputs, \auxinputs)
            \]
        \item $\inp{\mix}.\zkproof \gets \zkp$
    \end{enumerate}

\item Encrypt all the $\auxinputs.\znotes{}$ using the recipient's \emph{payment address}, using the encryption scheme $\encscheme$ (see:~\cref{instantiation:enc}).
    \begin{enumerate}
        \item For all $j \in [\jsout]$, do:
        \begin{align*}
            & \ct_j = \encscheme.\enc(\auxinputs.\znotes{j}, \zparty{U}.\keystore[\varstyle{recipient}_j].\pubaddr.\pkenc)
        \end{align*}
        \item $\inp{\mix}.\ciphers \gets \indexedset{\ct_j}{j \in [\jsout]}$
    \end{enumerate}
\item \secfix{Generate a signature $\otsSigma$ on the inputs of the \mix~function, in order to prevent any malleability attacks~(c.f.~\cref{appendix:trnm}), using the desired signature scheme $\otsigscheme$ (see:~\cref{instantiation:otsig}):}
    \begin{enumerate}
        \item The one-time signature keypair has already been generated in~\cref{zeth-protocol:otsig-generation}.
            \begin{align*}
                & \datatobesigned = \eparty{S}.\addr \concat \inp{\mix}.\primInp \concat \inp{\mix}.\zkp \concat \inp{\mix}.\ciphers \\
                & \otsSigma = \otsigscheme.\sig(\sk_{\otsig}, \crhots{\datatobesigned})
            \end{align*}
        \item $\inp{\mix}.\otssig \gets \otsSigma$
        \item $\inp{\mix}.\otsvk \gets \vk_{\otsig}$
    \end{enumerate}
    Here, $\eparty{S}.\addr$ represents the address of the Ethereum user $\eparty{S}$ who must sign the transaction (see:~\cref{zeth-protocol:create-tx}). In general, this is likely to be owned by the holder $\zparty{U}$ of the Zeth notes to be spent, but this is not a requirement.
\end{enumerate}
